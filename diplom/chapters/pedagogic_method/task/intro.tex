Основой обучения является последовательное усвоение и обсуждение образовательных материалов с их закреплением путем самостоятельной работы.
Как правило, практика организована в виде решения задач с четко заданной целью, интуитивно понятной проблемой
и сроком выполнения. При составлении образовательной задачи необходимо учитывать, что она должна:
\begin{itemize}
    \item предоставлять возможность для развития критического мышления и применения знаний на практике;
    \item быть структурированной;
    \item иметь ясные и объективные критерии оценки.
\end{itemize}
По ответу задачи разделяются на открытые, предполагающие развернутый ответ с демонстрацией приобретенных знаний, 
и закрытые, подразумевающие лишь выбор правильного ответа. Современное образование активно использует оба подхода, 
объединяя их в единую контрольно-методическую работу, представляющую многогранную систему проверки знаний с заданными
критериями оценки. Для анализа результатов прохождения работы используются статистические методы оценки знания, учитывающие случайность в измерениях 
\cite{brennan2006educational}. Специалисты предполагают, что знание подлежит лишь косвенному измерению, путем проведения тестирования или 
конкурсного отбора. Для учета погрешности в измерениях этих инструментов теория использует модели заданий и умений обучающихся.
Наиболее совершенной теорией на данный момент считается теория отклика на задание (от \textit{англ.} Item Response Theory).
