
Количественная оценка успеваемости выполняется на основании контрольных работ. Решение об оценки может выноситься как учителем, так и автоматически с использованием приложения. Оценка при обучении выполняют множество задач:
- обучающегося
- оценка перспектив его родителей
- возможность сравнить различные подходы к изложению материала для педагога 


\texit{Определени} \textbf{Психометрия}- дисциплина, изучающая количественные способы оценки знания.


Для этого используются статистические методы оценки знания, учитывающий случайность в измерениях.
Педполагается, что знание не подлежит явному измерению,
а лишь косвенному путем проведения тестирования или анализа деятельности обучающегося. Для построения теории вводятся скрытые от наблюдения переменные называемыми латентными $\theta$. 

Теория тестов активно развивается. Основными подходами к оценке знаний, исходя из результатов, 
на текущий момент является классическая и (item response) теория.

Классическая теория предполагает, что результат тестирования задан случайной величиной. Её вид вид как правило предполагается нормальным:

$$ 
    s \sim \mathrm{N}(\theta,\sigma^2)
$$

,где $s$ задает экзаменационный результат обучающегося, параметр $\theta$ - истинный уровень знания, $\sigma^2$ - задает волатильность измерений. распределения с дисперсией, определяемой 

Как правило тесты подбирают таким образом, чтобы ошибка метода на всем промежутке результатов была минимальна

$$
    \int_0^1 \sigma^2(\theta) \rightarrow \min
$$


Существенным недостатком такой системы является предположение о равной сложности задач в контрольной работе.


Система тестирования IRT была предложена институтом [ETS](https://en.wikipedia.org/wiki/Educational_Testing_Service ) в 1950 году.

Основным предположением системы

Наиболее известным результатом системы является 3-х параметрическая логистическая модель, учитывающая сложность задачи, вероятность угадать и волатильность оценки.

$$
    p_i(\theta) = c_i + \frac{1-c_i}{1+e^{-a_i(\theta-b_i)}}
$$


Альтернативным подходом к оценки является  \cite{corbett1994knowledge}


Наличие **банка задач** и **алгоритма** подбора задач по уровню.

Как правило формат оценки

Выделяют три основных подхода:
- адаптивное тестирование [CAT](https://en.wikipedia.org/wiki/Computerized_adaptive_testing).
 В этом случае задания подбираются динамически во время тестирования
- классификационное тестирование[CCT](https://en.wikipedia.org/wiki/Computerized_classification_test)








https://aclanthology.org/L18-1644.pdf



Тестирование также может быть формой обучающего процесса. В этом случае задания подбираются исходя из 

:::note


:::


Наиболее известными примерами исследований является [FACT](https://en.wikipedia.org/wiki/Frankfurt_Adaptive_Concentration_Test)подготовленный [Гётте институтом](https://en.wikipedia.org/wiki/Goethe_University_Frankfurt)  
