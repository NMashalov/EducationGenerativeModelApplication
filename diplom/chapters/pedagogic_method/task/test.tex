В разделе описаны основные направление электронного образования. Данное направление чрезвычайно популярно и имеет важную экономическую роль.
\begin{itemize}
    \item классическое тестирование;
    \item классификационное тестирование[CCT];
    \item адаптивное тестирование [CAT];
    В этом случае задания подбираются динамически во время тестирования.
\end{itemize}


Платформы обучения языкам(ICT) 
\begin{itemize}
    \item Revita \cite{katinskaia2018revita}
\end{itemize}

\begin{figure}[h]
    \centering
    \includegraphics[width=0.5\textwidth]{assets/pedagogic/social/its.excalidraw.png}
    \caption{Схема организации обучения.}
    \label{bkt}
\end{figure}


\textit{Определение:} \textbf{Психометрия} --- дисциплина, изучающая количественные способы оценки знания.

Для этого используются статистические методы оценки знания, учитывающие случайность в измерениях. Предполагается, что знание не подлежит явному измерению,
а лишь косвенному, путем проведения тестирования или анализа деятельности обучаемого.

Для построения теории вводятся скрытые от наблюдения переменные называемыми латентными $\theta$. 
В области образования психомтерические исследования активно выполняются для тестовых заданий. Основными подходами к оценке знаний, исходя из результатов, 
на текущий момент является классическая и (item response) теория.

Классическая теория предполагает, что результат тестирования задан случайной величиной. Её вид вид, как правило, предполагается нормальным:

\begin{equation}
    s \sim \mathrm{N}(\theta,\sigma^2), 
\end{equation}
где $s$ задает экзаменационный результат обучаемого, параметр $\theta$ --- истинный уровень знания, 
$\sigma^2$ --- задает волатильность измерений. распределения с дисперсией, определяемой 

Как правило тесты подбирают таким образом, чтобы ошибка метода на множестве результатов была минимальна
$$
    \int_0^1 \sigma^2(\theta) \rightarrow \min
$$
Существенным недостатком такой системы является предположение о равной сложности задач в контрольной работе.

\begin{figure}[h]
    \centering
    \includegraphics[width=0.5\textwidth]{assets/pedagogic/social/irt.excalidraw.png}
    \caption{Матрица исходов модели Байесовской оценки на шаге t.}
    \label{irt_function}
\end{figure}

Система тестирования IRT была предложена институтом в 1950 году. Она активно используется в международных экзаменах языка
и делового знания GMAT и TOEFL.В отличие от классической теории система также учитывает текущий уровень знаний, что позволяет составлять набор заданий индивидуально.
Наиболее известным результатом системы является 3-х параметрическая логистическая модель \ref{irt_function}, учитывающая сложность задачи, вероятность угадать правильный ответ и волатильность оценки.
Одним из ключевых \cite{lord1956measurement}:
\begin{equation}
    p_i(\theta) = c_i + \frac{1-c_i}{1+e^{-a_i(\theta-b_i)}},
\end{equation}
где \begin{itemize}
    \item $b_i$ --- сложность задания;
    \item $a_i$ --- характерный масштаб;
    \item $c$ --- вероятность угадать решение.
\end{itemize}

\begin{figure}[h]
    \centering
    \includegraphics[width=0.5\textwidth]{assets/pedagogic/social/bkt.excalidraw.png}
    \caption{Эволюция представлений о знаниях учащегося.}
    \label{bkt}
\end{figure}
Альтернативным путем является подход байесовской оценки знания, описанный в работе \cite{corbett1994knowledge}.
Модель учитывает вероятность ошибки и вероятность ошибиться при наличии знания \ref{bkt}: 
\begin{itemize}
    \item $P(L_0)$ --- начальные знания в предмете;
    \item $P(S) = P(x=0| L_t = 1)$ --- вероятность просчета при наличи знаний;
    \item  $P(G) = P(x=1| L_t = 1)$ --- вероятность угадать при отсутствии знаний
\end{itemize}
Обновление представлений выполняется через Байесов подход согласно c правилами:
\begin{equation}
    \begin{aligned}
        &P(L_t| obs_t=1) = \frac{P(L_t)(1-P(S))}{P(L_t)(1-P(S)) + (1-P(L_t))P(G)} \\
        &P(L_t| obs_t=0) = \frac{P(L_t)P(S)}{P(L_t) P(S) + (1-P(L_t))(1-P(G))}
    \end{aligned}
\end{equation}
Отметим, что полученный вывод предполагает, что:
\begin{itemize}
    \item вероятность забыть знание равна нулю $ P(L_{t+1}=0|L_t=1)=0$
    \item $P(L_{t+1}) = P(L_t|obs_t) + \left(1 - P(L_t | obs_t)\right) P(T)$
\end{itemize}
\begin{figure}[h]
    \centering
    \includegraphics[width=0.5\textwidth]{assets/pedagogic/social/bkt_automata.excalidraw.png}
    \caption{Матрица исходов модели Байесовской оценки на шаге t.}
    \label{bkt_automata}
\end{figure}
Таким, образом тест можно представить в виде марковской цепи обновления представлений о знаниях учащегося \ref{bkt_automata}.

Адаптация для случая IRT \ref{IRT} позволяет учесть влияние сложности задания \cite{bulut2023introduction}:
\begin{equation}
    P(Y_{ij}=1| \theta_J, a_i,b_i,c_i) =
\end{equation}


