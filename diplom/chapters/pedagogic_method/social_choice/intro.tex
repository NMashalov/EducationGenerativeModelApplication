Введение изменений в методику преподавания требует
обосновании, выраженного в положительном изменении общественного блага.

В литературе популярен аналитический подход к расчету социального блага через функции утилитарности.
В основе направления лежит идея, что общественное благо может
быть определено путем суммирования утилитарных предпочтений индивидов в обществе.
Этот подход связан с концепцией утилитаризма, которая утверждает, 
что действия должны быть оценены исходя из их способности максимизировать общую полезность или счастье всех членов общества.

Известны чисто аналитические подходы к оценки общественного блага при планировании выборов,
 проектировании транспортных систем и сфере медицины.
