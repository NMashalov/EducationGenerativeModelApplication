Награды могут иметь разное происхождение, предназначение и формы. 
Например, в российском школьном образовании отличная учеба награждается медалями, а физкультурно-спортивные достижения сопровождаются знаками отличия.
Награды поощряют вовлеченную учебу и совокупно позволяют выделять талантливых учащихся. 

Известные компании, предоставляющие инструменты разработки, как Github \footnote{\url{https://cloud.google.com/}}
и Google Сloud \footnote{\url{https://cloud.google.com/}}, также используют систему знаков - \textit{бэйджей} 
для поощрения пользователей ресурса к эффективному использованию ресурсов платформы и прохождению сертификации. 
Предполагается, что наличие таких наград может стать причиной для делового сотрудничества или предложения работы. Другим примером является платформа [Stack Overflow](https://stackoverflow.com/),
на которой публикуются вопросы преимущественно технической тематики.
Согласно ее правилам бейдж присваивается в случае наличие лучшего ответа на популярный вопрос.
Такая награда поощряет положительную конкуренцию, поощряет к созданию конструктивных и этичных ответов \cite{yanovsky2021one}.
\begin{figure}[h]
    \centering
    \includegraphics[width=0.5\textwidth]{assets/pedagogic/social/badge.excalidraw.png}
    \caption{Упорядоченная система наград}
    \label{badge}
\end{figure}

Социальный статус тесно связан с теорией общественного консенсуса \cite{anderson2015desire}\cite{ridgeway2006consensus}. 

\textit{Определение}\textbf{Консенсус} является результатом достижения согласованного состояния между несколькими независимыми 
процессами или узлами в системе, которые могут взаимодействовать друг с другом. 

Для достижения консесуса необходимо выполнить условия \begin{enumerate}
    \item корректности: \forall i \in \{1, \ldots, n\}, \text{если } \text{input}(N_i) = v, \text{ то } \forall j \in \{1, \ldots, n\}, \text{output}(N_j) = v. Все узлы начинают с одним и тем же начальным значением v, то любое значение, принятое в результате выполнения протокола консенсуса, должно быть равно \( v \).
    \item единогласие: \forall i, j \in \{1, \ldots, n\}, \text{если } \text{output}(N_i) = v, \text{ то } \text{output}(N_j) = v. Если один узел завершает протокол с некоторым значением \( v \), то все другие узлы, которые также завершили протокол, должны иметь то же самое значение \( v \).
    \item завершение:\forall i \in \{1, \ldots, n\}, \text{ узел } N_i \text{ завершает выполнение протокола в конечное время}.
    \item 
\end{enumerate}

Наиболее полулярными алгоритмами достижения конcенсуса являются Raft \cite{lamport2019time} и Paxos \cite{pease1980reaching}

Теория бейджей основана на наличие системы упорядоченных наград \cite{Easley2013}. Опишем постановку для линейного
упорядоченного набора бейджей $m > m-1 > \dots > 0$ и $n$  пользователей системы.

Тогда Механизм бейджа задается как функция $r$ : $\mathrm{R}^+ \times \mathrm{R^+}^{n-1} \righarrow \left{1, \dots,m \right}$ 
от вкладов игроков $b_i$.

Каждый пользователь делает вклад $b_i$

$$
    S(t_i) - \frac{b_i}{v_i}
$$



В этом случае вводят функцию социального статуса $t$, задающаяся
отношением числа пользователей, имеющих бейдж равный или лучше.

$$

$$
