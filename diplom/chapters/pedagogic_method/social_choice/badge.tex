Награды могут иметь разное происхождение, предназначение и формы. 
Например, в российском школьном образовании отличная учеба награждается медалями, а физкультурно-спортивные достижения сопровождаются знаками отличия.
Награды поощряют вовлеченную учебу и совокупно позволяют выделять талантливых учащихся. 

Известные компании, предоставляющие инструменты разработки, как Github \footnote{\url{https://cloud.google.com/}}
и Google Сloud \footnote{\url{https://cloud.google.com/}}, также используют систему знаков - \textit{бэйджей} 
для поощрения пользователей ресурса к эффективному использованию ресурсов платформы и прохождению сертификации. 
Предполагается, что наличие таких наград может стать причиной для делового сотрудничества или предложения работы. Другим примером является платформа [Stack Overflow](https://stackoverflow.com/),
на которой публикуются вопросы преимущественно технической тематики.
Согласно ее правилам бейдж присваивается в случае наличие лучшего ответа на популярный вопрос.
Такая награда поощряет положительную конкуренцию, поощряет к созданию конструктивных и этичных ответов \cite{yanovsky2021one}.
\begin{figure}[h]
    \centering
    \includegraphics[width=0.5\textwidth]{assets/pedagogic/social/badge.excalidraw.png}
    \caption{Упорядоченная система наград}
    \label{badge}
\end{figure}
Теория бейджей основана на наличие системы упорядоченных наград \cite{Easley2013}. Опишем постановку для линейного
упорядоченного набора бейджей $m > m-1 > \dots > 0$ и $n$  пользователей системы.

\textit{Определение} Механизм бейджа задается как функция $r$ : $\mathrm{R}_{+} \times \mathrm{R}_{+}^{n-1} \rightarrow \left\{1, \dots,m \right\}$ 
от вкладов игроков $b_i$:
\begin{equation}
    S(t_i) - \frac{b_i}{v_i}
\end{equation}

Задача механизма обеспечить максимазиця суммарного вклада участников:
\begin{equation}
    \max_{P(m)} \sum_{i=1}^n b_i
\end{equation}

\textit{Определение} \textbf{Механизмом отсечки по порогу} называется механизм распределения бейджа $j \in \left\{0,\dots,m\right\}$, исходя
из преодоления линейно упорядоченных порогов $\mathbf{\theta} =(\theta_1,\dots,\theta_m)$, таким образом, что $b_i \in \left[\theta_j,\theta_{j+1}\right)$.

\textit{Определение} \textbf{Механизм таблицы лидеров} называется механизм распределения бейджей в порядке убывания по вкладу $b_i$.






