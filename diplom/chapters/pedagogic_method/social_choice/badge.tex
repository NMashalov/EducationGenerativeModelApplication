Награды могут иметь разное происхождение, предназначение и формы. 
Например, в российском школьном образовании отличная учеба награждается медалями, 
а физкультурно-спортивные достижения отмечаются знаками отличия. 
Награды поощряют вовлеченную учебу и совокупно позволяют выделять талантливых учащихся. 

Известные компании, предоставляющие инструменты разработки, такие как GitHub\footnote{\url{https://github.com/}}
и Google Сloud\footnote{\url{https://cloud.google.com/}}, также используют систему наград в виде знаков --- \textit{бэйджей}
для поощрения пользователей к эффективному использованию ресурсов платформы и прохождению сертификации. 
Предполагается, что наличие таких наград может стать причиной для делового сотрудничества или предложения работы. 
Другим примером является платформа Stack Overflow\footnote{\url{https://stackoverflow.com/}}, на которой обсуждаются вопросы, преимущественно, технической тематики.
Согласно ее правилам бейдж присваивается за наилучший ответ на популярный вопрос. Такая награда поощряет положительную конкуренцию, и мотивирует 
конструктивные и этичные ответы \cite{yanovsky2021one}.
\begin{figure}[h]
    \centering
    \includegraphics[width=0.5\textwidth]{assets/pedagogic/social/badge.excalidraw.png}
    \caption{Упорядоченная система наград}
    \label{badge}
\end{figure}

Теория бейджей основана на системе упорядоченных наград. Опишем постановку для линейного
упорядоченного набора бейджей $m > m-1 > \dots > 0$ и $n$ пользователей системы c оценкой полезности бейджа $S(t)$. 

\textit{Определение:} Механизм бейджа задается как функция $r$ : $\mathrm{R}_{+} \times \mathrm{R}_{+}^{n-1} \rightarrow \left\{1, \dots,m \right\}$ 
от вкладов игроков $b_i$:
\begin{equation}
    S(t_i) - \frac{b_i}{v_i}.
\end{equation}

Задача механизма обеспечить максимизацию суммарного вклада участников:
\begin{equation}
    \max_{P(m)} \sum_{i=1}^n b_i.
\end{equation}

\textit{Определение:} \textbf{Механизмом отсечки по порогу} называется механизм распределения
бейджа $j \in \left\{0,\dots,m\right\}$, исходя из преодоления линейно упорядоченных порогов
$\mathbf{\theta} =(\theta_1,\dots,\theta_m)$, таким образом, что $b_i \in \left[\theta_j,\theta_{j+1}\right)$.

Механизм порогов прост в использовании и понимании для агентов. Предварительное задание уровня награды позволяет
планировать путь к достижению, сосредотачиваясь на индивидуальном развитии. Тем не менее 
такой подход не поддерживает соревновательный дух, потому не всегда мотивирует кооперацию среди агентов.
Альтернативой является задание рейтинга согласно порядку вклада:

\textit{Определение:} \textbf{Механизмом таблицы лидеров} называется механизм распределения бейджей  $j \in \left\{0,\dots,m\right\}$,
в порядке убывания по вкладу $b_i$. Число бейджей и участников должно быть одинаково.

Такой подход естественно ограничивает кооперацию: агенты стремятся внести больший вклад чем прочие, максимизируя функцию утилиты.
Тем не менее такой подход стимулирует лишь относительный рост, что может быть недостаточным в задачах, требующих постоянной отдачи от участников. 
Также слишком большое число бейджей неудобно для разработки на практике. 

Работа \cite{Easley2013} предлагает объединение двух подходов для разрешения перечисленных проблем.

\textit{Определение:} \textbf{Механизмом таблицы лидеров с порогом отсечки} называется механизм распределения бейджей
в порядке убывания по вкладу $b_i$ при условии $b_i > \theta$, где $\theta$ --- уровень отсечки.

Такая постановка одновременно стимулирует относительный и абсолютный рост. Единый порог $\theta$ вводится исходя из простоты
анализа постановки. 

\textit{\textbf{Теорема:} Об оптимальном механизме распределения наград} \cite{Easley2013} Оптимальный порог отсечки $\theta$ для таблицы лидеров с порогом отсечки задается как $\theta=v(k^*) S_n(k^*)$,
где $S_n(q_i) = \sum_{\nu=0}^{n-1} S\left(\frac{\nu}{n-1}\right) \beta_{\nu,n-1} (q_I)$ --- $n$ полином Бернштейна и $k^*$ --- пороговый 
квантиль полезности, заданный уравнением $b(k^*) = v(k^*) S_n(k^*)$. Заданный порог обеспечивает уникальное равновесие Байеса-Нэша
с наибольшей суммой вкладов $\sum_i b_i$.