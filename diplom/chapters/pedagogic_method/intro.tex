Создание и передача научных знаний --- основа общественного развития \cite{bell2019coming}.
Повсеместное распространение знания в профессиональных коллективах позволяет вести эффективную совместную деятельность,
автоматическим образом выполнять рутинную работу и системно оценивать риски новых проектов. Одной из важных проблем
передачи знаний является их объективная оценка. Современный путь разрешения проблемы заключается в количественной
оценке знания с применением контрольно-методических материалов. Анализ и совершенствование систем оценки знаний изучается
дисциплиной психометрии. Специалисты в этой области создают и апробируют измерительные инструменты, включающие опросники, 
тесты и методики описания личности. Также изучаются модели отклика на контрольно-методические и образовательные материалы.
Направление включает в себя три основные исследовательские задачи:
\begin{itemize}
    \item cоздание инструментов и построение процедур измерения;
    \item cоздание новых математических моделей вероятности наблюдения определенных элементов поведения;
    \item развитие и усовершенствование теоретических подходов к измерению.
\end{itemize}

В главе описаны современные подходы к созданию систем тестирования. Также представлен раздел с 
изложением общих положений из психологических теорий Льва Выгосткого, Жана Пиаже и Михаи Чиксентмихайи, задающие 
порядок воспитательно-образовательного процесса и рекомендации по выбору задач оптимального уровня сложности.

