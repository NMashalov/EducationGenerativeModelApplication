Современные образовательные материалы и инструменты требуют разработки интерфейсов. 
Одним из ключевых направлений является гештальт-дизайн \cite{wertheimer1938laws}, описанный в работах Гештальт группы.

\texit{Определение} \textbf{Гештальт} - психологическая концепция,акцентирующая внимание на восприятии целостных структур, а не на отдельных элементах, из которых они состоят.




Макса Вертхаймера, Курта Коффки и Вольфганга Кёлера \сite{}. Теория описывает основные принципы общечеловеческого визуального восприятия: 
\begin{enumerate}
    \item близость -стимулы, расположенные рядом, имеют тенденцию восприниматься вместе
    \item схожесть - стимулы, схожие по размеру, очертаниям, цвету или форме, имеют тенденцию восприниматься вместе
    \item целостность -восприятие имеет тенденцию к упрощению и целостности
    \item замкнутость - отражает тенденцию завершать фигуру так, что она приобретает полную форму
    \item смежность - близость стимулов во времени и пространстве. Смежность может предопределять восприятие, когда одно событие вызывает другое),
    \item общая зона - принципы гештальта формируют наше повседневное восприятие наравне с научением и прошлым опытом. Предвосхищающие мысли и ожидания также активно руководят нашей интерпретацией ощущений).
\end{enumerate}


