Дидактический педагогический метод наиболее распространенный метод обучения.
Такие методы задают практики организации учебно-познавательной деятельности 
учащихся в рамках педагогической системы. Ключевым инструментом такого подхода является 
количественная оценка успеваемости выполняется на основании контрольных работ. Решение об оценки может выноситься как учителем,
 так и автоматически с использованием приложения. Оценка выполняет множество задач: \begin{itemize}
    \item мотивирует обучающегося к познанию
    \item определяет перспективные направления для дальнейшего обучения для родителей
    \item позволяет сравнить различные подходы к изложению материала для педагога 
\end{itemize}

Известный психолог педагог Бенжамин Блум в работе \cite{bloom1984} изучает проблемы массового образования.
Согласно его исследованию обучающиеся по программам индивидуального образования имели результаты статистически лучше
посещающих общеобразовательные учреждения. Автор, опираясь на статистический анализ практики
своих ассистентов предлагает практики для общего образования, демонстрировавшие наибольшую эффективность \ref{bloom_table}

\begin{table}
    \centering
    \begin{tabular}{|c |c| c |}
        Объект изменения &	Название практики &  Результат\\
        \hline
        Учитель & Курсы повышения специализации &	2.00 \\	
        Учитель	& Обеспечение	& 1.2	\\	 
        Ученик	& Пороговый балл прохождения курса & 1.00	\\	
        Учитель	& Предоставление частичного решения, подсказки	& 1.00	\\	
        Учитель, Ученик	& Вовлеченность в занятие & 1.00	\\	
        Ученик	& Задание строго время выполнения задания &	1.00	\\
        Ученик & Упор на навыки письма и чтения & 	1.00	\\
        Домашнее окружение & Обучение с родителями &	0.80 \\	
        Учитель	& Балловая оценка домашнего задания	&0.80	\\
        Учитель	& Поддержание духа класса &	0.60	\\
        Ученик	& Отбор учащихся на основании интеллектуальных способностей	 &0.60	\\
        Домашнее окружение & Изменение домашней обстановки & 0.50 \\
        hline	
    \end{tabular}
    \caption{Влияние образовательных практик согласно \cite{bloom1984}}
    \label{bloom_table}
\end{table}

Автор предлагает дидактические практики, ставя акцент при обучении на контроле за успеваемостью учащихся.
В таком подходе  регулярно проводятся контрольные занятия, представляющие количественную оценку знаний и навыков.
Для этого занятия сопровождаются подготовленной средой: учебником, рабочей тетрадью и лабораторной установкой.

Противопоставленным дидактическому методу обучения является проблемно-ориентированный подход.
Эмпирический подход ставит акцентируется на активном участии обучающегося в исследовании
и поиске знаний через решение проблем и практическую деятельность.
Этот подход поддерживает самостоятельное мышление, исследовательские навыки и обучение через самостоятельное осмысление опыта.
Эмпирический подход часто ассоциируется с \textit{активными методами обучения}, описанными в работах Кругликова. \cite{кругликов2006деловые}
В таком подходе учащиеся самостоятельно формулируют и анализируют проблемы, разрабатывая стратегии исследования.
К  методам обучения относится регулярное решение актуальных постановок, пришедших из индивидуальной практики, организация исследовательских проектов и
акцент на командной работе.

