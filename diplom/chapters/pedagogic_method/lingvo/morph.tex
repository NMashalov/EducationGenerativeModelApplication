

\textit{Определение} \textbf{Морфологическим анализом} называют процесс разложения 
слова $w$ на его морфемы, например, префикс P, корень R и суффикс S, из словаря $S$

\textit{Определение} \textbf{Морфологическим синтезом} называется функция $f$,
формирующая слова $w$ из леммы $l$ и морфологических характеристик $m$. 
Примерами морфологических характеристик являются число, род, падеж.
 
\textit{Определение} \textbf{Правилами словообразования} называют ограничения,
 определяющие трансформации между леммами и словоформами. 
 Пусть \( T \) — множество таких правил. Каждое правило \( t \in T \) можно представить в виде
 $t: (l, m) \rightarrow w$

\textit{Определение} \textbf{Лексикон} называется декартово произведение лемм 
из словаря $\Sigma$ и возможных атрибутов $A$. $L = \Sigma \times A$
 

\textit{Определение} \textbf{Формальная грамматика} задается четверкой
$ G  = \left(N, \Sigma, R, S\right)$, где $N$ — множество нетерминальных символов, 
$\Sigma$ — множество терминальных символов, 
$R$ — множество правил, 
$S$ — стартовый символ.
 
 
 Эти формальные выражения и определения дают основу для создания и анализа алгоритмов в вычислительной морфологии, обеспечивая строгую и систематическую обработку естественного языка.
