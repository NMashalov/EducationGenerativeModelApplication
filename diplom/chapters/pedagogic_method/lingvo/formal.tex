Формальные языки широко используются в математике, логике, лингвистике и компьютерных науках. 
В программировании, например, формальные языки включают языки программирования и описания данных, 
где синтаксис строго определён для обеспечения корректности и предсказуемости выполнения программ.

Введем ключевые предметные определения, позволяющие формализовать анализ и 
синтез предложений в естественном языке, что важно для многих приложений 
в области обработки естественного языка и вычислительной лингвистики.
 
\textit{Определение} Формального языка является совокупностью:
\begin{itemize}
    \item \textit{алфавита} - конечного множества символов, из которых строятся строки языка.
    \item \textit{строки} - последовательности символов из алфавита, которые принадлежат языку.
    \item \textit{грамматики} - набора правил, которые определяют, какие строки являются допустимыми в языке.
\end{itemize}

\textit{Определение} \textbf{Формальный язык} — строго определённое множество строк $S$, составленных из конечного алфавита символов $\Omega$,
 которые подчиняются определённым правилам или грамматике.
Эти правила определяют синтаксис языка, то есть допустимые последовательности символов,
 и часто могут быть представлены с помощью формальных грамматик, таких как контекстно-свободные или регулярные грамматики.

Для определения синтаксических связей как правило используют графовые методы.

\textit{Определение} \textbf{Синтаксическое дерево}(parse tree) описывается как взаимодействие между \begin{itemize}
    \item упорядоченным деревом $T$, которое представляет синтаксическую структуру предложения.
    \item синтаксической категории (например, S, NP, VP), либо терминальному символу, присеваемому множеству вершин дерева $T$ $N$ . 
    \item корню дерева $T$  $r$, обозначающем всю структуру предложения. 
    \item Листья представляют собой слова в предложении. Листья** $L \subseteq N$ — это узлы, которые не имеют дочерних узлов. 
    \item Cинтаксические категории или промежуточные составляющие, соответствующие $I = N \setminus L$ — это узлам, у которых имеется хотя бы одного потомка. 
\end{itemize}

Тогда дерево разбора $T$ для строки $w = w_1 w_2 \ldots w_n$ запишется как четверка $T = (N, E, r, L)$, где \begin{itemize}
    \item $N$ — конечное множество узлов,
    \item $E \subseteq N \times N$ — множество ребер, каждое из которых соединяет пару узлов (родитель — потомок),
    \item  $r \in N$ — корень дерева,
    \item $L = \{ w_1, w_2, \ldots, w_n \} \subseteq N$ — множество листьев, соответствующих словам в предложении.
\end{itemize}






