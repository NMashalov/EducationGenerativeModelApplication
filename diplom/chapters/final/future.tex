Направления генеративного искусственного интеллекта в области имеет высокий потенциал для изучения. 
Автор считает перспективным изучение применимости генеративного музыкального творчества \cite{vinze2021application}.
Предполагается, что дополнение голоса начинающего певца зрелыми интонациями и тембром может ускорить обучение.
Также интересны применения декомпозиции музыкальных произведений \cite{simpson2015deep}. 
Таким образом, обучающиеся могут явным образом услышать различия в построении партии для музыкальных инструментов.

Автор считает, что для задач всеобщего образования актуальны системы интерпретируемого автоматической оценки задач.
Наибольшую потенциал на текущий имеет в направлении имеет символьная регрессия,
использованная в известной работы команды DeepMind \cite{trinh2024solving}. Предложенный метод позволяет решать
задачи из международной олимпиады по математике на уровне победителей.

С развитием языковых моделей появилась потребность в формировании высококачественных педагогических данных, 
адапатированных для обучения ассистента. Один из подходов к автоматической структуризации предлагается направлением Data Mining.
Его применение описано \cite{romero2013data}. Предполагается создание графа знаний, связывающие термины семантическими связями.

