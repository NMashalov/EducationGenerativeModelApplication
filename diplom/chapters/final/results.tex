Поставленные цели по адаптации большой языковой цели были успешно выполнены \begin{enumerate}
    \item Разработка ассистента
\end{enumerate}.


Исследована постановка алгоритма Роббинса-Монро в условиях отклика, представленной случайной бернуллевской величины. 
Для случая ответа в виде логистической функции получен адаптивный численный алгоритм.К его ключевые преимуществам можно отнести:
\begin{itemize}
    \item оптимальную скорость сходимости при выполнение условий теоремы \ref{algo}
    \item стабильную дисперсию $s$ и $d$ на всех шагах оптимизации
    \item работу с естественными параметрами логистического распределения на всем интервале оптимизации. Это обстоятельство выгодно выделяет метод от классических методов, требующих подбора шага оптимизации.
\end{itemize}
Тем не менее алгоритм требует выбора априорных представлений о наклоне функции логистического распределения \ref{exp3:lose_effictivness}. Выполнить такой расчет можно на экспериментальных данных, использовав 
в качестве бинарного классификатора логистическую регрессию.
