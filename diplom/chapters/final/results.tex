Поставленная в работе цель по адаптации большой языковой цели к задачам разработки
интеллектуального ассистента была успешно выполнена. Разработанный интеллектуальный ассистент сопровождает процесс обучения 
игре в шахматы и рисованию, задает оптимальный уровень сложности и поддерживает тематический разговор. Полученные в ходе разработки 
алгоритма адаптивной сложности результаты имеют научную ценность, заключающаяся в повышении
скорости спуска к заданному решению. Подтверждение полученного результата выполнено путем проведения численного эксперимента с особым вниманием к крайним случаям.

Для выполнения поставленных задачи были исследованы подходы к созданию и обучению больших языковых моделей, представляющих выделенный класс
порождающих моделей. Значительный объем обучающей выборки позволяет таким моделям вырабатывать эвристические
правила обработки естественного языка, позволившие расширить ряд задач, выполняемых интеллектуальными ассистентами.
Исследователи демонстрируют выдающиеся результаты больших языковых моделей по выполнению инструкций, 
заданных на естественном языке, 
включающих задачи перевода, пересказа и информационного поиска. Адаптация моделей для выполнения задач в предметных дисциплинах
выполняется путем создания корпусов текстов, включающих примеры использования профессиональных знаний в общении специалистов и в описание
решения практических задач. Обучение выполняется экспертом,системно исправляющим ошибки в ответах модели, таким образом совершенствуя 
ее знания.  

Для демонстрации практического применения было разработано интерактивное приложение с 
учетом текущих возможностей больших языковых моделей. В качестве исследуемой модели была использована архитектура Llama3,
к текущему моменту имеющая наилучшие метрические показатели. Для адаптации модели в области естественных наук был собран корпус
документов образовательной тематики мощностью порядка миллион слов и десятка тысяч изображений. Также были разработаны 
модули очистки речи от бранных слов и исключения изображений неприемлемых по содержанию. В качестве предмета персонального обучения были 
выбраны шахматы и рисование из соображений развития структурного и стратегического мышления. Заданные предметы имеют ясную проблематику и 
интерпретируемый уровень сложности.

Аналитический обзор литературы диалоговых систем показал, что практическое применение больших моделей 
требует создания дополнительных модулей, выполняющих задачи планирования и оптимизации целевых показателей. Поэтому для адаптации
большой языковой модели был разработана модификация алгоритма Роббинса-Монро для задания оптимальной сложности задач. Численная
схема была получена путем минимизации дисперсии на каждом шаге спуска с явным учетом вида отклика, заданного логистической функции.
Приближения гауссовым интегралом позволило получить аналитический вид коэффициентов схемы. Алгоритм имеет ряд весомых преимуществ для эксперта,
заключающихся в ускоренном спуске и возможности выбора параметра шага спуска исходя из априорных представлений 
о знаниях учащегося. Валидация эксперимента в крайних постановках подтверждает эффективность алгоритма в сравнение с классическими подходами.

Проведенное исследование имеет практическое применение в оптимизации работы банковской коммуникаций с клиентами.
Интеллектуальные ассистенты высвобождают операторов поддержки и берут на себя диалог по наиболее часто возникающим
вопросам. Вариант разработанного алгоритма используется в системе телефонного информирования клиентов,
и позволяет сформировать индивидуальный график контактов с учетом требований федерального закона.
В качестве переменных используются длительность и время коммуникации. 
В совокупности с использованием большой языковой модели коммуникация с клиентами может строиться менее
формально и в стиле, учитывающем реакцию клиента. Исследования позволили сократить время
обработки контакта оператором и повысить удовлетворенность клиентов.

