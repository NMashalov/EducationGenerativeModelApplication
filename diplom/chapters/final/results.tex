Поставленные цели по адаптации большой языковой цели к задачам разработки интеллектуального ассистента были успешно достигнуты.
Ассистент сопровождает процесс обучения шахматам и  

Исследована постановка алгоритма Роббинса-Монро в условиях отклика представленной случайной бернуллевской величины. 
Для случая ответа в виде логистической функции получен адаптивный численный алгоритм. К его ключевые преимуществам можно отнести:
\begin{itemize}
    \item оптимальную скорость сходимости при выполнение условий теоремы \ref{algo};
    \item стабильную дисперсию $s$ и $d$ на всех шагах оптимизации;
    \item работу с естественными параметрами логистического распределения на всем интервале оптимизации.
    Это обстоятельство выгодно отличает метод от классических методов, требующих подбора шага оптимизации.
\end{itemize}
Тем не менее алгоритм требует выбора априорных представлений о наклоне функции логистического распределения \ref{exp3:lose_effictivness}. 
Выполнить данный расчет возможно на экспериментальных данных, используя в качестве бинарного классификатора логистическую регрессию.
