Теория машинное обучения объединяет статистические подходы к выявлению
эмпирических зависимостей из данных. Базовыми задачами направления являются определение корреляций между признаками,
выделение характерных подгрупп и создание порождающих функций. Для этого используется методы, основанные на 
теории вероятности, меры, графов, и оптимизации. 

Прогресс в развитие вычислительных технологий позволил хранить и обрабатывать значительные объемы данных. Для изучения
стали доступны корпусы текстов объемом до триллиона слов на разнообразных языках мирах, повсеместно стала доступна 
обработка и модификация коллекций изображений высокого качества. Также значительное число данных приобрело доступный 
к исследованию контекст взаимодействия, включающий комментарии к изображениям, ответы на экспертных порталах и открытые репозитории
программного кода. Современное развитие машинного обучения тесно связано с этим трендом. Наиболее цитируемыми современными работами
являются численно эффективные алгоритмы обработки данных, представленных в естественной для восприятия человека форме 
текста \cite{vaswani2017attention} \cite{radford2019language} и изображения \cite{kingma2013aut} \cite{song2020score}. Базовой
моделью обработки являются искусственные нейросети, получившие распространение с методами автоматического дифференцирования 
\cite{paszke2017automatic}\cite{baydin2018automatic}
и обратного распространения ошибки \cite{rumelhart1986learning}

В главе приведено описание современных методов машинного обучения, включающих введение в нейросети и механизм внимания
\cite{vaswani2017attention}, теоретический вывод избранных порождающих моделей и машинные подходы к обработке естественного языка.

