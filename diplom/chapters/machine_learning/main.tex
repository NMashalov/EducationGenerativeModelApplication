Машинное обучение направление объединяющее алгоритмические и статистические подходы к обработке информации.
Основными задачами является анализ и генерация данных в постановках с ограниченным набором примеров генеральной совокупности.
Для выполнения задач используются методы основанные на теории вероятности, графов, игр и оптимизации. 


и машинного зрения \cite{rombach2022highresolution},\cite{song2020generative} определили

\section{Использование нейросетевых подходов}

Порождающие модели современное и быстро развивающие направление работы с данными, направленное на их
создание и получение вероятностной массы. Ключевыми достижениями в дисциплине были \begin{enumerate}
    \item порождающие грамматики \cite{chomsky2002syntactic}
    \item графические вероятностные модели \cite{pearl1988probabilistic}
    \item состязательные порождающие модели \cite{goodfellow2020generative}
    \item диффузионные порождающие модели \cite{song2020score}
\end{enumerate}
Такие модели также представляют интерес для предметных исследователей, поскольку позволяют аналитически изучать
элементарные механизмы задания графа.

Порождающие модели задают совместное распределение наблюдаемого объекта $x$ и его черт $y$ -  $p(x,y)$. В этом заключается 
ключевое различие между порождающими и дискриминирующими моделями $p(y|x)$ \ref{discr_vs_gen}.

\begin{figure}[h]
    \centering
    \includegraphics[width=0.5\textwidth]{assets/ml/generation/stable_diffusion.png}
    \caption{Архитектура современной модели Stable Diffusion}
    \label{discr_vs_gen}
\end{figure}

Порождающие модели, используют параметрические модели $p_\theta$ для аппроксимации истинных функций распределений на наборе обучающих данных.
Выбор параметрической функции аппроксимации, как правило, зависит от числа примеров в коллекции данных. Для больших наборов данных как правило используют нейросети.
Простейшим видом порождающей модели является авторегрессионная модель модель, использующая предшествующий контекст для предсказания следующего элемента.

\textit{Определение } \textbf{Авторегрессионные модели} представляют собой класс порождающих моделей,
с вычислимой  вероятностью, выполняющие генерацию через цепочку последовательных преобразований \begin{equation}
    p(x^{(1)},\dots,x^{(t)}) = \prod_{t=1}^T p_\theta(x^{(t)}|x^{(1)},\dots,x^{(t-1)})
\end{equation}

Как правило, авторегресионные модели используются для генерации последовательностей, временных рядов и текста.
Класс плохо применим к данные не подлежащие однозначному упорядочиванию или с неравномерным шагом. 

Для оценки разницы между вероятностными распределениями используются \textbf{дивергенции} с набором правил \begin{enumerate}
    \item $\forall p,q \in M \rightarrow D(p,q) \ge 0$  
    \item $p=q \leftrightarrow D(p,q) = 0$
    \item $\forall p \rightarrow D(p,p+dp)$ положительно определенная квадратичная фоорма  
\end{enumerate}

В отличие  от метрики дивергенции не обязаны быть симметричны.ьНа практике как правило используются специальный класс $f$-дивергенций, задающихся
через матожидание.

\textit{Определение} $f$-дивергенцией называется выпуклая функция, удовлетворяющая равенству $f(1)=0$.
$$
    D_f{\pi \parallel \rho} = \mathrm E_{\rho(x)} f\left(\frac{\pi(x)}{\rho(x)}\right)
$$

Семейство $f$-дивергенций включает функции \begin{enumerate}
    \item дивергенция Кульбака-Лейбнера $u logu $
    \item обратная дивергенция Кульбака-Лейбнера $-ln u$
    \item дивергенция Йенсена-Шэннона  $\frac{1}{2}\left(u ln u - (u+1) ln(\frac{u+1}{2})\right)$
\end{enumerate}


Нижней вариационной оценкой называется техника максимизации подпирающей границы параметрического распределения $p(\mathbf{x},\mathbf{z})$ вторым $q(\mathbf{x},\mathbf{z})$,
где переменная $\mathbf{z}$ называется скрытой . В аналитической форме нижняя граница записывается как 
$$
    \mathcal{L}(\phi,\theta;x) = \mathbb{E}_{z \sim q_\phi(z|x)} \left[\ln \frac{p_{\theta}(x,z)}{q_{\phi}(z|x)}\right],
$$

Перепишем через KL-дивергенцию:
\begin{equation}
    \begin{aligned}
        & D_\text{KL}( q_\phi(\mathbf{z}\vert\mathbf{x}) \| p_\theta(\mathbf{z}\vert\mathbf{x}) ) & \\
        &=\int q_\phi(\mathbf{z} \vert \mathbf{x}) \log\frac{q_\phi(\mathbf{z} \vert \mathbf{x})}{p_\theta(\mathbf{z} \vert \mathbf{x})} d\mathbf{z} & \\
        &=\int q_\phi(\mathbf{z} \vert \mathbf{x}) \log\frac{q_\phi(\mathbf{z} \vert \mathbf{x})p_\theta(\mathbf{x})}{p_\theta(\mathbf{z}, \mathbf{x})} d\mathbf{z}\\
        &=\int q_\phi(\mathbf{z} \vert \mathbf{x}) \big( \log p_\theta(\mathbf{x}) + \log\frac{q_\phi(\mathbf{z} \vert \mathbf{x})}{p_\theta(\mathbf{z}, \mathbf{x})} \big) d\mathbf{z} & \\
        &=\log p_\theta(\mathbf{x}) + \int q_\phi(\mathbf{z} \vert \mathbf{x})\log\frac{q_\phi(\mathbf{z} \vert \mathbf{x})}{p_\theta(\mathbf{z}, \mathbf{x})} d\mathbf{z} \\
        &=\log p_\theta(\mathbf{x}) + \int q_\phi(\mathbf{z} \vert \mathbf{x})\log\frac{q_\phi(\mathbf{z} \vert \mathbf{x})}{p_\theta(\mathbf{x}\vert\mathbf{z})p_\theta(\mathbf{z})} d\mathbf{z} \\
        &=\log p_\theta(\mathbf{x}) + \mathbb{E}_{\mathbf{z}\sim q_\phi(\mathbf{z} \vert \mathbf{x})}[\log \frac{q_\phi(\mathbf{z} \vert \mathbf{x})}{p_\theta(\mathbf{z})} - \log p_\theta(\mathbf{x} \vert \mathbf{z})] &\\
        &=\log p_\theta(\mathbf{x}) + D_\text{KL}(q_\phi(\mathbf{z}\vert\mathbf{x}) \| p_\theta(\mathbf{z})) - \mathbb{E}_{\mathbf{z}\sim q_\phi(\mathbf{z}\vert\mathbf{x})}\log p_\theta(\mathbf{x}\vert\mathbf{z}) &
    \end{aligned}
\end{equation}

Следовательно:
\begin{equation}
    \log p_\theta(\mathbf{x}) - D_\text{KL}( q_\phi(\mathbf{z}\vert\mathbf{x}) \| p_\theta(\mathbf{z}\vert\mathbf{x}) ) = \mathbb{E}_{\mathbf{z}\sim q_\phi(\mathbf{z}\vert\mathbf{x})}\log p_\theta(\mathbf{x}\vert\mathbf{z}) - D_\text{KL}(q_\phi(\mathbf{z}\vert\mathbf{x}) \| p_\theta(\mathbf{z}))
\end{equation}
    
Таким образом, из неравенства Йенсена получаем: 
$$
    \ln p_\theta(x) \le \mathcal{L}(\phi,\theta;x)
$$
Базовым алгоритмом оптимизации вариационных моделей является EM-алгоритм, состоящий из последовательного обновления
скрытых представлений и максимизации правдоподобия с заданным параметрами.

\textit{Определение} \textbf{EM-алгоритм} - алгоритм для нахождения оценок
максимального правдоподобия параметров  вероятностных моделей с скрытыми переменными $\theta$.

Аналитически шаги алгоритма записываются как: \begin{itemize}
    \item расчет матожидания при заданном на шаге $t$ параметре $\theta^{(t)}$.
    Шаг обноez \begin{equation}
        Q(\theta| \theta^{(t)}) = \mathbb{E}_{\mathbf{Z} \sim p(\mathbf{Z}|X,\theta^{(t)})} \left[ \log p(\mathbf{X},\mathbf{Z}|\mathbf{\theta})\right]
    \end{equation}
    \item максимизации полученного выражения для нового шага $\mathbf{\theta}^{(t+1)}$: \begin{equation}
        \mathbf{\theta}^{(t+1)} = \text{arg} \max_{\theta} Q(\mathbf{\theta}|\theta^{(t)})
    \end{equation}  
\end{itemize}






\subsection{Оптимизация графа}




Байесовы сети применяются в причинно-следственном анализе.

Итоговая модель представляет собой статистическую модель явления, которую можно использовать для
управления и анализа системой. 
Описанный подход называется \textit{вариационным причиннно-следственным выводом}.

В теории причинно-следственного анализа введенную величину
свидетельство. (\texit{англ.} Evidence).

Максимизация свидетельства
позволяет выбирать модели согласно принципу бритву Оккама, исключая
параметры не вносящие существенного смысла для модели. Для расчета свидетельства необходимо выполнить маргинализацию по параметрам
модели $\int P(X, \theta) d\theta$. В общем случае задача принадлежит 
NP-классу сложности, рассчитывается за время, экспоненциально зависящее
от числа параметров. 

\texit{Определение} События $A$ и $B$ cчитаются независимыми в условиях:
\begin{equation}
    P(A \cup B) = P(A) \times P(B)
\end{equation}

На практике в системах подлинная независимость случайных величин $x$ и $y$
$x \perp y$ встречается не всегда. Чаще достижима условная независимость, наблюдаемая при
фиксации третьего фактора $z$.

\texit{Определение} События $A$ и $B$ cчитаются условно независимыми 
для заданного события $C$ в условиях:
\begin{equation}
    P(A \cup B |C) = P(A|C) \times P(B|C)
\end{equation}

\begin{figure}[h]
    \centering
    \includegraphics[width=0.5\textwidth]{assets/math/discrete/bayes_net.excalidraw.png}
    \caption{Посредник(\textit{англ.} mediator), общий предок(\textit{англ.} cofounder), 
 общий родственник \textit{англ.} collider) }
    \label{discr_vs_gen}
\end{figure}

Виды вариационного вывода можно разделить на три ключевых направлениях: \begin{itemize}
    \item прогнозирование
    \item обратный - объяснение причины на основании
\end{itemize}

Вывод выполняется путем маргинализации распределения по 



Многокомпонетный осложняется неоднозначной трактовкой исхода. 
Для анализа сложных систем предпочтителен однофакторный анализ, выполняющий
для его выполнения необходимо перекрыть потоки зависимостей (\textit{англ.} dependency flow) от
прочих переменных.

\texit{Определение} Интервенцией называется изменением 

\begin{figure}[h]
    \centering
    \includegraphics[width=0.5\textwidth]{assets/math/discrete/dep_flow.excalidraw.png}
    \caption{Интервенция}
    \label{discr_vs_gen}
\end{figure}





\texit{Определение} События $A$ и $B$ cчитаются независимыми в условиях:
\begin{equation}
    P(A \cup B) = P(A) \times P(B)
\end{equation}





\begin{figure}[h]
    \centering
    \includegraphics[width=0.5\textwidth]{assets/math/discrete/bayes_net.excalidraw.png}
    \caption{Посредник(\textit{англ.} mediator), общий предок(\textit{англ.} cofounder), 
 общий родственник \textit{англ.} collider) }
    \label{discr_vs_gen}
\end{figure}




Условная независимость позволяет 
$x \perp y$ встречается не всегда.




\subsection{Оптимизация графа}

\textit{Определение} Графической вероятностной моделью





Принципиально выполняется задача факторизации 


Пробалистические языки программирования.





\subsection{Оптимизация графа}

Задача оптимального транспорта (Optimal Transport)\cite{villani2009optimal} является одной из ключевых  
в области теории вероятностей и машинного обучения.
Она представляет собой проблему определения оптимального способа перемещения вероятностной массы из одной 
распределенной системы в другую с минимальными затратами или стоимостью. Формально задача состоит в составлении 
транспортного плана \( T: \mathcal{X} \rightarrow \mathcal{Y} \), 
который переводит распределение \( \mu \subset \mathcal{X}\) в распределение \( \nu \subset \mathcal{Y} \), минимизируя некоторую функцию стоимости. 
Функция стоимости $c: \mathcal{X} \times \mathcal{Y} \rightarrow \mathbb{R}$ обычно является мерой сходства между элементами из \( \mathcal{X} \) и \( \mathcal{Y} \), 
такой как квадрат расстояния. 

\textit{Определение} (Монже): \textbf{Оптимальный транспорт} по Монже вводится путем рассмотрения вероятностных 
распределений \( \mu \) и \( \nu \) на метрических пространствах \( \mathcal{X} \) и \( \mathcal{Y} \):
\begin{equation}
    \inf_{\gamma \in \Pi(\mu, \nu)} \int_{\mathcal{X} \times \mathcal{Y}} c(x,y) \, d\gamma(x,y),
\end{equation}
где \( \Pi(\mu, \nu) \) обозначает множество всех возможных совместных распределений 
\( \gamma \) на \( \mathcal{X} \times \mathcal{Y} \) с фиксированными маргинальными
распределениями \( \mu \) и \( \nu \), а \( c(x,y) \) — функция стоимости перевозки массы из \( x \) в \( y \).

\textit{Определение} (Канторович): \textbf{Оптимальный транспорт} по Канторовичу  вводится через потенциал $\phi$, 
который минимизирует функционал стоимости:
\begin{equation}
    \inf_{\phi} \left( \int_{\mathcal{X}} \phi(x) \, d\mu(x) + \int_{\mathcal{Y}} \psi(y) \, d\nu(y) \right),
\end{equation}
где \( \psi \) — обратная функция к \( \phi \). Таким образом, 
отображение \( T: \mathcal{X} \rightarrow \mathcal{Y} \) вытекает из градиента потенциала.

Заметим, что постановка Канторовича обобщает постановку Монже\ref{monge_vs_kantarovich}. В отличие от постановки Монже 
оптимальный транспорт по Канторовичу допускает распределение вероятностной массы в непрерывном случае.

\begin{figure}[h]
    \centering
    \includegraphics[width=0.7\textwidth]{assets/math/transport/optimal_transport.excalidraw.png}
    \caption{Различие в подходе по Монже и Кантаровичу. В постановке Канторовича задача релаксирует до 
    непрерывного распределения}
    \label{monge_vs_kantarovich}
\end{figure}

Итоговая стоимость оптимального транспортного плана называется метрикой Вассерштейна.

\textit{Определение:} Пусть \((X, d)\) --- метрическое пространство и \(P(X)\) --- множество всех вероятностных мер на \(X\). 
Для двух вероятностных мер \(\mu\) и \(\nu\) на \(X\) \textbf{метрика Вассерштейна} порядка \(p\), где \(p \geq 1\), определяется как:
\begin{equation}
    W_p(\mu, \nu) = \left( \inf_{\gamma \in \Gamma(\mu, \nu)} \int_{X \times X} d(x, y)^p \, d\gamma(x, y) \right)^{1/p},
\end{equation}
где \(\Gamma(\mu, \nu)\) — множество всех сопряжённых мер \(\gamma\) на \(X \times X\) с маргиналами \(\mu\) и \(\nu\).

Метрика имеет практическое применение для задач физики, биологии и машинного обучения, поскольку задает 
дифференцируемую разность между распределениями.

\textit{Определение:} \textbf{Метрическая производная} кривой $\rho_t,t \in [0,T]$ в вероятностном 
пространстве $\mathcal{P}_2(\mathbb{R}^N)$ запишется как:
\begin{equation}
    |\rho_t'| = \lim_{dt \rightarrow 0} \frac{\mathcal{W}_2(\rho_t, \rho_{t+dt})}{dt}
\end{equation}

Метод оптимального транспорта также активно применяется для анализа стохастических процессов. Базовой моделью, 
описывающей стохастическое движение с смещением, является процесс Ланжевена.

\textit{Определение:} Процесс Ланжевена называется случайнный процесс вида
\begin{equation}
    d X_t = - \nabla \Phi(x) dt + \sqrt{2 \beta^{-1}} d W_t,
\end{equation}
где $\Phi(X)$ --- потенциал задающий снос частицы, $\beta$ - масштаб блуждания и  $d W_t$ - процесс Винера.

\begin{figure}[h]
    \centering
    \includegraphics[width=0.5\textwidth]{assets/math/transport/fokker-plank.excalidraw.png}
    \caption{Эволюция вероятностной массы в уравнение Ланжевена}
    \label{opt_transport}
\end{figure}

Стохастическое усреднение процесса Ланжевена можно описать с помощью уравнения Колмогорова-Фоккера-Планка, 
задающего эволюцию вероятностной массы в дифференциальной форме  $\rho_t(x)$:
\begin{equation}
    \frac{\partial \rho_t}{\partial t} = \text{div}(\nabla \Phi(x) \rho_t) + \beta^{-1} \Delta \rho_t.
\end{equation}

Для естественной работы в энергетических постановках водится функционал, задающий коэффициента сноса с потенциалом $\Phi$.
Таким образом, исходное уравнение можно переписать в вариационной постановке \ref{variation_fp}.

\textit{Определение:} \textbf{Функционал Фоккера-Планка} для распределения $\rho$ записывается как: 
\begin{equation}
    \mathcal{F}_{FP}(\rho) = \int  \Phi(x) d\rho(x) + \beta^{-1} \int \log \rho(x) d \rho(x).
\end{equation}
\begin{figure}[h]
    \centering
    \includegraphics[width=0.5\textwidth]{assets/math/transport/functional.excalidraw.png}
    \caption{Визуализация постановки уравнения Фоккера-Планка в вариационной форме}
    \label{variation_fp}
\end{figure}
Научная группа Йордана-Кинана-Отто в работе \cite{jordan1998variational} показала, что маргинальные вероятностные
меры процесса Ланжевена подчиняются уравнению градиентного потока Вассерштейна относительно функционала Фоккера-Планка.

\textit{Определение:}  \textbf{Схема Йордана-Кинана-Отто} задает правило обновления уравнения вероятности в виде
минимизации функционала энергии и расстояния:
\begin{equation}
    \rho^{n+1} = \underset{\rho}{\operatorname{argmin}} \left( \frac{1}{2\tau} W_2^2(\rho, \rho^n) + \mathcal{F}(\rho) \right),
\end{equation}
где:\begin{itemize}
    \item \(\tau > 0\) --- шаг по времени;
    \item \(W_2(\rho, \rho^n)\) --- метрика Вассерштейна 2-го порядка между плотностями \(\rho\) и \(\rho^n\);
    \item  \(\mathcal{F}(\rho)\) --- функционал свободной энергии, который может включать в себя энтропийный член и 
    потенциальную энергию системы.
\end{itemize}
Функционал свободной энергии\(\mathcal{F}(\rho)\) задается в виде:
\begin{equation}
    \mathcal{F}(\rho) = \int_V f(\rho(x)) \, dx + \int_V V(x) \rho(x) \, dx,
\end{equation}
где \(f(\rho)\) — внутренний энергетический член, зависящий от плотности, а \(V(x)\) — внешний потенциал.






\subsection{Оптимизация графа}

Задача оптимального транспорта (Optimal Transport)\cite{villani2009optimal} является одной из ключевых  
в области теории вероятностей и машинного обучения.
Она представляет собой проблему определения оптимального способа перемещения вероятностной массы из одной 
распределенной системы в другую с минимальными затратами или стоимостью. Формально задача состоит в составлении 
транспортного плана \( T: \mathcal{X} \rightarrow \mathcal{Y} \), 
который переводит распределение \( \mu \subset \mathcal{X}\) в распределение \( \nu \subset \mathcal{Y} \), минимизируя некоторую функцию стоимости. 
Функция стоимости $c: \mathcal{X} \times \mathcal{Y} \rightarrow \mathbb{R}$ обычно является мерой сходства между элементами из \( \mathcal{X} \) и \( \mathcal{Y} \), 
такой как квадрат расстояния. 

\textit{Определение} (Монже): \textbf{Оптимальный транспорт} по Монже вводится путем рассмотрения вероятностных 
распределений \( \mu \) и \( \nu \) на метрических пространствах \( \mathcal{X} \) и \( \mathcal{Y} \):
\begin{equation}
    \inf_{\gamma \in \Pi(\mu, \nu)} \int_{\mathcal{X} \times \mathcal{Y}} c(x,y) \, d\gamma(x,y),
\end{equation}
где \( \Pi(\mu, \nu) \) обозначает множество всех возможных совместных распределений 
\( \gamma \) на \( \mathcal{X} \times \mathcal{Y} \) с фиксированными маргинальными
распределениями \( \mu \) и \( \nu \), а \( c(x,y) \) — функция стоимости перевозки массы из \( x \) в \( y \).

\textit{Определение} (Канторович): \textbf{Оптимальный транспорт} по Канторовичу  вводится через потенциал $\phi$, 
который минимизирует функционал стоимости:
\begin{equation}
    \inf_{\phi} \left( \int_{\mathcal{X}} \phi(x) \, d\mu(x) + \int_{\mathcal{Y}} \psi(y) \, d\nu(y) \right),
\end{equation}
где \( \psi \) — обратная функция к \( \phi \). Таким образом, 
отображение \( T: \mathcal{X} \rightarrow \mathcal{Y} \) вытекает из градиента потенциала.

Заметим, что постановка Канторовича обобщает постановку Монже\ref{monge_vs_kantarovich}. В отличие от постановки Монже 
оптимальный транспорт по Канторовичу допускает распределение вероятностной массы в непрерывном случае.

\begin{figure}[h]
    \centering
    \includegraphics[width=0.7\textwidth]{assets/math/transport/optimal_transport.excalidraw.png}
    \caption{Различие в подходе по Монже и Кантаровичу. В постановке Канторовича задача релаксирует до 
    непрерывного распределения}
    \label{monge_vs_kantarovich}
\end{figure}

Итоговая стоимость оптимального транспортного плана называется метрикой Вассерштейна.

\textit{Определение:} Пусть \((X, d)\) --- метрическое пространство и \(P(X)\) --- множество всех вероятностных мер на \(X\). 
Для двух вероятностных мер \(\mu\) и \(\nu\) на \(X\) \textbf{метрика Вассерштейна} порядка \(p\), где \(p \geq 1\), определяется как:
\begin{equation}
    W_p(\mu, \nu) = \left( \inf_{\gamma \in \Gamma(\mu, \nu)} \int_{X \times X} d(x, y)^p \, d\gamma(x, y) \right)^{1/p},
\end{equation}
где \(\Gamma(\mu, \nu)\) — множество всех сопряжённых мер \(\gamma\) на \(X \times X\) с маргиналами \(\mu\) и \(\nu\).

Метрика имеет практическое применение для задач физики, биологии и машинного обучения, поскольку задает 
дифференцируемую разность между распределениями.

\textit{Определение:} \textbf{Метрическая производная} кривой $\rho_t,t \in [0,T]$ в вероятностном 
пространстве $\mathcal{P}_2(\mathbb{R}^N)$ запишется как:
\begin{equation}
    |\rho_t'| = \lim_{dt \rightarrow 0} \frac{\mathcal{W}_2(\rho_t, \rho_{t+dt})}{dt}
\end{equation}

Метод оптимального транспорта также активно применяется для анализа стохастических процессов. Базовой моделью, 
описывающей стохастическое движение с смещением, является процесс Ланжевена.

\textit{Определение:} Процесс Ланжевена называется случайнный процесс вида
\begin{equation}
    d X_t = - \nabla \Phi(x) dt + \sqrt{2 \beta^{-1}} d W_t,
\end{equation}
где $\Phi(X)$ --- потенциал задающий снос частицы, $\beta$ - масштаб блуждания и  $d W_t$ - процесс Винера.

\begin{figure}[h]
    \centering
    \includegraphics[width=0.5\textwidth]{assets/math/transport/fokker-plank.excalidraw.png}
    \caption{Эволюция вероятностной массы в уравнение Ланжевена}
    \label{opt_transport}
\end{figure}

Стохастическое усреднение процесса Ланжевена можно описать с помощью уравнения Колмогорова-Фоккера-Планка, 
задающего эволюцию вероятностной массы в дифференциальной форме  $\rho_t(x)$:
\begin{equation}
    \frac{\partial \rho_t}{\partial t} = \text{div}(\nabla \Phi(x) \rho_t) + \beta^{-1} \Delta \rho_t.
\end{equation}

Для естественной работы в энергетических постановках водится функционал, задающий коэффициента сноса с потенциалом $\Phi$.
Таким образом, исходное уравнение можно переписать в вариационной постановке \ref{variation_fp}.

\textit{Определение:} \textbf{Функционал Фоккера-Планка} для распределения $\rho$ записывается как: 
\begin{equation}
    \mathcal{F}_{FP}(\rho) = \int  \Phi(x) d\rho(x) + \beta^{-1} \int \log \rho(x) d \rho(x).
\end{equation}
\begin{figure}[h]
    \centering
    \includegraphics[width=0.5\textwidth]{assets/math/transport/functional.excalidraw.png}
    \caption{Визуализация постановки уравнения Фоккера-Планка в вариационной форме}
    \label{variation_fp}
\end{figure}
Научная группа Йордана-Кинана-Отто в работе \cite{jordan1998variational} показала, что маргинальные вероятностные
меры процесса Ланжевена подчиняются уравнению градиентного потока Вассерштейна относительно функционала Фоккера-Планка.

\textit{Определение:}  \textbf{Схема Йордана-Кинана-Отто} задает правило обновления уравнения вероятности в виде
минимизации функционала энергии и расстояния:
\begin{equation}
    \rho^{n+1} = \underset{\rho}{\operatorname{argmin}} \left( \frac{1}{2\tau} W_2^2(\rho, \rho^n) + \mathcal{F}(\rho) \right),
\end{equation}
где:\begin{itemize}
    \item \(\tau > 0\) --- шаг по времени;
    \item \(W_2(\rho, \rho^n)\) --- метрика Вассерштейна 2-го порядка между плотностями \(\rho\) и \(\rho^n\);
    \item  \(\mathcal{F}(\rho)\) --- функционал свободной энергии, который может включать в себя энтропийный член и 
    потенциальную энергию системы.
\end{itemize}
Функционал свободной энергии\(\mathcal{F}(\rho)\) задается в виде:
\begin{equation}
    \mathcal{F}(\rho) = \int_V f(\rho(x)) \, dx + \int_V V(x) \rho(x) \, dx,
\end{equation}
где \(f(\rho)\) — внутренний энергетический член, зависящий от плотности, а \(V(x)\) — внешний потенциал.





\subsection{Оптимизация графа}

Задача оптимального транспорта (Optimal Transport)\cite{villani2009optimal} является одной из ключевых  
в области теории вероятностей и машинного обучения.
Она представляет собой проблему определения оптимального способа перемещения вероятностной массы из одной 
распределенной системы в другую с минимальными затратами или стоимостью. Формально задача состоит в составлении 
транспортного плана \( T: \mathcal{X} \rightarrow \mathcal{Y} \), 
который переводит распределение \( \mu \subset \mathcal{X}\) в распределение \( \nu \subset \mathcal{Y} \), минимизируя некоторую функцию стоимости. 
Функция стоимости $c: \mathcal{X} \times \mathcal{Y} \rightarrow \mathbb{R}$ обычно является мерой сходства между элементами из \( \mathcal{X} \) и \( \mathcal{Y} \), 
такой как квадрат расстояния. 

\textit{Определение} (Монже): \textbf{Оптимальный транспорт} по Монже вводится путем рассмотрения вероятностных 
распределений \( \mu \) и \( \nu \) на метрических пространствах \( \mathcal{X} \) и \( \mathcal{Y} \):
\begin{equation}
    \inf_{\gamma \in \Pi(\mu, \nu)} \int_{\mathcal{X} \times \mathcal{Y}} c(x,y) \, d\gamma(x,y),
\end{equation}
где \( \Pi(\mu, \nu) \) обозначает множество всех возможных совместных распределений 
\( \gamma \) на \( \mathcal{X} \times \mathcal{Y} \) с фиксированными маргинальными
распределениями \( \mu \) и \( \nu \), а \( c(x,y) \) — функция стоимости перевозки массы из \( x \) в \( y \).

\textit{Определение} (Канторович): \textbf{Оптимальный транспорт} по Канторовичу  вводится через потенциал $\phi$, 
который минимизирует функционал стоимости:
\begin{equation}
    \inf_{\phi} \left( \int_{\mathcal{X}} \phi(x) \, d\mu(x) + \int_{\mathcal{Y}} \psi(y) \, d\nu(y) \right),
\end{equation}
где \( \psi \) — обратная функция к \( \phi \). Таким образом, 
отображение \( T: \mathcal{X} \rightarrow \mathcal{Y} \) вытекает из градиента потенциала.

Заметим, что постановка Канторовича обобщает постановку Монже\ref{monge_vs_kantarovich}. В отличие от постановки Монже 
оптимальный транспорт по Канторовичу допускает распределение вероятностной массы в непрерывном случае.

\begin{figure}[h]
    \centering
    \includegraphics[width=0.7\textwidth]{assets/math/transport/optimal_transport.excalidraw.png}
    \caption{Различие в подходе по Монже и Кантаровичу. В постановке Канторовича задача релаксирует до 
    непрерывного распределения}
    \label{monge_vs_kantarovich}
\end{figure}

Итоговая стоимость оптимального транспортного плана называется метрикой Вассерштейна.

\textit{Определение:} Пусть \((X, d)\) --- метрическое пространство и \(P(X)\) --- множество всех вероятностных мер на \(X\). 
Для двух вероятностных мер \(\mu\) и \(\nu\) на \(X\) \textbf{метрика Вассерштейна} порядка \(p\), где \(p \geq 1\), определяется как:
\begin{equation}
    W_p(\mu, \nu) = \left( \inf_{\gamma \in \Gamma(\mu, \nu)} \int_{X \times X} d(x, y)^p \, d\gamma(x, y) \right)^{1/p},
\end{equation}
где \(\Gamma(\mu, \nu)\) — множество всех сопряжённых мер \(\gamma\) на \(X \times X\) с маргиналами \(\mu\) и \(\nu\).

Метрика имеет практическое применение для задач физики, биологии и машинного обучения, поскольку задает 
дифференцируемую разность между распределениями.

\textit{Определение:} \textbf{Метрическая производная} кривой $\rho_t,t \in [0,T]$ в вероятностном 
пространстве $\mathcal{P}_2(\mathbb{R}^N)$ запишется как:
\begin{equation}
    |\rho_t'| = \lim_{dt \rightarrow 0} \frac{\mathcal{W}_2(\rho_t, \rho_{t+dt})}{dt}
\end{equation}

Метод оптимального транспорта также активно применяется для анализа стохастических процессов. Базовой моделью, 
описывающей стохастическое движение с смещением, является процесс Ланжевена.

\textit{Определение:} Процесс Ланжевена называется случайнный процесс вида
\begin{equation}
    d X_t = - \nabla \Phi(x) dt + \sqrt{2 \beta^{-1}} d W_t,
\end{equation}
где $\Phi(X)$ --- потенциал задающий снос частицы, $\beta$ - масштаб блуждания и  $d W_t$ - процесс Винера.

\begin{figure}[h]
    \centering
    \includegraphics[width=0.5\textwidth]{assets/math/transport/fokker-plank.excalidraw.png}
    \caption{Эволюция вероятностной массы в уравнение Ланжевена}
    \label{opt_transport}
\end{figure}

Стохастическое усреднение процесса Ланжевена можно описать с помощью уравнения Колмогорова-Фоккера-Планка, 
задающего эволюцию вероятностной массы в дифференциальной форме  $\rho_t(x)$:
\begin{equation}
    \frac{\partial \rho_t}{\partial t} = \text{div}(\nabla \Phi(x) \rho_t) + \beta^{-1} \Delta \rho_t.
\end{equation}

Для естественной работы в энергетических постановках водится функционал, задающий коэффициента сноса с потенциалом $\Phi$.
Таким образом, исходное уравнение можно переписать в вариационной постановке \ref{variation_fp}.

\textit{Определение:} \textbf{Функционал Фоккера-Планка} для распределения $\rho$ записывается как: 
\begin{equation}
    \mathcal{F}_{FP}(\rho) = \int  \Phi(x) d\rho(x) + \beta^{-1} \int \log \rho(x) d \rho(x).
\end{equation}
\begin{figure}[h]
    \centering
    \includegraphics[width=0.5\textwidth]{assets/math/transport/functional.excalidraw.png}
    \caption{Визуализация постановки уравнения Фоккера-Планка в вариационной форме}
    \label{variation_fp}
\end{figure}
Научная группа Йордана-Кинана-Отто в работе \cite{jordan1998variational} показала, что маргинальные вероятностные
меры процесса Ланжевена подчиняются уравнению градиентного потока Вассерштейна относительно функционала Фоккера-Планка.

\textit{Определение:}  \textbf{Схема Йордана-Кинана-Отто} задает правило обновления уравнения вероятности в виде
минимизации функционала энергии и расстояния:
\begin{equation}
    \rho^{n+1} = \underset{\rho}{\operatorname{argmin}} \left( \frac{1}{2\tau} W_2^2(\rho, \rho^n) + \mathcal{F}(\rho) \right),
\end{equation}
где:\begin{itemize}
    \item \(\tau > 0\) --- шаг по времени;
    \item \(W_2(\rho, \rho^n)\) --- метрика Вассерштейна 2-го порядка между плотностями \(\rho\) и \(\rho^n\);
    \item  \(\mathcal{F}(\rho)\) --- функционал свободной энергии, который может включать в себя энтропийный член и 
    потенциальную энергию системы.
\end{itemize}
Функционал свободной энергии\(\mathcal{F}(\rho)\) задается в виде:
\begin{equation}
    \mathcal{F}(\rho) = \int_V f(\rho(x)) \, dx + \int_V V(x) \rho(x) \, dx,
\end{equation}
где \(f(\rho)\) — внутренний энергетический член, зависящий от плотности, а \(V(x)\) — внешний потенциал.





\section{Генеративные подходы}

Порождающие модели современное и быстро развивающие направление работы с данными, направленное на их
создание и получение вероятностной массы. Ключевыми достижениями в дисциплине были \begin{enumerate}
    \item порождающие грамматики \cite{chomsky2002syntactic}
    \item графические вероятностные модели \cite{pearl1988probabilistic}
    \item состязательные порождающие модели \cite{goodfellow2020generative}
    \item диффузионные порождающие модели \cite{song2020score}
\end{enumerate}
Такие модели также представляют интерес для предметных исследователей, поскольку позволяют аналитически изучать
элементарные механизмы задания графа.

Порождающие модели задают совместное распределение наблюдаемого объекта $x$ и его черт $y$ -  $p(x,y)$. В этом заключается 
ключевое различие между порождающими и дискриминирующими моделями $p(y|x)$ \ref{discr_vs_gen}.

\begin{figure}[h]
    \centering
    \includegraphics[width=0.5\textwidth]{assets/ml/generation/stable_diffusion.png}
    \caption{Архитектура современной модели Stable Diffusion}
    \label{discr_vs_gen}
\end{figure}

Порождающие модели, используют параметрические модели $p_\theta$ для аппроксимации истинных функций распределений на наборе обучающих данных.
Выбор параметрической функции аппроксимации, как правило, зависит от числа примеров в коллекции данных. Для больших наборов данных как правило используют нейросети.
Простейшим видом порождающей модели является авторегрессионная модель модель, использующая предшествующий контекст для предсказания следующего элемента.

\textit{Определение } \textbf{Авторегрессионные модели} представляют собой класс порождающих моделей,
с вычислимой  вероятностью, выполняющие генерацию через цепочку последовательных преобразований \begin{equation}
    p(x^{(1)},\dots,x^{(t)}) = \prod_{t=1}^T p_\theta(x^{(t)}|x^{(1)},\dots,x^{(t-1)})
\end{equation}

Как правило, авторегресионные модели используются для генерации последовательностей, временных рядов и текста.
Класс плохо применим к данные не подлежащие однозначному упорядочиванию или с неравномерным шагом. 

Для оценки разницы между вероятностными распределениями используются \textbf{дивергенции} с набором правил \begin{enumerate}
    \item $\forall p,q \in M \rightarrow D(p,q) \ge 0$  
    \item $p=q \leftrightarrow D(p,q) = 0$
    \item $\forall p \rightarrow D(p,p+dp)$ положительно определенная квадратичная фоорма  
\end{enumerate}

В отличие  от метрики дивергенции не обязаны быть симметричны.ьНа практике как правило используются специальный класс $f$-дивергенций, задающихся
через матожидание.

\textit{Определение} $f$-дивергенцией называется выпуклая функция, удовлетворяющая равенству $f(1)=0$.
$$
    D_f{\pi \parallel \rho} = \mathrm E_{\rho(x)} f\left(\frac{\pi(x)}{\rho(x)}\right)
$$

Семейство $f$-дивергенций включает функции \begin{enumerate}
    \item дивергенция Кульбака-Лейбнера $u logu $
    \item обратная дивергенция Кульбака-Лейбнера $-ln u$
    \item дивергенция Йенсена-Шэннона  $\frac{1}{2}\left(u ln u - (u+1) ln(\frac{u+1}{2})\right)$
\end{enumerate}


Нижней вариационной оценкой называется техника максимизации подпирающей границы параметрического распределения $p(\mathbf{x},\mathbf{z})$ вторым $q(\mathbf{x},\mathbf{z})$,
где переменная $\mathbf{z}$ называется скрытой . В аналитической форме нижняя граница записывается как 
$$
    \mathcal{L}(\phi,\theta;x) = \mathbb{E}_{z \sim q_\phi(z|x)} \left[\ln \frac{p_{\theta}(x,z)}{q_{\phi}(z|x)}\right],
$$

Перепишем через KL-дивергенцию:
\begin{equation}
    \begin{aligned}
        & D_\text{KL}( q_\phi(\mathbf{z}\vert\mathbf{x}) \| p_\theta(\mathbf{z}\vert\mathbf{x}) ) & \\
        &=\int q_\phi(\mathbf{z} \vert \mathbf{x}) \log\frac{q_\phi(\mathbf{z} \vert \mathbf{x})}{p_\theta(\mathbf{z} \vert \mathbf{x})} d\mathbf{z} & \\
        &=\int q_\phi(\mathbf{z} \vert \mathbf{x}) \log\frac{q_\phi(\mathbf{z} \vert \mathbf{x})p_\theta(\mathbf{x})}{p_\theta(\mathbf{z}, \mathbf{x})} d\mathbf{z}\\
        &=\int q_\phi(\mathbf{z} \vert \mathbf{x}) \big( \log p_\theta(\mathbf{x}) + \log\frac{q_\phi(\mathbf{z} \vert \mathbf{x})}{p_\theta(\mathbf{z}, \mathbf{x})} \big) d\mathbf{z} & \\
        &=\log p_\theta(\mathbf{x}) + \int q_\phi(\mathbf{z} \vert \mathbf{x})\log\frac{q_\phi(\mathbf{z} \vert \mathbf{x})}{p_\theta(\mathbf{z}, \mathbf{x})} d\mathbf{z} \\
        &=\log p_\theta(\mathbf{x}) + \int q_\phi(\mathbf{z} \vert \mathbf{x})\log\frac{q_\phi(\mathbf{z} \vert \mathbf{x})}{p_\theta(\mathbf{x}\vert\mathbf{z})p_\theta(\mathbf{z})} d\mathbf{z} \\
        &=\log p_\theta(\mathbf{x}) + \mathbb{E}_{\mathbf{z}\sim q_\phi(\mathbf{z} \vert \mathbf{x})}[\log \frac{q_\phi(\mathbf{z} \vert \mathbf{x})}{p_\theta(\mathbf{z})} - \log p_\theta(\mathbf{x} \vert \mathbf{z})] &\\
        &=\log p_\theta(\mathbf{x}) + D_\text{KL}(q_\phi(\mathbf{z}\vert\mathbf{x}) \| p_\theta(\mathbf{z})) - \mathbb{E}_{\mathbf{z}\sim q_\phi(\mathbf{z}\vert\mathbf{x})}\log p_\theta(\mathbf{x}\vert\mathbf{z}) &
    \end{aligned}
\end{equation}

Следовательно:
\begin{equation}
    \log p_\theta(\mathbf{x}) - D_\text{KL}( q_\phi(\mathbf{z}\vert\mathbf{x}) \| p_\theta(\mathbf{z}\vert\mathbf{x}) ) = \mathbb{E}_{\mathbf{z}\sim q_\phi(\mathbf{z}\vert\mathbf{x})}\log p_\theta(\mathbf{x}\vert\mathbf{z}) - D_\text{KL}(q_\phi(\mathbf{z}\vert\mathbf{x}) \| p_\theta(\mathbf{z}))
\end{equation}
    
Таким образом, из неравенства Йенсена получаем: 
$$
    \ln p_\theta(x) \le \mathcal{L}(\phi,\theta;x)
$$
Базовым алгоритмом оптимизации вариационных моделей является EM-алгоритм, состоящий из последовательного обновления
скрытых представлений и максимизации правдоподобия с заданным параметрами.

\textit{Определение} \textbf{EM-алгоритм} - алгоритм для нахождения оценок
максимального правдоподобия параметров  вероятностных моделей с скрытыми переменными $\theta$.

Аналитически шаги алгоритма записываются как: \begin{itemize}
    \item расчет матожидания при заданном на шаге $t$ параметре $\theta^{(t)}$.
    Шаг обноez \begin{equation}
        Q(\theta| \theta^{(t)}) = \mathbb{E}_{\mathbf{Z} \sim p(\mathbf{Z}|X,\theta^{(t)})} \left[ \log p(\mathbf{X},\mathbf{Z}|\mathbf{\theta})\right]
    \end{equation}
    \item максимизации полученного выражения для нового шага $\mathbf{\theta}^{(t+1)}$: \begin{equation}
        \mathbf{\theta}^{(t+1)} = \text{arg} \max_{\theta} Q(\mathbf{\theta}|\theta^{(t)})
    \end{equation}  
\end{itemize}






\subsection{Оптимизация графа}




Байесовы сети применяются в причинно-следственном анализе.

Итоговая модель представляет собой статистическую модель явления, которую можно использовать для
управления и анализа системой. 
Описанный подход называется \textit{вариационным причиннно-следственным выводом}.

В теории причинно-следственного анализа введенную величину
свидетельство. (\texit{англ.} Evidence).

Максимизация свидетельства
позволяет выбирать модели согласно принципу бритву Оккама, исключая
параметры не вносящие существенного смысла для модели. Для расчета свидетельства необходимо выполнить маргинализацию по параметрам
модели $\int P(X, \theta) d\theta$. В общем случае задача принадлежит 
NP-классу сложности, рассчитывается за время, экспоненциально зависящее
от числа параметров. 

\texit{Определение} События $A$ и $B$ cчитаются независимыми в условиях:
\begin{equation}
    P(A \cup B) = P(A) \times P(B)
\end{equation}

На практике в системах подлинная независимость случайных величин $x$ и $y$
$x \perp y$ встречается не всегда. Чаще достижима условная независимость, наблюдаемая при
фиксации третьего фактора $z$.

\texit{Определение} События $A$ и $B$ cчитаются условно независимыми 
для заданного события $C$ в условиях:
\begin{equation}
    P(A \cup B |C) = P(A|C) \times P(B|C)
\end{equation}

\begin{figure}[h]
    \centering
    \includegraphics[width=0.5\textwidth]{assets/math/discrete/bayes_net.excalidraw.png}
    \caption{Посредник(\textit{англ.} mediator), общий предок(\textit{англ.} cofounder), 
 общий родственник \textit{англ.} collider) }
    \label{discr_vs_gen}
\end{figure}

Виды вариационного вывода можно разделить на три ключевых направлениях: \begin{itemize}
    \item прогнозирование
    \item обратный - объяснение причины на основании
\end{itemize}

Вывод выполняется путем маргинализации распределения по 



Многокомпонетный осложняется неоднозначной трактовкой исхода. 
Для анализа сложных систем предпочтителен однофакторный анализ, выполняющий
для его выполнения необходимо перекрыть потоки зависимостей (\textit{англ.} dependency flow) от
прочих переменных.

\texit{Определение} Интервенцией называется изменением 

\begin{figure}[h]
    \centering
    \includegraphics[width=0.5\textwidth]{assets/math/discrete/dep_flow.excalidraw.png}
    \caption{Интервенция}
    \label{discr_vs_gen}
\end{figure}





\texit{Определение} События $A$ и $B$ cчитаются независимыми в условиях:
\begin{equation}
    P(A \cup B) = P(A) \times P(B)
\end{equation}





\begin{figure}[h]
    \centering
    \includegraphics[width=0.5\textwidth]{assets/math/discrete/bayes_net.excalidraw.png}
    \caption{Посредник(\textit{англ.} mediator), общий предок(\textit{англ.} cofounder), 
 общий родственник \textit{англ.} collider) }
    \label{discr_vs_gen}
\end{figure}




Условная независимость позволяет 
$x \perp y$ встречается не всегда.




\subsection{Оптимизация графа}

\textit{Определение} Графической вероятностной моделью





Принципиально выполняется задача факторизации 


Пробалистические языки программирования.





\subsection{Оптимизация графа}

Задача оптимального транспорта (Optimal Transport)\cite{villani2009optimal} является одной из ключевых  
в области теории вероятностей и машинного обучения.
Она представляет собой проблему определения оптимального способа перемещения вероятностной массы из одной 
распределенной системы в другую с минимальными затратами или стоимостью. Формально задача состоит в составлении 
транспортного плана \( T: \mathcal{X} \rightarrow \mathcal{Y} \), 
который переводит распределение \( \mu \subset \mathcal{X}\) в распределение \( \nu \subset \mathcal{Y} \), минимизируя некоторую функцию стоимости. 
Функция стоимости $c: \mathcal{X} \times \mathcal{Y} \rightarrow \mathbb{R}$ обычно является мерой сходства между элементами из \( \mathcal{X} \) и \( \mathcal{Y} \), 
такой как квадрат расстояния. 

\textit{Определение} (Монже): \textbf{Оптимальный транспорт} по Монже вводится путем рассмотрения вероятностных 
распределений \( \mu \) и \( \nu \) на метрических пространствах \( \mathcal{X} \) и \( \mathcal{Y} \):
\begin{equation}
    \inf_{\gamma \in \Pi(\mu, \nu)} \int_{\mathcal{X} \times \mathcal{Y}} c(x,y) \, d\gamma(x,y),
\end{equation}
где \( \Pi(\mu, \nu) \) обозначает множество всех возможных совместных распределений 
\( \gamma \) на \( \mathcal{X} \times \mathcal{Y} \) с фиксированными маргинальными
распределениями \( \mu \) и \( \nu \), а \( c(x,y) \) — функция стоимости перевозки массы из \( x \) в \( y \).

\textit{Определение} (Канторович): \textbf{Оптимальный транспорт} по Канторовичу  вводится через потенциал $\phi$, 
который минимизирует функционал стоимости:
\begin{equation}
    \inf_{\phi} \left( \int_{\mathcal{X}} \phi(x) \, d\mu(x) + \int_{\mathcal{Y}} \psi(y) \, d\nu(y) \right),
\end{equation}
где \( \psi \) — обратная функция к \( \phi \). Таким образом, 
отображение \( T: \mathcal{X} \rightarrow \mathcal{Y} \) вытекает из градиента потенциала.

Заметим, что постановка Канторовича обобщает постановку Монже\ref{monge_vs_kantarovich}. В отличие от постановки Монже 
оптимальный транспорт по Канторовичу допускает распределение вероятностной массы в непрерывном случае.

\begin{figure}[h]
    \centering
    \includegraphics[width=0.7\textwidth]{assets/math/transport/optimal_transport.excalidraw.png}
    \caption{Различие в подходе по Монже и Кантаровичу. В постановке Канторовича задача релаксирует до 
    непрерывного распределения}
    \label{monge_vs_kantarovich}
\end{figure}

Итоговая стоимость оптимального транспортного плана называется метрикой Вассерштейна.

\textit{Определение:} Пусть \((X, d)\) --- метрическое пространство и \(P(X)\) --- множество всех вероятностных мер на \(X\). 
Для двух вероятностных мер \(\mu\) и \(\nu\) на \(X\) \textbf{метрика Вассерштейна} порядка \(p\), где \(p \geq 1\), определяется как:
\begin{equation}
    W_p(\mu, \nu) = \left( \inf_{\gamma \in \Gamma(\mu, \nu)} \int_{X \times X} d(x, y)^p \, d\gamma(x, y) \right)^{1/p},
\end{equation}
где \(\Gamma(\mu, \nu)\) — множество всех сопряжённых мер \(\gamma\) на \(X \times X\) с маргиналами \(\mu\) и \(\nu\).

Метрика имеет практическое применение для задач физики, биологии и машинного обучения, поскольку задает 
дифференцируемую разность между распределениями.

\textit{Определение:} \textbf{Метрическая производная} кривой $\rho_t,t \in [0,T]$ в вероятностном 
пространстве $\mathcal{P}_2(\mathbb{R}^N)$ запишется как:
\begin{equation}
    |\rho_t'| = \lim_{dt \rightarrow 0} \frac{\mathcal{W}_2(\rho_t, \rho_{t+dt})}{dt}
\end{equation}

Метод оптимального транспорта также активно применяется для анализа стохастических процессов. Базовой моделью, 
описывающей стохастическое движение с смещением, является процесс Ланжевена.

\textit{Определение:} Процесс Ланжевена называется случайнный процесс вида
\begin{equation}
    d X_t = - \nabla \Phi(x) dt + \sqrt{2 \beta^{-1}} d W_t,
\end{equation}
где $\Phi(X)$ --- потенциал задающий снос частицы, $\beta$ - масштаб блуждания и  $d W_t$ - процесс Винера.

\begin{figure}[h]
    \centering
    \includegraphics[width=0.5\textwidth]{assets/math/transport/fokker-plank.excalidraw.png}
    \caption{Эволюция вероятностной массы в уравнение Ланжевена}
    \label{opt_transport}
\end{figure}

Стохастическое усреднение процесса Ланжевена можно описать с помощью уравнения Колмогорова-Фоккера-Планка, 
задающего эволюцию вероятностной массы в дифференциальной форме  $\rho_t(x)$:
\begin{equation}
    \frac{\partial \rho_t}{\partial t} = \text{div}(\nabla \Phi(x) \rho_t) + \beta^{-1} \Delta \rho_t.
\end{equation}

Для естественной работы в энергетических постановках водится функционал, задающий коэффициента сноса с потенциалом $\Phi$.
Таким образом, исходное уравнение можно переписать в вариационной постановке \ref{variation_fp}.

\textit{Определение:} \textbf{Функционал Фоккера-Планка} для распределения $\rho$ записывается как: 
\begin{equation}
    \mathcal{F}_{FP}(\rho) = \int  \Phi(x) d\rho(x) + \beta^{-1} \int \log \rho(x) d \rho(x).
\end{equation}
\begin{figure}[h]
    \centering
    \includegraphics[width=0.5\textwidth]{assets/math/transport/functional.excalidraw.png}
    \caption{Визуализация постановки уравнения Фоккера-Планка в вариационной форме}
    \label{variation_fp}
\end{figure}
Научная группа Йордана-Кинана-Отто в работе \cite{jordan1998variational} показала, что маргинальные вероятностные
меры процесса Ланжевена подчиняются уравнению градиентного потока Вассерштейна относительно функционала Фоккера-Планка.

\textit{Определение:}  \textbf{Схема Йордана-Кинана-Отто} задает правило обновления уравнения вероятности в виде
минимизации функционала энергии и расстояния:
\begin{equation}
    \rho^{n+1} = \underset{\rho}{\operatorname{argmin}} \left( \frac{1}{2\tau} W_2^2(\rho, \rho^n) + \mathcal{F}(\rho) \right),
\end{equation}
где:\begin{itemize}
    \item \(\tau > 0\) --- шаг по времени;
    \item \(W_2(\rho, \rho^n)\) --- метрика Вассерштейна 2-го порядка между плотностями \(\rho\) и \(\rho^n\);
    \item  \(\mathcal{F}(\rho)\) --- функционал свободной энергии, который может включать в себя энтропийный член и 
    потенциальную энергию системы.
\end{itemize}
Функционал свободной энергии\(\mathcal{F}(\rho)\) задается в виде:
\begin{equation}
    \mathcal{F}(\rho) = \int_V f(\rho(x)) \, dx + \int_V V(x) \rho(x) \, dx,
\end{equation}
где \(f(\rho)\) — внутренний энергетический член, зависящий от плотности, а \(V(x)\) — внешний потенциал.






\subsection{Оптимизация графа}

Задача оптимального транспорта (Optimal Transport)\cite{villani2009optimal} является одной из ключевых  
в области теории вероятностей и машинного обучения.
Она представляет собой проблему определения оптимального способа перемещения вероятностной массы из одной 
распределенной системы в другую с минимальными затратами или стоимостью. Формально задача состоит в составлении 
транспортного плана \( T: \mathcal{X} \rightarrow \mathcal{Y} \), 
который переводит распределение \( \mu \subset \mathcal{X}\) в распределение \( \nu \subset \mathcal{Y} \), минимизируя некоторую функцию стоимости. 
Функция стоимости $c: \mathcal{X} \times \mathcal{Y} \rightarrow \mathbb{R}$ обычно является мерой сходства между элементами из \( \mathcal{X} \) и \( \mathcal{Y} \), 
такой как квадрат расстояния. 

\textit{Определение} (Монже): \textbf{Оптимальный транспорт} по Монже вводится путем рассмотрения вероятностных 
распределений \( \mu \) и \( \nu \) на метрических пространствах \( \mathcal{X} \) и \( \mathcal{Y} \):
\begin{equation}
    \inf_{\gamma \in \Pi(\mu, \nu)} \int_{\mathcal{X} \times \mathcal{Y}} c(x,y) \, d\gamma(x,y),
\end{equation}
где \( \Pi(\mu, \nu) \) обозначает множество всех возможных совместных распределений 
\( \gamma \) на \( \mathcal{X} \times \mathcal{Y} \) с фиксированными маргинальными
распределениями \( \mu \) и \( \nu \), а \( c(x,y) \) — функция стоимости перевозки массы из \( x \) в \( y \).

\textit{Определение} (Канторович): \textbf{Оптимальный транспорт} по Канторовичу  вводится через потенциал $\phi$, 
который минимизирует функционал стоимости:
\begin{equation}
    \inf_{\phi} \left( \int_{\mathcal{X}} \phi(x) \, d\mu(x) + \int_{\mathcal{Y}} \psi(y) \, d\nu(y) \right),
\end{equation}
где \( \psi \) — обратная функция к \( \phi \). Таким образом, 
отображение \( T: \mathcal{X} \rightarrow \mathcal{Y} \) вытекает из градиента потенциала.

Заметим, что постановка Канторовича обобщает постановку Монже\ref{monge_vs_kantarovich}. В отличие от постановки Монже 
оптимальный транспорт по Канторовичу допускает распределение вероятностной массы в непрерывном случае.

\begin{figure}[h]
    \centering
    \includegraphics[width=0.7\textwidth]{assets/math/transport/optimal_transport.excalidraw.png}
    \caption{Различие в подходе по Монже и Кантаровичу. В постановке Канторовича задача релаксирует до 
    непрерывного распределения}
    \label{monge_vs_kantarovich}
\end{figure}

Итоговая стоимость оптимального транспортного плана называется метрикой Вассерштейна.

\textit{Определение:} Пусть \((X, d)\) --- метрическое пространство и \(P(X)\) --- множество всех вероятностных мер на \(X\). 
Для двух вероятностных мер \(\mu\) и \(\nu\) на \(X\) \textbf{метрика Вассерштейна} порядка \(p\), где \(p \geq 1\), определяется как:
\begin{equation}
    W_p(\mu, \nu) = \left( \inf_{\gamma \in \Gamma(\mu, \nu)} \int_{X \times X} d(x, y)^p \, d\gamma(x, y) \right)^{1/p},
\end{equation}
где \(\Gamma(\mu, \nu)\) — множество всех сопряжённых мер \(\gamma\) на \(X \times X\) с маргиналами \(\mu\) и \(\nu\).

Метрика имеет практическое применение для задач физики, биологии и машинного обучения, поскольку задает 
дифференцируемую разность между распределениями.

\textit{Определение:} \textbf{Метрическая производная} кривой $\rho_t,t \in [0,T]$ в вероятностном 
пространстве $\mathcal{P}_2(\mathbb{R}^N)$ запишется как:
\begin{equation}
    |\rho_t'| = \lim_{dt \rightarrow 0} \frac{\mathcal{W}_2(\rho_t, \rho_{t+dt})}{dt}
\end{equation}

Метод оптимального транспорта также активно применяется для анализа стохастических процессов. Базовой моделью, 
описывающей стохастическое движение с смещением, является процесс Ланжевена.

\textit{Определение:} Процесс Ланжевена называется случайнный процесс вида
\begin{equation}
    d X_t = - \nabla \Phi(x) dt + \sqrt{2 \beta^{-1}} d W_t,
\end{equation}
где $\Phi(X)$ --- потенциал задающий снос частицы, $\beta$ - масштаб блуждания и  $d W_t$ - процесс Винера.

\begin{figure}[h]
    \centering
    \includegraphics[width=0.5\textwidth]{assets/math/transport/fokker-plank.excalidraw.png}
    \caption{Эволюция вероятностной массы в уравнение Ланжевена}
    \label{opt_transport}
\end{figure}

Стохастическое усреднение процесса Ланжевена можно описать с помощью уравнения Колмогорова-Фоккера-Планка, 
задающего эволюцию вероятностной массы в дифференциальной форме  $\rho_t(x)$:
\begin{equation}
    \frac{\partial \rho_t}{\partial t} = \text{div}(\nabla \Phi(x) \rho_t) + \beta^{-1} \Delta \rho_t.
\end{equation}

Для естественной работы в энергетических постановках водится функционал, задающий коэффициента сноса с потенциалом $\Phi$.
Таким образом, исходное уравнение можно переписать в вариационной постановке \ref{variation_fp}.

\textit{Определение:} \textbf{Функционал Фоккера-Планка} для распределения $\rho$ записывается как: 
\begin{equation}
    \mathcal{F}_{FP}(\rho) = \int  \Phi(x) d\rho(x) + \beta^{-1} \int \log \rho(x) d \rho(x).
\end{equation}
\begin{figure}[h]
    \centering
    \includegraphics[width=0.5\textwidth]{assets/math/transport/functional.excalidraw.png}
    \caption{Визуализация постановки уравнения Фоккера-Планка в вариационной форме}
    \label{variation_fp}
\end{figure}
Научная группа Йордана-Кинана-Отто в работе \cite{jordan1998variational} показала, что маргинальные вероятностные
меры процесса Ланжевена подчиняются уравнению градиентного потока Вассерштейна относительно функционала Фоккера-Планка.

\textit{Определение:}  \textbf{Схема Йордана-Кинана-Отто} задает правило обновления уравнения вероятности в виде
минимизации функционала энергии и расстояния:
\begin{equation}
    \rho^{n+1} = \underset{\rho}{\operatorname{argmin}} \left( \frac{1}{2\tau} W_2^2(\rho, \rho^n) + \mathcal{F}(\rho) \right),
\end{equation}
где:\begin{itemize}
    \item \(\tau > 0\) --- шаг по времени;
    \item \(W_2(\rho, \rho^n)\) --- метрика Вассерштейна 2-го порядка между плотностями \(\rho\) и \(\rho^n\);
    \item  \(\mathcal{F}(\rho)\) --- функционал свободной энергии, который может включать в себя энтропийный член и 
    потенциальную энергию системы.
\end{itemize}
Функционал свободной энергии\(\mathcal{F}(\rho)\) задается в виде:
\begin{equation}
    \mathcal{F}(\rho) = \int_V f(\rho(x)) \, dx + \int_V V(x) \rho(x) \, dx,
\end{equation}
где \(f(\rho)\) — внутренний энергетический член, зависящий от плотности, а \(V(x)\) — внешний потенциал.





\subsection{Оптимизация графа}

Задача оптимального транспорта (Optimal Transport)\cite{villani2009optimal} является одной из ключевых  
в области теории вероятностей и машинного обучения.
Она представляет собой проблему определения оптимального способа перемещения вероятностной массы из одной 
распределенной системы в другую с минимальными затратами или стоимостью. Формально задача состоит в составлении 
транспортного плана \( T: \mathcal{X} \rightarrow \mathcal{Y} \), 
который переводит распределение \( \mu \subset \mathcal{X}\) в распределение \( \nu \subset \mathcal{Y} \), минимизируя некоторую функцию стоимости. 
Функция стоимости $c: \mathcal{X} \times \mathcal{Y} \rightarrow \mathbb{R}$ обычно является мерой сходства между элементами из \( \mathcal{X} \) и \( \mathcal{Y} \), 
такой как квадрат расстояния. 

\textit{Определение} (Монже): \textbf{Оптимальный транспорт} по Монже вводится путем рассмотрения вероятностных 
распределений \( \mu \) и \( \nu \) на метрических пространствах \( \mathcal{X} \) и \( \mathcal{Y} \):
\begin{equation}
    \inf_{\gamma \in \Pi(\mu, \nu)} \int_{\mathcal{X} \times \mathcal{Y}} c(x,y) \, d\gamma(x,y),
\end{equation}
где \( \Pi(\mu, \nu) \) обозначает множество всех возможных совместных распределений 
\( \gamma \) на \( \mathcal{X} \times \mathcal{Y} \) с фиксированными маргинальными
распределениями \( \mu \) и \( \nu \), а \( c(x,y) \) — функция стоимости перевозки массы из \( x \) в \( y \).

\textit{Определение} (Канторович): \textbf{Оптимальный транспорт} по Канторовичу  вводится через потенциал $\phi$, 
который минимизирует функционал стоимости:
\begin{equation}
    \inf_{\phi} \left( \int_{\mathcal{X}} \phi(x) \, d\mu(x) + \int_{\mathcal{Y}} \psi(y) \, d\nu(y) \right),
\end{equation}
где \( \psi \) — обратная функция к \( \phi \). Таким образом, 
отображение \( T: \mathcal{X} \rightarrow \mathcal{Y} \) вытекает из градиента потенциала.

Заметим, что постановка Канторовича обобщает постановку Монже\ref{monge_vs_kantarovich}. В отличие от постановки Монже 
оптимальный транспорт по Канторовичу допускает распределение вероятностной массы в непрерывном случае.

\begin{figure}[h]
    \centering
    \includegraphics[width=0.7\textwidth]{assets/math/transport/optimal_transport.excalidraw.png}
    \caption{Различие в подходе по Монже и Кантаровичу. В постановке Канторовича задача релаксирует до 
    непрерывного распределения}
    \label{monge_vs_kantarovich}
\end{figure}

Итоговая стоимость оптимального транспортного плана называется метрикой Вассерштейна.

\textit{Определение:} Пусть \((X, d)\) --- метрическое пространство и \(P(X)\) --- множество всех вероятностных мер на \(X\). 
Для двух вероятностных мер \(\mu\) и \(\nu\) на \(X\) \textbf{метрика Вассерштейна} порядка \(p\), где \(p \geq 1\), определяется как:
\begin{equation}
    W_p(\mu, \nu) = \left( \inf_{\gamma \in \Gamma(\mu, \nu)} \int_{X \times X} d(x, y)^p \, d\gamma(x, y) \right)^{1/p},
\end{equation}
где \(\Gamma(\mu, \nu)\) — множество всех сопряжённых мер \(\gamma\) на \(X \times X\) с маргиналами \(\mu\) и \(\nu\).

Метрика имеет практическое применение для задач физики, биологии и машинного обучения, поскольку задает 
дифференцируемую разность между распределениями.

\textit{Определение:} \textbf{Метрическая производная} кривой $\rho_t,t \in [0,T]$ в вероятностном 
пространстве $\mathcal{P}_2(\mathbb{R}^N)$ запишется как:
\begin{equation}
    |\rho_t'| = \lim_{dt \rightarrow 0} \frac{\mathcal{W}_2(\rho_t, \rho_{t+dt})}{dt}
\end{equation}

Метод оптимального транспорта также активно применяется для анализа стохастических процессов. Базовой моделью, 
описывающей стохастическое движение с смещением, является процесс Ланжевена.

\textit{Определение:} Процесс Ланжевена называется случайнный процесс вида
\begin{equation}
    d X_t = - \nabla \Phi(x) dt + \sqrt{2 \beta^{-1}} d W_t,
\end{equation}
где $\Phi(X)$ --- потенциал задающий снос частицы, $\beta$ - масштаб блуждания и  $d W_t$ - процесс Винера.

\begin{figure}[h]
    \centering
    \includegraphics[width=0.5\textwidth]{assets/math/transport/fokker-plank.excalidraw.png}
    \caption{Эволюция вероятностной массы в уравнение Ланжевена}
    \label{opt_transport}
\end{figure}

Стохастическое усреднение процесса Ланжевена можно описать с помощью уравнения Колмогорова-Фоккера-Планка, 
задающего эволюцию вероятностной массы в дифференциальной форме  $\rho_t(x)$:
\begin{equation}
    \frac{\partial \rho_t}{\partial t} = \text{div}(\nabla \Phi(x) \rho_t) + \beta^{-1} \Delta \rho_t.
\end{equation}

Для естественной работы в энергетических постановках водится функционал, задающий коэффициента сноса с потенциалом $\Phi$.
Таким образом, исходное уравнение можно переписать в вариационной постановке \ref{variation_fp}.

\textit{Определение:} \textbf{Функционал Фоккера-Планка} для распределения $\rho$ записывается как: 
\begin{equation}
    \mathcal{F}_{FP}(\rho) = \int  \Phi(x) d\rho(x) + \beta^{-1} \int \log \rho(x) d \rho(x).
\end{equation}
\begin{figure}[h]
    \centering
    \includegraphics[width=0.5\textwidth]{assets/math/transport/functional.excalidraw.png}
    \caption{Визуализация постановки уравнения Фоккера-Планка в вариационной форме}
    \label{variation_fp}
\end{figure}
Научная группа Йордана-Кинана-Отто в работе \cite{jordan1998variational} показала, что маргинальные вероятностные
меры процесса Ланжевена подчиняются уравнению градиентного потока Вассерштейна относительно функционала Фоккера-Планка.

\textit{Определение:}  \textbf{Схема Йордана-Кинана-Отто} задает правило обновления уравнения вероятности в виде
минимизации функционала энергии и расстояния:
\begin{equation}
    \rho^{n+1} = \underset{\rho}{\operatorname{argmin}} \left( \frac{1}{2\tau} W_2^2(\rho, \rho^n) + \mathcal{F}(\rho) \right),
\end{equation}
где:\begin{itemize}
    \item \(\tau > 0\) --- шаг по времени;
    \item \(W_2(\rho, \rho^n)\) --- метрика Вассерштейна 2-го порядка между плотностями \(\rho\) и \(\rho^n\);
    \item  \(\mathcal{F}(\rho)\) --- функционал свободной энергии, который может включать в себя энтропийный член и 
    потенциальную энергию системы.
\end{itemize}
Функционал свободной энергии\(\mathcal{F}(\rho)\) задается в виде:
\begin{equation}
    \mathcal{F}(\rho) = \int_V f(\rho(x)) \, dx + \int_V V(x) \rho(x) \, dx,
\end{equation}
где \(f(\rho)\) — внутренний энергетический член, зависящий от плотности, а \(V(x)\) — внешний потенциал.





\section{Обработка естественного языка}

Порождающие модели современное и быстро развивающие направление работы с данными, направленное на их
создание и получение вероятностной массы. Ключевыми достижениями в дисциплине были \begin{enumerate}
    \item порождающие грамматики \cite{chomsky2002syntactic}
    \item графические вероятностные модели \cite{pearl1988probabilistic}
    \item состязательные порождающие модели \cite{goodfellow2020generative}
    \item диффузионные порождающие модели \cite{song2020score}
\end{enumerate}
Такие модели также представляют интерес для предметных исследователей, поскольку позволяют аналитически изучать
элементарные механизмы задания графа.

Порождающие модели задают совместное распределение наблюдаемого объекта $x$ и его черт $y$ -  $p(x,y)$. В этом заключается 
ключевое различие между порождающими и дискриминирующими моделями $p(y|x)$ \ref{discr_vs_gen}.

\begin{figure}[h]
    \centering
    \includegraphics[width=0.5\textwidth]{assets/ml/generation/stable_diffusion.png}
    \caption{Архитектура современной модели Stable Diffusion}
    \label{discr_vs_gen}
\end{figure}

Порождающие модели, используют параметрические модели $p_\theta$ для аппроксимации истинных функций распределений на наборе обучающих данных.
Выбор параметрической функции аппроксимации, как правило, зависит от числа примеров в коллекции данных. Для больших наборов данных как правило используют нейросети.
Простейшим видом порождающей модели является авторегрессионная модель модель, использующая предшествующий контекст для предсказания следующего элемента.

\textit{Определение } \textbf{Авторегрессионные модели} представляют собой класс порождающих моделей,
с вычислимой  вероятностью, выполняющие генерацию через цепочку последовательных преобразований \begin{equation}
    p(x^{(1)},\dots,x^{(t)}) = \prod_{t=1}^T p_\theta(x^{(t)}|x^{(1)},\dots,x^{(t-1)})
\end{equation}

Как правило, авторегресионные модели используются для генерации последовательностей, временных рядов и текста.
Класс плохо применим к данные не подлежащие однозначному упорядочиванию или с неравномерным шагом. 

Для оценки разницы между вероятностными распределениями используются \textbf{дивергенции} с набором правил \begin{enumerate}
    \item $\forall p,q \in M \rightarrow D(p,q) \ge 0$  
    \item $p=q \leftrightarrow D(p,q) = 0$
    \item $\forall p \rightarrow D(p,p+dp)$ положительно определенная квадратичная фоорма  
\end{enumerate}

В отличие  от метрики дивергенции не обязаны быть симметричны.ьНа практике как правило используются специальный класс $f$-дивергенций, задающихся
через матожидание.

\textit{Определение} $f$-дивергенцией называется выпуклая функция, удовлетворяющая равенству $f(1)=0$.
$$
    D_f{\pi \parallel \rho} = \mathrm E_{\rho(x)} f\left(\frac{\pi(x)}{\rho(x)}\right)
$$

Семейство $f$-дивергенций включает функции \begin{enumerate}
    \item дивергенция Кульбака-Лейбнера $u logu $
    \item обратная дивергенция Кульбака-Лейбнера $-ln u$
    \item дивергенция Йенсена-Шэннона  $\frac{1}{2}\left(u ln u - (u+1) ln(\frac{u+1}{2})\right)$
\end{enumerate}


Нижней вариационной оценкой называется техника максимизации подпирающей границы параметрического распределения $p(\mathbf{x},\mathbf{z})$ вторым $q(\mathbf{x},\mathbf{z})$,
где переменная $\mathbf{z}$ называется скрытой . В аналитической форме нижняя граница записывается как 
$$
    \mathcal{L}(\phi,\theta;x) = \mathbb{E}_{z \sim q_\phi(z|x)} \left[\ln \frac{p_{\theta}(x,z)}{q_{\phi}(z|x)}\right],
$$

Перепишем через KL-дивергенцию:
\begin{equation}
    \begin{aligned}
        & D_\text{KL}( q_\phi(\mathbf{z}\vert\mathbf{x}) \| p_\theta(\mathbf{z}\vert\mathbf{x}) ) & \\
        &=\int q_\phi(\mathbf{z} \vert \mathbf{x}) \log\frac{q_\phi(\mathbf{z} \vert \mathbf{x})}{p_\theta(\mathbf{z} \vert \mathbf{x})} d\mathbf{z} & \\
        &=\int q_\phi(\mathbf{z} \vert \mathbf{x}) \log\frac{q_\phi(\mathbf{z} \vert \mathbf{x})p_\theta(\mathbf{x})}{p_\theta(\mathbf{z}, \mathbf{x})} d\mathbf{z}\\
        &=\int q_\phi(\mathbf{z} \vert \mathbf{x}) \big( \log p_\theta(\mathbf{x}) + \log\frac{q_\phi(\mathbf{z} \vert \mathbf{x})}{p_\theta(\mathbf{z}, \mathbf{x})} \big) d\mathbf{z} & \\
        &=\log p_\theta(\mathbf{x}) + \int q_\phi(\mathbf{z} \vert \mathbf{x})\log\frac{q_\phi(\mathbf{z} \vert \mathbf{x})}{p_\theta(\mathbf{z}, \mathbf{x})} d\mathbf{z} \\
        &=\log p_\theta(\mathbf{x}) + \int q_\phi(\mathbf{z} \vert \mathbf{x})\log\frac{q_\phi(\mathbf{z} \vert \mathbf{x})}{p_\theta(\mathbf{x}\vert\mathbf{z})p_\theta(\mathbf{z})} d\mathbf{z} \\
        &=\log p_\theta(\mathbf{x}) + \mathbb{E}_{\mathbf{z}\sim q_\phi(\mathbf{z} \vert \mathbf{x})}[\log \frac{q_\phi(\mathbf{z} \vert \mathbf{x})}{p_\theta(\mathbf{z})} - \log p_\theta(\mathbf{x} \vert \mathbf{z})] &\\
        &=\log p_\theta(\mathbf{x}) + D_\text{KL}(q_\phi(\mathbf{z}\vert\mathbf{x}) \| p_\theta(\mathbf{z})) - \mathbb{E}_{\mathbf{z}\sim q_\phi(\mathbf{z}\vert\mathbf{x})}\log p_\theta(\mathbf{x}\vert\mathbf{z}) &
    \end{aligned}
\end{equation}

Следовательно:
\begin{equation}
    \log p_\theta(\mathbf{x}) - D_\text{KL}( q_\phi(\mathbf{z}\vert\mathbf{x}) \| p_\theta(\mathbf{z}\vert\mathbf{x}) ) = \mathbb{E}_{\mathbf{z}\sim q_\phi(\mathbf{z}\vert\mathbf{x})}\log p_\theta(\mathbf{x}\vert\mathbf{z}) - D_\text{KL}(q_\phi(\mathbf{z}\vert\mathbf{x}) \| p_\theta(\mathbf{z}))
\end{equation}
    
Таким образом, из неравенства Йенсена получаем: 
$$
    \ln p_\theta(x) \le \mathcal{L}(\phi,\theta;x)
$$
Базовым алгоритмом оптимизации вариационных моделей является EM-алгоритм, состоящий из последовательного обновления
скрытых представлений и максимизации правдоподобия с заданным параметрами.

\textit{Определение} \textbf{EM-алгоритм} - алгоритм для нахождения оценок
максимального правдоподобия параметров  вероятностных моделей с скрытыми переменными $\theta$.

Аналитически шаги алгоритма записываются как: \begin{itemize}
    \item расчет матожидания при заданном на шаге $t$ параметре $\theta^{(t)}$.
    Шаг обноez \begin{equation}
        Q(\theta| \theta^{(t)}) = \mathbb{E}_{\mathbf{Z} \sim p(\mathbf{Z}|X,\theta^{(t)})} \left[ \log p(\mathbf{X},\mathbf{Z}|\mathbf{\theta})\right]
    \end{equation}
    \item максимизации полученного выражения для нового шага $\mathbf{\theta}^{(t+1)}$: \begin{equation}
        \mathbf{\theta}^{(t+1)} = \text{arg} \max_{\theta} Q(\mathbf{\theta}|\theta^{(t)})
    \end{equation}  
\end{itemize}






\subsection{Оптимизация графа}




Байесовы сети применяются в причинно-следственном анализе.

Итоговая модель представляет собой статистическую модель явления, которую можно использовать для
управления и анализа системой. 
Описанный подход называется \textit{вариационным причиннно-следственным выводом}.

В теории причинно-следственного анализа введенную величину
свидетельство. (\texit{англ.} Evidence).

Максимизация свидетельства
позволяет выбирать модели согласно принципу бритву Оккама, исключая
параметры не вносящие существенного смысла для модели. Для расчета свидетельства необходимо выполнить маргинализацию по параметрам
модели $\int P(X, \theta) d\theta$. В общем случае задача принадлежит 
NP-классу сложности, рассчитывается за время, экспоненциально зависящее
от числа параметров. 

\texit{Определение} События $A$ и $B$ cчитаются независимыми в условиях:
\begin{equation}
    P(A \cup B) = P(A) \times P(B)
\end{equation}

На практике в системах подлинная независимость случайных величин $x$ и $y$
$x \perp y$ встречается не всегда. Чаще достижима условная независимость, наблюдаемая при
фиксации третьего фактора $z$.

\texit{Определение} События $A$ и $B$ cчитаются условно независимыми 
для заданного события $C$ в условиях:
\begin{equation}
    P(A \cup B |C) = P(A|C) \times P(B|C)
\end{equation}

\begin{figure}[h]
    \centering
    \includegraphics[width=0.5\textwidth]{assets/math/discrete/bayes_net.excalidraw.png}
    \caption{Посредник(\textit{англ.} mediator), общий предок(\textit{англ.} cofounder), 
 общий родственник \textit{англ.} collider) }
    \label{discr_vs_gen}
\end{figure}

Виды вариационного вывода можно разделить на три ключевых направлениях: \begin{itemize}
    \item прогнозирование
    \item обратный - объяснение причины на основании
\end{itemize}

Вывод выполняется путем маргинализации распределения по 



Многокомпонетный осложняется неоднозначной трактовкой исхода. 
Для анализа сложных систем предпочтителен однофакторный анализ, выполняющий
для его выполнения необходимо перекрыть потоки зависимостей (\textit{англ.} dependency flow) от
прочих переменных.

\texit{Определение} Интервенцией называется изменением 

\begin{figure}[h]
    \centering
    \includegraphics[width=0.5\textwidth]{assets/math/discrete/dep_flow.excalidraw.png}
    \caption{Интервенция}
    \label{discr_vs_gen}
\end{figure}





\texit{Определение} События $A$ и $B$ cчитаются независимыми в условиях:
\begin{equation}
    P(A \cup B) = P(A) \times P(B)
\end{equation}





\begin{figure}[h]
    \centering
    \includegraphics[width=0.5\textwidth]{assets/math/discrete/bayes_net.excalidraw.png}
    \caption{Посредник(\textit{англ.} mediator), общий предок(\textit{англ.} cofounder), 
 общий родственник \textit{англ.} collider) }
    \label{discr_vs_gen}
\end{figure}




Условная независимость позволяет 
$x \perp y$ встречается не всегда.




\subsection{Оптимизация графа}

\textit{Определение} Графической вероятностной моделью





Принципиально выполняется задача факторизации 


Пробалистические языки программирования.





\subsection{Оптимизация графа}

Задача оптимального транспорта (Optimal Transport)\cite{villani2009optimal} является одной из ключевых  
в области теории вероятностей и машинного обучения.
Она представляет собой проблему определения оптимального способа перемещения вероятностной массы из одной 
распределенной системы в другую с минимальными затратами или стоимостью. Формально задача состоит в составлении 
транспортного плана \( T: \mathcal{X} \rightarrow \mathcal{Y} \), 
который переводит распределение \( \mu \subset \mathcal{X}\) в распределение \( \nu \subset \mathcal{Y} \), минимизируя некоторую функцию стоимости. 
Функция стоимости $c: \mathcal{X} \times \mathcal{Y} \rightarrow \mathbb{R}$ обычно является мерой сходства между элементами из \( \mathcal{X} \) и \( \mathcal{Y} \), 
такой как квадрат расстояния. 

\textit{Определение} (Монже): \textbf{Оптимальный транспорт} по Монже вводится путем рассмотрения вероятностных 
распределений \( \mu \) и \( \nu \) на метрических пространствах \( \mathcal{X} \) и \( \mathcal{Y} \):
\begin{equation}
    \inf_{\gamma \in \Pi(\mu, \nu)} \int_{\mathcal{X} \times \mathcal{Y}} c(x,y) \, d\gamma(x,y),
\end{equation}
где \( \Pi(\mu, \nu) \) обозначает множество всех возможных совместных распределений 
\( \gamma \) на \( \mathcal{X} \times \mathcal{Y} \) с фиксированными маргинальными
распределениями \( \mu \) и \( \nu \), а \( c(x,y) \) — функция стоимости перевозки массы из \( x \) в \( y \).

\textit{Определение} (Канторович): \textbf{Оптимальный транспорт} по Канторовичу  вводится через потенциал $\phi$, 
который минимизирует функционал стоимости:
\begin{equation}
    \inf_{\phi} \left( \int_{\mathcal{X}} \phi(x) \, d\mu(x) + \int_{\mathcal{Y}} \psi(y) \, d\nu(y) \right),
\end{equation}
где \( \psi \) — обратная функция к \( \phi \). Таким образом, 
отображение \( T: \mathcal{X} \rightarrow \mathcal{Y} \) вытекает из градиента потенциала.

Заметим, что постановка Канторовича обобщает постановку Монже\ref{monge_vs_kantarovich}. В отличие от постановки Монже 
оптимальный транспорт по Канторовичу допускает распределение вероятностной массы в непрерывном случае.

\begin{figure}[h]
    \centering
    \includegraphics[width=0.7\textwidth]{assets/math/transport/optimal_transport.excalidraw.png}
    \caption{Различие в подходе по Монже и Кантаровичу. В постановке Канторовича задача релаксирует до 
    непрерывного распределения}
    \label{monge_vs_kantarovich}
\end{figure}

Итоговая стоимость оптимального транспортного плана называется метрикой Вассерштейна.

\textit{Определение:} Пусть \((X, d)\) --- метрическое пространство и \(P(X)\) --- множество всех вероятностных мер на \(X\). 
Для двух вероятностных мер \(\mu\) и \(\nu\) на \(X\) \textbf{метрика Вассерштейна} порядка \(p\), где \(p \geq 1\), определяется как:
\begin{equation}
    W_p(\mu, \nu) = \left( \inf_{\gamma \in \Gamma(\mu, \nu)} \int_{X \times X} d(x, y)^p \, d\gamma(x, y) \right)^{1/p},
\end{equation}
где \(\Gamma(\mu, \nu)\) — множество всех сопряжённых мер \(\gamma\) на \(X \times X\) с маргиналами \(\mu\) и \(\nu\).

Метрика имеет практическое применение для задач физики, биологии и машинного обучения, поскольку задает 
дифференцируемую разность между распределениями.

\textit{Определение:} \textbf{Метрическая производная} кривой $\rho_t,t \in [0,T]$ в вероятностном 
пространстве $\mathcal{P}_2(\mathbb{R}^N)$ запишется как:
\begin{equation}
    |\rho_t'| = \lim_{dt \rightarrow 0} \frac{\mathcal{W}_2(\rho_t, \rho_{t+dt})}{dt}
\end{equation}

Метод оптимального транспорта также активно применяется для анализа стохастических процессов. Базовой моделью, 
описывающей стохастическое движение с смещением, является процесс Ланжевена.

\textit{Определение:} Процесс Ланжевена называется случайнный процесс вида
\begin{equation}
    d X_t = - \nabla \Phi(x) dt + \sqrt{2 \beta^{-1}} d W_t,
\end{equation}
где $\Phi(X)$ --- потенциал задающий снос частицы, $\beta$ - масштаб блуждания и  $d W_t$ - процесс Винера.

\begin{figure}[h]
    \centering
    \includegraphics[width=0.5\textwidth]{assets/math/transport/fokker-plank.excalidraw.png}
    \caption{Эволюция вероятностной массы в уравнение Ланжевена}
    \label{opt_transport}
\end{figure}

Стохастическое усреднение процесса Ланжевена можно описать с помощью уравнения Колмогорова-Фоккера-Планка, 
задающего эволюцию вероятностной массы в дифференциальной форме  $\rho_t(x)$:
\begin{equation}
    \frac{\partial \rho_t}{\partial t} = \text{div}(\nabla \Phi(x) \rho_t) + \beta^{-1} \Delta \rho_t.
\end{equation}

Для естественной работы в энергетических постановках водится функционал, задающий коэффициента сноса с потенциалом $\Phi$.
Таким образом, исходное уравнение можно переписать в вариационной постановке \ref{variation_fp}.

\textit{Определение:} \textbf{Функционал Фоккера-Планка} для распределения $\rho$ записывается как: 
\begin{equation}
    \mathcal{F}_{FP}(\rho) = \int  \Phi(x) d\rho(x) + \beta^{-1} \int \log \rho(x) d \rho(x).
\end{equation}
\begin{figure}[h]
    \centering
    \includegraphics[width=0.5\textwidth]{assets/math/transport/functional.excalidraw.png}
    \caption{Визуализация постановки уравнения Фоккера-Планка в вариационной форме}
    \label{variation_fp}
\end{figure}
Научная группа Йордана-Кинана-Отто в работе \cite{jordan1998variational} показала, что маргинальные вероятностные
меры процесса Ланжевена подчиняются уравнению градиентного потока Вассерштейна относительно функционала Фоккера-Планка.

\textit{Определение:}  \textbf{Схема Йордана-Кинана-Отто} задает правило обновления уравнения вероятности в виде
минимизации функционала энергии и расстояния:
\begin{equation}
    \rho^{n+1} = \underset{\rho}{\operatorname{argmin}} \left( \frac{1}{2\tau} W_2^2(\rho, \rho^n) + \mathcal{F}(\rho) \right),
\end{equation}
где:\begin{itemize}
    \item \(\tau > 0\) --- шаг по времени;
    \item \(W_2(\rho, \rho^n)\) --- метрика Вассерштейна 2-го порядка между плотностями \(\rho\) и \(\rho^n\);
    \item  \(\mathcal{F}(\rho)\) --- функционал свободной энергии, который может включать в себя энтропийный член и 
    потенциальную энергию системы.
\end{itemize}
Функционал свободной энергии\(\mathcal{F}(\rho)\) задается в виде:
\begin{equation}
    \mathcal{F}(\rho) = \int_V f(\rho(x)) \, dx + \int_V V(x) \rho(x) \, dx,
\end{equation}
где \(f(\rho)\) — внутренний энергетический член, зависящий от плотности, а \(V(x)\) — внешний потенциал.






\subsection{Оптимизация графа}

Задача оптимального транспорта (Optimal Transport)\cite{villani2009optimal} является одной из ключевых  
в области теории вероятностей и машинного обучения.
Она представляет собой проблему определения оптимального способа перемещения вероятностной массы из одной 
распределенной системы в другую с минимальными затратами или стоимостью. Формально задача состоит в составлении 
транспортного плана \( T: \mathcal{X} \rightarrow \mathcal{Y} \), 
который переводит распределение \( \mu \subset \mathcal{X}\) в распределение \( \nu \subset \mathcal{Y} \), минимизируя некоторую функцию стоимости. 
Функция стоимости $c: \mathcal{X} \times \mathcal{Y} \rightarrow \mathbb{R}$ обычно является мерой сходства между элементами из \( \mathcal{X} \) и \( \mathcal{Y} \), 
такой как квадрат расстояния. 

\textit{Определение} (Монже): \textbf{Оптимальный транспорт} по Монже вводится путем рассмотрения вероятностных 
распределений \( \mu \) и \( \nu \) на метрических пространствах \( \mathcal{X} \) и \( \mathcal{Y} \):
\begin{equation}
    \inf_{\gamma \in \Pi(\mu, \nu)} \int_{\mathcal{X} \times \mathcal{Y}} c(x,y) \, d\gamma(x,y),
\end{equation}
где \( \Pi(\mu, \nu) \) обозначает множество всех возможных совместных распределений 
\( \gamma \) на \( \mathcal{X} \times \mathcal{Y} \) с фиксированными маргинальными
распределениями \( \mu \) и \( \nu \), а \( c(x,y) \) — функция стоимости перевозки массы из \( x \) в \( y \).

\textit{Определение} (Канторович): \textbf{Оптимальный транспорт} по Канторовичу  вводится через потенциал $\phi$, 
который минимизирует функционал стоимости:
\begin{equation}
    \inf_{\phi} \left( \int_{\mathcal{X}} \phi(x) \, d\mu(x) + \int_{\mathcal{Y}} \psi(y) \, d\nu(y) \right),
\end{equation}
где \( \psi \) — обратная функция к \( \phi \). Таким образом, 
отображение \( T: \mathcal{X} \rightarrow \mathcal{Y} \) вытекает из градиента потенциала.

Заметим, что постановка Канторовича обобщает постановку Монже\ref{monge_vs_kantarovich}. В отличие от постановки Монже 
оптимальный транспорт по Канторовичу допускает распределение вероятностной массы в непрерывном случае.

\begin{figure}[h]
    \centering
    \includegraphics[width=0.7\textwidth]{assets/math/transport/optimal_transport.excalidraw.png}
    \caption{Различие в подходе по Монже и Кантаровичу. В постановке Канторовича задача релаксирует до 
    непрерывного распределения}
    \label{monge_vs_kantarovich}
\end{figure}

Итоговая стоимость оптимального транспортного плана называется метрикой Вассерштейна.

\textit{Определение:} Пусть \((X, d)\) --- метрическое пространство и \(P(X)\) --- множество всех вероятностных мер на \(X\). 
Для двух вероятностных мер \(\mu\) и \(\nu\) на \(X\) \textbf{метрика Вассерштейна} порядка \(p\), где \(p \geq 1\), определяется как:
\begin{equation}
    W_p(\mu, \nu) = \left( \inf_{\gamma \in \Gamma(\mu, \nu)} \int_{X \times X} d(x, y)^p \, d\gamma(x, y) \right)^{1/p},
\end{equation}
где \(\Gamma(\mu, \nu)\) — множество всех сопряжённых мер \(\gamma\) на \(X \times X\) с маргиналами \(\mu\) и \(\nu\).

Метрика имеет практическое применение для задач физики, биологии и машинного обучения, поскольку задает 
дифференцируемую разность между распределениями.

\textit{Определение:} \textbf{Метрическая производная} кривой $\rho_t,t \in [0,T]$ в вероятностном 
пространстве $\mathcal{P}_2(\mathbb{R}^N)$ запишется как:
\begin{equation}
    |\rho_t'| = \lim_{dt \rightarrow 0} \frac{\mathcal{W}_2(\rho_t, \rho_{t+dt})}{dt}
\end{equation}

Метод оптимального транспорта также активно применяется для анализа стохастических процессов. Базовой моделью, 
описывающей стохастическое движение с смещением, является процесс Ланжевена.

\textit{Определение:} Процесс Ланжевена называется случайнный процесс вида
\begin{equation}
    d X_t = - \nabla \Phi(x) dt + \sqrt{2 \beta^{-1}} d W_t,
\end{equation}
где $\Phi(X)$ --- потенциал задающий снос частицы, $\beta$ - масштаб блуждания и  $d W_t$ - процесс Винера.

\begin{figure}[h]
    \centering
    \includegraphics[width=0.5\textwidth]{assets/math/transport/fokker-plank.excalidraw.png}
    \caption{Эволюция вероятностной массы в уравнение Ланжевена}
    \label{opt_transport}
\end{figure}

Стохастическое усреднение процесса Ланжевена можно описать с помощью уравнения Колмогорова-Фоккера-Планка, 
задающего эволюцию вероятностной массы в дифференциальной форме  $\rho_t(x)$:
\begin{equation}
    \frac{\partial \rho_t}{\partial t} = \text{div}(\nabla \Phi(x) \rho_t) + \beta^{-1} \Delta \rho_t.
\end{equation}

Для естественной работы в энергетических постановках водится функционал, задающий коэффициента сноса с потенциалом $\Phi$.
Таким образом, исходное уравнение можно переписать в вариационной постановке \ref{variation_fp}.

\textit{Определение:} \textbf{Функционал Фоккера-Планка} для распределения $\rho$ записывается как: 
\begin{equation}
    \mathcal{F}_{FP}(\rho) = \int  \Phi(x) d\rho(x) + \beta^{-1} \int \log \rho(x) d \rho(x).
\end{equation}
\begin{figure}[h]
    \centering
    \includegraphics[width=0.5\textwidth]{assets/math/transport/functional.excalidraw.png}
    \caption{Визуализация постановки уравнения Фоккера-Планка в вариационной форме}
    \label{variation_fp}
\end{figure}
Научная группа Йордана-Кинана-Отто в работе \cite{jordan1998variational} показала, что маргинальные вероятностные
меры процесса Ланжевена подчиняются уравнению градиентного потока Вассерштейна относительно функционала Фоккера-Планка.

\textit{Определение:}  \textbf{Схема Йордана-Кинана-Отто} задает правило обновления уравнения вероятности в виде
минимизации функционала энергии и расстояния:
\begin{equation}
    \rho^{n+1} = \underset{\rho}{\operatorname{argmin}} \left( \frac{1}{2\tau} W_2^2(\rho, \rho^n) + \mathcal{F}(\rho) \right),
\end{equation}
где:\begin{itemize}
    \item \(\tau > 0\) --- шаг по времени;
    \item \(W_2(\rho, \rho^n)\) --- метрика Вассерштейна 2-го порядка между плотностями \(\rho\) и \(\rho^n\);
    \item  \(\mathcal{F}(\rho)\) --- функционал свободной энергии, который может включать в себя энтропийный член и 
    потенциальную энергию системы.
\end{itemize}
Функционал свободной энергии\(\mathcal{F}(\rho)\) задается в виде:
\begin{equation}
    \mathcal{F}(\rho) = \int_V f(\rho(x)) \, dx + \int_V V(x) \rho(x) \, dx,
\end{equation}
где \(f(\rho)\) — внутренний энергетический член, зависящий от плотности, а \(V(x)\) — внешний потенциал.





\subsection{Оптимизация графа}

Задача оптимального транспорта (Optimal Transport)\cite{villani2009optimal} является одной из ключевых  
в области теории вероятностей и машинного обучения.
Она представляет собой проблему определения оптимального способа перемещения вероятностной массы из одной 
распределенной системы в другую с минимальными затратами или стоимостью. Формально задача состоит в составлении 
транспортного плана \( T: \mathcal{X} \rightarrow \mathcal{Y} \), 
который переводит распределение \( \mu \subset \mathcal{X}\) в распределение \( \nu \subset \mathcal{Y} \), минимизируя некоторую функцию стоимости. 
Функция стоимости $c: \mathcal{X} \times \mathcal{Y} \rightarrow \mathbb{R}$ обычно является мерой сходства между элементами из \( \mathcal{X} \) и \( \mathcal{Y} \), 
такой как квадрат расстояния. 

\textit{Определение} (Монже): \textbf{Оптимальный транспорт} по Монже вводится путем рассмотрения вероятностных 
распределений \( \mu \) и \( \nu \) на метрических пространствах \( \mathcal{X} \) и \( \mathcal{Y} \):
\begin{equation}
    \inf_{\gamma \in \Pi(\mu, \nu)} \int_{\mathcal{X} \times \mathcal{Y}} c(x,y) \, d\gamma(x,y),
\end{equation}
где \( \Pi(\mu, \nu) \) обозначает множество всех возможных совместных распределений 
\( \gamma \) на \( \mathcal{X} \times \mathcal{Y} \) с фиксированными маргинальными
распределениями \( \mu \) и \( \nu \), а \( c(x,y) \) — функция стоимости перевозки массы из \( x \) в \( y \).

\textit{Определение} (Канторович): \textbf{Оптимальный транспорт} по Канторовичу  вводится через потенциал $\phi$, 
который минимизирует функционал стоимости:
\begin{equation}
    \inf_{\phi} \left( \int_{\mathcal{X}} \phi(x) \, d\mu(x) + \int_{\mathcal{Y}} \psi(y) \, d\nu(y) \right),
\end{equation}
где \( \psi \) — обратная функция к \( \phi \). Таким образом, 
отображение \( T: \mathcal{X} \rightarrow \mathcal{Y} \) вытекает из градиента потенциала.

Заметим, что постановка Канторовича обобщает постановку Монже\ref{monge_vs_kantarovich}. В отличие от постановки Монже 
оптимальный транспорт по Канторовичу допускает распределение вероятностной массы в непрерывном случае.

\begin{figure}[h]
    \centering
    \includegraphics[width=0.7\textwidth]{assets/math/transport/optimal_transport.excalidraw.png}
    \caption{Различие в подходе по Монже и Кантаровичу. В постановке Канторовича задача релаксирует до 
    непрерывного распределения}
    \label{monge_vs_kantarovich}
\end{figure}

Итоговая стоимость оптимального транспортного плана называется метрикой Вассерштейна.

\textit{Определение:} Пусть \((X, d)\) --- метрическое пространство и \(P(X)\) --- множество всех вероятностных мер на \(X\). 
Для двух вероятностных мер \(\mu\) и \(\nu\) на \(X\) \textbf{метрика Вассерштейна} порядка \(p\), где \(p \geq 1\), определяется как:
\begin{equation}
    W_p(\mu, \nu) = \left( \inf_{\gamma \in \Gamma(\mu, \nu)} \int_{X \times X} d(x, y)^p \, d\gamma(x, y) \right)^{1/p},
\end{equation}
где \(\Gamma(\mu, \nu)\) — множество всех сопряжённых мер \(\gamma\) на \(X \times X\) с маргиналами \(\mu\) и \(\nu\).

Метрика имеет практическое применение для задач физики, биологии и машинного обучения, поскольку задает 
дифференцируемую разность между распределениями.

\textit{Определение:} \textbf{Метрическая производная} кривой $\rho_t,t \in [0,T]$ в вероятностном 
пространстве $\mathcal{P}_2(\mathbb{R}^N)$ запишется как:
\begin{equation}
    |\rho_t'| = \lim_{dt \rightarrow 0} \frac{\mathcal{W}_2(\rho_t, \rho_{t+dt})}{dt}
\end{equation}

Метод оптимального транспорта также активно применяется для анализа стохастических процессов. Базовой моделью, 
описывающей стохастическое движение с смещением, является процесс Ланжевена.

\textit{Определение:} Процесс Ланжевена называется случайнный процесс вида
\begin{equation}
    d X_t = - \nabla \Phi(x) dt + \sqrt{2 \beta^{-1}} d W_t,
\end{equation}
где $\Phi(X)$ --- потенциал задающий снос частицы, $\beta$ - масштаб блуждания и  $d W_t$ - процесс Винера.

\begin{figure}[h]
    \centering
    \includegraphics[width=0.5\textwidth]{assets/math/transport/fokker-plank.excalidraw.png}
    \caption{Эволюция вероятностной массы в уравнение Ланжевена}
    \label{opt_transport}
\end{figure}

Стохастическое усреднение процесса Ланжевена можно описать с помощью уравнения Колмогорова-Фоккера-Планка, 
задающего эволюцию вероятностной массы в дифференциальной форме  $\rho_t(x)$:
\begin{equation}
    \frac{\partial \rho_t}{\partial t} = \text{div}(\nabla \Phi(x) \rho_t) + \beta^{-1} \Delta \rho_t.
\end{equation}

Для естественной работы в энергетических постановках водится функционал, задающий коэффициента сноса с потенциалом $\Phi$.
Таким образом, исходное уравнение можно переписать в вариационной постановке \ref{variation_fp}.

\textit{Определение:} \textbf{Функционал Фоккера-Планка} для распределения $\rho$ записывается как: 
\begin{equation}
    \mathcal{F}_{FP}(\rho) = \int  \Phi(x) d\rho(x) + \beta^{-1} \int \log \rho(x) d \rho(x).
\end{equation}
\begin{figure}[h]
    \centering
    \includegraphics[width=0.5\textwidth]{assets/math/transport/functional.excalidraw.png}
    \caption{Визуализация постановки уравнения Фоккера-Планка в вариационной форме}
    \label{variation_fp}
\end{figure}
Научная группа Йордана-Кинана-Отто в работе \cite{jordan1998variational} показала, что маргинальные вероятностные
меры процесса Ланжевена подчиняются уравнению градиентного потока Вассерштейна относительно функционала Фоккера-Планка.

\textit{Определение:}  \textbf{Схема Йордана-Кинана-Отто} задает правило обновления уравнения вероятности в виде
минимизации функционала энергии и расстояния:
\begin{equation}
    \rho^{n+1} = \underset{\rho}{\operatorname{argmin}} \left( \frac{1}{2\tau} W_2^2(\rho, \rho^n) + \mathcal{F}(\rho) \right),
\end{equation}
где:\begin{itemize}
    \item \(\tau > 0\) --- шаг по времени;
    \item \(W_2(\rho, \rho^n)\) --- метрика Вассерштейна 2-го порядка между плотностями \(\rho\) и \(\rho^n\);
    \item  \(\mathcal{F}(\rho)\) --- функционал свободной энергии, который может включать в себя энтропийный член и 
    потенциальную энергию системы.
\end{itemize}
Функционал свободной энергии\(\mathcal{F}(\rho)\) задается в виде:
\begin{equation}
    \mathcal{F}(\rho) = \int_V f(\rho(x)) \, dx + \int_V V(x) \rho(x) \, dx,
\end{equation}
где \(f(\rho)\) — внутренний энергетический член, зависящий от плотности, а \(V(x)\) — внешний потенциал.





\section{Компьютерное зрение}

Порождающие модели современное и быстро развивающие направление работы с данными, направленное на их
создание и получение вероятностной массы. Ключевыми достижениями в дисциплине были \begin{enumerate}
    \item порождающие грамматики \cite{chomsky2002syntactic}
    \item графические вероятностные модели \cite{pearl1988probabilistic}
    \item состязательные порождающие модели \cite{goodfellow2020generative}
    \item диффузионные порождающие модели \cite{song2020score}
\end{enumerate}
Такие модели также представляют интерес для предметных исследователей, поскольку позволяют аналитически изучать
элементарные механизмы задания графа.

Порождающие модели задают совместное распределение наблюдаемого объекта $x$ и его черт $y$ -  $p(x,y)$. В этом заключается 
ключевое различие между порождающими и дискриминирующими моделями $p(y|x)$ \ref{discr_vs_gen}.

\begin{figure}[h]
    \centering
    \includegraphics[width=0.5\textwidth]{assets/ml/generation/stable_diffusion.png}
    \caption{Архитектура современной модели Stable Diffusion}
    \label{discr_vs_gen}
\end{figure}

Порождающие модели, используют параметрические модели $p_\theta$ для аппроксимации истинных функций распределений на наборе обучающих данных.
Выбор параметрической функции аппроксимации, как правило, зависит от числа примеров в коллекции данных. Для больших наборов данных как правило используют нейросети.
Простейшим видом порождающей модели является авторегрессионная модель модель, использующая предшествующий контекст для предсказания следующего элемента.

\textit{Определение } \textbf{Авторегрессионные модели} представляют собой класс порождающих моделей,
с вычислимой  вероятностью, выполняющие генерацию через цепочку последовательных преобразований \begin{equation}
    p(x^{(1)},\dots,x^{(t)}) = \prod_{t=1}^T p_\theta(x^{(t)}|x^{(1)},\dots,x^{(t-1)})
\end{equation}

Как правило, авторегресионные модели используются для генерации последовательностей, временных рядов и текста.
Класс плохо применим к данные не подлежащие однозначному упорядочиванию или с неравномерным шагом. 

Для оценки разницы между вероятностными распределениями используются \textbf{дивергенции} с набором правил \begin{enumerate}
    \item $\forall p,q \in M \rightarrow D(p,q) \ge 0$  
    \item $p=q \leftrightarrow D(p,q) = 0$
    \item $\forall p \rightarrow D(p,p+dp)$ положительно определенная квадратичная фоорма  
\end{enumerate}

В отличие  от метрики дивергенции не обязаны быть симметричны.ьНа практике как правило используются специальный класс $f$-дивергенций, задающихся
через матожидание.

\textit{Определение} $f$-дивергенцией называется выпуклая функция, удовлетворяющая равенству $f(1)=0$.
$$
    D_f{\pi \parallel \rho} = \mathrm E_{\rho(x)} f\left(\frac{\pi(x)}{\rho(x)}\right)
$$

Семейство $f$-дивергенций включает функции \begin{enumerate}
    \item дивергенция Кульбака-Лейбнера $u logu $
    \item обратная дивергенция Кульбака-Лейбнера $-ln u$
    \item дивергенция Йенсена-Шэннона  $\frac{1}{2}\left(u ln u - (u+1) ln(\frac{u+1}{2})\right)$
\end{enumerate}


Нижней вариационной оценкой называется техника максимизации подпирающей границы параметрического распределения $p(\mathbf{x},\mathbf{z})$ вторым $q(\mathbf{x},\mathbf{z})$,
где переменная $\mathbf{z}$ называется скрытой . В аналитической форме нижняя граница записывается как 
$$
    \mathcal{L}(\phi,\theta;x) = \mathbb{E}_{z \sim q_\phi(z|x)} \left[\ln \frac{p_{\theta}(x,z)}{q_{\phi}(z|x)}\right],
$$

Перепишем через KL-дивергенцию:
\begin{equation}
    \begin{aligned}
        & D_\text{KL}( q_\phi(\mathbf{z}\vert\mathbf{x}) \| p_\theta(\mathbf{z}\vert\mathbf{x}) ) & \\
        &=\int q_\phi(\mathbf{z} \vert \mathbf{x}) \log\frac{q_\phi(\mathbf{z} \vert \mathbf{x})}{p_\theta(\mathbf{z} \vert \mathbf{x})} d\mathbf{z} & \\
        &=\int q_\phi(\mathbf{z} \vert \mathbf{x}) \log\frac{q_\phi(\mathbf{z} \vert \mathbf{x})p_\theta(\mathbf{x})}{p_\theta(\mathbf{z}, \mathbf{x})} d\mathbf{z}\\
        &=\int q_\phi(\mathbf{z} \vert \mathbf{x}) \big( \log p_\theta(\mathbf{x}) + \log\frac{q_\phi(\mathbf{z} \vert \mathbf{x})}{p_\theta(\mathbf{z}, \mathbf{x})} \big) d\mathbf{z} & \\
        &=\log p_\theta(\mathbf{x}) + \int q_\phi(\mathbf{z} \vert \mathbf{x})\log\frac{q_\phi(\mathbf{z} \vert \mathbf{x})}{p_\theta(\mathbf{z}, \mathbf{x})} d\mathbf{z} \\
        &=\log p_\theta(\mathbf{x}) + \int q_\phi(\mathbf{z} \vert \mathbf{x})\log\frac{q_\phi(\mathbf{z} \vert \mathbf{x})}{p_\theta(\mathbf{x}\vert\mathbf{z})p_\theta(\mathbf{z})} d\mathbf{z} \\
        &=\log p_\theta(\mathbf{x}) + \mathbb{E}_{\mathbf{z}\sim q_\phi(\mathbf{z} \vert \mathbf{x})}[\log \frac{q_\phi(\mathbf{z} \vert \mathbf{x})}{p_\theta(\mathbf{z})} - \log p_\theta(\mathbf{x} \vert \mathbf{z})] &\\
        &=\log p_\theta(\mathbf{x}) + D_\text{KL}(q_\phi(\mathbf{z}\vert\mathbf{x}) \| p_\theta(\mathbf{z})) - \mathbb{E}_{\mathbf{z}\sim q_\phi(\mathbf{z}\vert\mathbf{x})}\log p_\theta(\mathbf{x}\vert\mathbf{z}) &
    \end{aligned}
\end{equation}

Следовательно:
\begin{equation}
    \log p_\theta(\mathbf{x}) - D_\text{KL}( q_\phi(\mathbf{z}\vert\mathbf{x}) \| p_\theta(\mathbf{z}\vert\mathbf{x}) ) = \mathbb{E}_{\mathbf{z}\sim q_\phi(\mathbf{z}\vert\mathbf{x})}\log p_\theta(\mathbf{x}\vert\mathbf{z}) - D_\text{KL}(q_\phi(\mathbf{z}\vert\mathbf{x}) \| p_\theta(\mathbf{z}))
\end{equation}
    
Таким образом, из неравенства Йенсена получаем: 
$$
    \ln p_\theta(x) \le \mathcal{L}(\phi,\theta;x)
$$
Базовым алгоритмом оптимизации вариационных моделей является EM-алгоритм, состоящий из последовательного обновления
скрытых представлений и максимизации правдоподобия с заданным параметрами.

\textit{Определение} \textbf{EM-алгоритм} - алгоритм для нахождения оценок
максимального правдоподобия параметров  вероятностных моделей с скрытыми переменными $\theta$.

Аналитически шаги алгоритма записываются как: \begin{itemize}
    \item расчет матожидания при заданном на шаге $t$ параметре $\theta^{(t)}$.
    Шаг обноez \begin{equation}
        Q(\theta| \theta^{(t)}) = \mathbb{E}_{\mathbf{Z} \sim p(\mathbf{Z}|X,\theta^{(t)})} \left[ \log p(\mathbf{X},\mathbf{Z}|\mathbf{\theta})\right]
    \end{equation}
    \item максимизации полученного выражения для нового шага $\mathbf{\theta}^{(t+1)}$: \begin{equation}
        \mathbf{\theta}^{(t+1)} = \text{arg} \max_{\theta} Q(\mathbf{\theta}|\theta^{(t)})
    \end{equation}  
\end{itemize}






\subsection{Оптимизация графа}




Байесовы сети применяются в причинно-следственном анализе.

Итоговая модель представляет собой статистическую модель явления, которую можно использовать для
управления и анализа системой. 
Описанный подход называется \textit{вариационным причиннно-следственным выводом}.

В теории причинно-следственного анализа введенную величину
свидетельство. (\texit{англ.} Evidence).

Максимизация свидетельства
позволяет выбирать модели согласно принципу бритву Оккама, исключая
параметры не вносящие существенного смысла для модели. Для расчета свидетельства необходимо выполнить маргинализацию по параметрам
модели $\int P(X, \theta) d\theta$. В общем случае задача принадлежит 
NP-классу сложности, рассчитывается за время, экспоненциально зависящее
от числа параметров. 

\texit{Определение} События $A$ и $B$ cчитаются независимыми в условиях:
\begin{equation}
    P(A \cup B) = P(A) \times P(B)
\end{equation}

На практике в системах подлинная независимость случайных величин $x$ и $y$
$x \perp y$ встречается не всегда. Чаще достижима условная независимость, наблюдаемая при
фиксации третьего фактора $z$.

\texit{Определение} События $A$ и $B$ cчитаются условно независимыми 
для заданного события $C$ в условиях:
\begin{equation}
    P(A \cup B |C) = P(A|C) \times P(B|C)
\end{equation}

\begin{figure}[h]
    \centering
    \includegraphics[width=0.5\textwidth]{assets/math/discrete/bayes_net.excalidraw.png}
    \caption{Посредник(\textit{англ.} mediator), общий предок(\textit{англ.} cofounder), 
 общий родственник \textit{англ.} collider) }
    \label{discr_vs_gen}
\end{figure}

Виды вариационного вывода можно разделить на три ключевых направлениях: \begin{itemize}
    \item прогнозирование
    \item обратный - объяснение причины на основании
\end{itemize}

Вывод выполняется путем маргинализации распределения по 



Многокомпонетный осложняется неоднозначной трактовкой исхода. 
Для анализа сложных систем предпочтителен однофакторный анализ, выполняющий
для его выполнения необходимо перекрыть потоки зависимостей (\textit{англ.} dependency flow) от
прочих переменных.

\texit{Определение} Интервенцией называется изменением 

\begin{figure}[h]
    \centering
    \includegraphics[width=0.5\textwidth]{assets/math/discrete/dep_flow.excalidraw.png}
    \caption{Интервенция}
    \label{discr_vs_gen}
\end{figure}





\texit{Определение} События $A$ и $B$ cчитаются независимыми в условиях:
\begin{equation}
    P(A \cup B) = P(A) \times P(B)
\end{equation}





\begin{figure}[h]
    \centering
    \includegraphics[width=0.5\textwidth]{assets/math/discrete/bayes_net.excalidraw.png}
    \caption{Посредник(\textit{англ.} mediator), общий предок(\textit{англ.} cofounder), 
 общий родственник \textit{англ.} collider) }
    \label{discr_vs_gen}
\end{figure}




Условная независимость позволяет 
$x \perp y$ встречается не всегда.




\subsection{Оптимизация графа}

\textit{Определение} Графической вероятностной моделью





Принципиально выполняется задача факторизации 


Пробалистические языки программирования.





\subsection{Оптимизация графа}

Задача оптимального транспорта (Optimal Transport)\cite{villani2009optimal} является одной из ключевых  
в области теории вероятностей и машинного обучения.
Она представляет собой проблему определения оптимального способа перемещения вероятностной массы из одной 
распределенной системы в другую с минимальными затратами или стоимостью. Формально задача состоит в составлении 
транспортного плана \( T: \mathcal{X} \rightarrow \mathcal{Y} \), 
который переводит распределение \( \mu \subset \mathcal{X}\) в распределение \( \nu \subset \mathcal{Y} \), минимизируя некоторую функцию стоимости. 
Функция стоимости $c: \mathcal{X} \times \mathcal{Y} \rightarrow \mathbb{R}$ обычно является мерой сходства между элементами из \( \mathcal{X} \) и \( \mathcal{Y} \), 
такой как квадрат расстояния. 

\textit{Определение} (Монже): \textbf{Оптимальный транспорт} по Монже вводится путем рассмотрения вероятностных 
распределений \( \mu \) и \( \nu \) на метрических пространствах \( \mathcal{X} \) и \( \mathcal{Y} \):
\begin{equation}
    \inf_{\gamma \in \Pi(\mu, \nu)} \int_{\mathcal{X} \times \mathcal{Y}} c(x,y) \, d\gamma(x,y),
\end{equation}
где \( \Pi(\mu, \nu) \) обозначает множество всех возможных совместных распределений 
\( \gamma \) на \( \mathcal{X} \times \mathcal{Y} \) с фиксированными маргинальными
распределениями \( \mu \) и \( \nu \), а \( c(x,y) \) — функция стоимости перевозки массы из \( x \) в \( y \).

\textit{Определение} (Канторович): \textbf{Оптимальный транспорт} по Канторовичу  вводится через потенциал $\phi$, 
который минимизирует функционал стоимости:
\begin{equation}
    \inf_{\phi} \left( \int_{\mathcal{X}} \phi(x) \, d\mu(x) + \int_{\mathcal{Y}} \psi(y) \, d\nu(y) \right),
\end{equation}
где \( \psi \) — обратная функция к \( \phi \). Таким образом, 
отображение \( T: \mathcal{X} \rightarrow \mathcal{Y} \) вытекает из градиента потенциала.

Заметим, что постановка Канторовича обобщает постановку Монже\ref{monge_vs_kantarovich}. В отличие от постановки Монже 
оптимальный транспорт по Канторовичу допускает распределение вероятностной массы в непрерывном случае.

\begin{figure}[h]
    \centering
    \includegraphics[width=0.7\textwidth]{assets/math/transport/optimal_transport.excalidraw.png}
    \caption{Различие в подходе по Монже и Кантаровичу. В постановке Канторовича задача релаксирует до 
    непрерывного распределения}
    \label{monge_vs_kantarovich}
\end{figure}

Итоговая стоимость оптимального транспортного плана называется метрикой Вассерштейна.

\textit{Определение:} Пусть \((X, d)\) --- метрическое пространство и \(P(X)\) --- множество всех вероятностных мер на \(X\). 
Для двух вероятностных мер \(\mu\) и \(\nu\) на \(X\) \textbf{метрика Вассерштейна} порядка \(p\), где \(p \geq 1\), определяется как:
\begin{equation}
    W_p(\mu, \nu) = \left( \inf_{\gamma \in \Gamma(\mu, \nu)} \int_{X \times X} d(x, y)^p \, d\gamma(x, y) \right)^{1/p},
\end{equation}
где \(\Gamma(\mu, \nu)\) — множество всех сопряжённых мер \(\gamma\) на \(X \times X\) с маргиналами \(\mu\) и \(\nu\).

Метрика имеет практическое применение для задач физики, биологии и машинного обучения, поскольку задает 
дифференцируемую разность между распределениями.

\textit{Определение:} \textbf{Метрическая производная} кривой $\rho_t,t \in [0,T]$ в вероятностном 
пространстве $\mathcal{P}_2(\mathbb{R}^N)$ запишется как:
\begin{equation}
    |\rho_t'| = \lim_{dt \rightarrow 0} \frac{\mathcal{W}_2(\rho_t, \rho_{t+dt})}{dt}
\end{equation}

Метод оптимального транспорта также активно применяется для анализа стохастических процессов. Базовой моделью, 
описывающей стохастическое движение с смещением, является процесс Ланжевена.

\textit{Определение:} Процесс Ланжевена называется случайнный процесс вида
\begin{equation}
    d X_t = - \nabla \Phi(x) dt + \sqrt{2 \beta^{-1}} d W_t,
\end{equation}
где $\Phi(X)$ --- потенциал задающий снос частицы, $\beta$ - масштаб блуждания и  $d W_t$ - процесс Винера.

\begin{figure}[h]
    \centering
    \includegraphics[width=0.5\textwidth]{assets/math/transport/fokker-plank.excalidraw.png}
    \caption{Эволюция вероятностной массы в уравнение Ланжевена}
    \label{opt_transport}
\end{figure}

Стохастическое усреднение процесса Ланжевена можно описать с помощью уравнения Колмогорова-Фоккера-Планка, 
задающего эволюцию вероятностной массы в дифференциальной форме  $\rho_t(x)$:
\begin{equation}
    \frac{\partial \rho_t}{\partial t} = \text{div}(\nabla \Phi(x) \rho_t) + \beta^{-1} \Delta \rho_t.
\end{equation}

Для естественной работы в энергетических постановках водится функционал, задающий коэффициента сноса с потенциалом $\Phi$.
Таким образом, исходное уравнение можно переписать в вариационной постановке \ref{variation_fp}.

\textit{Определение:} \textbf{Функционал Фоккера-Планка} для распределения $\rho$ записывается как: 
\begin{equation}
    \mathcal{F}_{FP}(\rho) = \int  \Phi(x) d\rho(x) + \beta^{-1} \int \log \rho(x) d \rho(x).
\end{equation}
\begin{figure}[h]
    \centering
    \includegraphics[width=0.5\textwidth]{assets/math/transport/functional.excalidraw.png}
    \caption{Визуализация постановки уравнения Фоккера-Планка в вариационной форме}
    \label{variation_fp}
\end{figure}
Научная группа Йордана-Кинана-Отто в работе \cite{jordan1998variational} показала, что маргинальные вероятностные
меры процесса Ланжевена подчиняются уравнению градиентного потока Вассерштейна относительно функционала Фоккера-Планка.

\textit{Определение:}  \textbf{Схема Йордана-Кинана-Отто} задает правило обновления уравнения вероятности в виде
минимизации функционала энергии и расстояния:
\begin{equation}
    \rho^{n+1} = \underset{\rho}{\operatorname{argmin}} \left( \frac{1}{2\tau} W_2^2(\rho, \rho^n) + \mathcal{F}(\rho) \right),
\end{equation}
где:\begin{itemize}
    \item \(\tau > 0\) --- шаг по времени;
    \item \(W_2(\rho, \rho^n)\) --- метрика Вассерштейна 2-го порядка между плотностями \(\rho\) и \(\rho^n\);
    \item  \(\mathcal{F}(\rho)\) --- функционал свободной энергии, который может включать в себя энтропийный член и 
    потенциальную энергию системы.
\end{itemize}
Функционал свободной энергии\(\mathcal{F}(\rho)\) задается в виде:
\begin{equation}
    \mathcal{F}(\rho) = \int_V f(\rho(x)) \, dx + \int_V V(x) \rho(x) \, dx,
\end{equation}
где \(f(\rho)\) — внутренний энергетический член, зависящий от плотности, а \(V(x)\) — внешний потенциал.






\subsection{Оптимизация графа}

Задача оптимального транспорта (Optimal Transport)\cite{villani2009optimal} является одной из ключевых  
в области теории вероятностей и машинного обучения.
Она представляет собой проблему определения оптимального способа перемещения вероятностной массы из одной 
распределенной системы в другую с минимальными затратами или стоимостью. Формально задача состоит в составлении 
транспортного плана \( T: \mathcal{X} \rightarrow \mathcal{Y} \), 
который переводит распределение \( \mu \subset \mathcal{X}\) в распределение \( \nu \subset \mathcal{Y} \), минимизируя некоторую функцию стоимости. 
Функция стоимости $c: \mathcal{X} \times \mathcal{Y} \rightarrow \mathbb{R}$ обычно является мерой сходства между элементами из \( \mathcal{X} \) и \( \mathcal{Y} \), 
такой как квадрат расстояния. 

\textit{Определение} (Монже): \textbf{Оптимальный транспорт} по Монже вводится путем рассмотрения вероятностных 
распределений \( \mu \) и \( \nu \) на метрических пространствах \( \mathcal{X} \) и \( \mathcal{Y} \):
\begin{equation}
    \inf_{\gamma \in \Pi(\mu, \nu)} \int_{\mathcal{X} \times \mathcal{Y}} c(x,y) \, d\gamma(x,y),
\end{equation}
где \( \Pi(\mu, \nu) \) обозначает множество всех возможных совместных распределений 
\( \gamma \) на \( \mathcal{X} \times \mathcal{Y} \) с фиксированными маргинальными
распределениями \( \mu \) и \( \nu \), а \( c(x,y) \) — функция стоимости перевозки массы из \( x \) в \( y \).

\textit{Определение} (Канторович): \textbf{Оптимальный транспорт} по Канторовичу  вводится через потенциал $\phi$, 
который минимизирует функционал стоимости:
\begin{equation}
    \inf_{\phi} \left( \int_{\mathcal{X}} \phi(x) \, d\mu(x) + \int_{\mathcal{Y}} \psi(y) \, d\nu(y) \right),
\end{equation}
где \( \psi \) — обратная функция к \( \phi \). Таким образом, 
отображение \( T: \mathcal{X} \rightarrow \mathcal{Y} \) вытекает из градиента потенциала.

Заметим, что постановка Канторовича обобщает постановку Монже\ref{monge_vs_kantarovich}. В отличие от постановки Монже 
оптимальный транспорт по Канторовичу допускает распределение вероятностной массы в непрерывном случае.

\begin{figure}[h]
    \centering
    \includegraphics[width=0.7\textwidth]{assets/math/transport/optimal_transport.excalidraw.png}
    \caption{Различие в подходе по Монже и Кантаровичу. В постановке Канторовича задача релаксирует до 
    непрерывного распределения}
    \label{monge_vs_kantarovich}
\end{figure}

Итоговая стоимость оптимального транспортного плана называется метрикой Вассерштейна.

\textit{Определение:} Пусть \((X, d)\) --- метрическое пространство и \(P(X)\) --- множество всех вероятностных мер на \(X\). 
Для двух вероятностных мер \(\mu\) и \(\nu\) на \(X\) \textbf{метрика Вассерштейна} порядка \(p\), где \(p \geq 1\), определяется как:
\begin{equation}
    W_p(\mu, \nu) = \left( \inf_{\gamma \in \Gamma(\mu, \nu)} \int_{X \times X} d(x, y)^p \, d\gamma(x, y) \right)^{1/p},
\end{equation}
где \(\Gamma(\mu, \nu)\) — множество всех сопряжённых мер \(\gamma\) на \(X \times X\) с маргиналами \(\mu\) и \(\nu\).

Метрика имеет практическое применение для задач физики, биологии и машинного обучения, поскольку задает 
дифференцируемую разность между распределениями.

\textit{Определение:} \textbf{Метрическая производная} кривой $\rho_t,t \in [0,T]$ в вероятностном 
пространстве $\mathcal{P}_2(\mathbb{R}^N)$ запишется как:
\begin{equation}
    |\rho_t'| = \lim_{dt \rightarrow 0} \frac{\mathcal{W}_2(\rho_t, \rho_{t+dt})}{dt}
\end{equation}

Метод оптимального транспорта также активно применяется для анализа стохастических процессов. Базовой моделью, 
описывающей стохастическое движение с смещением, является процесс Ланжевена.

\textit{Определение:} Процесс Ланжевена называется случайнный процесс вида
\begin{equation}
    d X_t = - \nabla \Phi(x) dt + \sqrt{2 \beta^{-1}} d W_t,
\end{equation}
где $\Phi(X)$ --- потенциал задающий снос частицы, $\beta$ - масштаб блуждания и  $d W_t$ - процесс Винера.

\begin{figure}[h]
    \centering
    \includegraphics[width=0.5\textwidth]{assets/math/transport/fokker-plank.excalidraw.png}
    \caption{Эволюция вероятностной массы в уравнение Ланжевена}
    \label{opt_transport}
\end{figure}

Стохастическое усреднение процесса Ланжевена можно описать с помощью уравнения Колмогорова-Фоккера-Планка, 
задающего эволюцию вероятностной массы в дифференциальной форме  $\rho_t(x)$:
\begin{equation}
    \frac{\partial \rho_t}{\partial t} = \text{div}(\nabla \Phi(x) \rho_t) + \beta^{-1} \Delta \rho_t.
\end{equation}

Для естественной работы в энергетических постановках водится функционал, задающий коэффициента сноса с потенциалом $\Phi$.
Таким образом, исходное уравнение можно переписать в вариационной постановке \ref{variation_fp}.

\textit{Определение:} \textbf{Функционал Фоккера-Планка} для распределения $\rho$ записывается как: 
\begin{equation}
    \mathcal{F}_{FP}(\rho) = \int  \Phi(x) d\rho(x) + \beta^{-1} \int \log \rho(x) d \rho(x).
\end{equation}
\begin{figure}[h]
    \centering
    \includegraphics[width=0.5\textwidth]{assets/math/transport/functional.excalidraw.png}
    \caption{Визуализация постановки уравнения Фоккера-Планка в вариационной форме}
    \label{variation_fp}
\end{figure}
Научная группа Йордана-Кинана-Отто в работе \cite{jordan1998variational} показала, что маргинальные вероятностные
меры процесса Ланжевена подчиняются уравнению градиентного потока Вассерштейна относительно функционала Фоккера-Планка.

\textit{Определение:}  \textbf{Схема Йордана-Кинана-Отто} задает правило обновления уравнения вероятности в виде
минимизации функционала энергии и расстояния:
\begin{equation}
    \rho^{n+1} = \underset{\rho}{\operatorname{argmin}} \left( \frac{1}{2\tau} W_2^2(\rho, \rho^n) + \mathcal{F}(\rho) \right),
\end{equation}
где:\begin{itemize}
    \item \(\tau > 0\) --- шаг по времени;
    \item \(W_2(\rho, \rho^n)\) --- метрика Вассерштейна 2-го порядка между плотностями \(\rho\) и \(\rho^n\);
    \item  \(\mathcal{F}(\rho)\) --- функционал свободной энергии, который может включать в себя энтропийный член и 
    потенциальную энергию системы.
\end{itemize}
Функционал свободной энергии\(\mathcal{F}(\rho)\) задается в виде:
\begin{equation}
    \mathcal{F}(\rho) = \int_V f(\rho(x)) \, dx + \int_V V(x) \rho(x) \, dx,
\end{equation}
где \(f(\rho)\) — внутренний энергетический член, зависящий от плотности, а \(V(x)\) — внешний потенциал.





\subsection{Оптимизация графа}

Задача оптимального транспорта (Optimal Transport)\cite{villani2009optimal} является одной из ключевых  
в области теории вероятностей и машинного обучения.
Она представляет собой проблему определения оптимального способа перемещения вероятностной массы из одной 
распределенной системы в другую с минимальными затратами или стоимостью. Формально задача состоит в составлении 
транспортного плана \( T: \mathcal{X} \rightarrow \mathcal{Y} \), 
который переводит распределение \( \mu \subset \mathcal{X}\) в распределение \( \nu \subset \mathcal{Y} \), минимизируя некоторую функцию стоимости. 
Функция стоимости $c: \mathcal{X} \times \mathcal{Y} \rightarrow \mathbb{R}$ обычно является мерой сходства между элементами из \( \mathcal{X} \) и \( \mathcal{Y} \), 
такой как квадрат расстояния. 

\textit{Определение} (Монже): \textbf{Оптимальный транспорт} по Монже вводится путем рассмотрения вероятностных 
распределений \( \mu \) и \( \nu \) на метрических пространствах \( \mathcal{X} \) и \( \mathcal{Y} \):
\begin{equation}
    \inf_{\gamma \in \Pi(\mu, \nu)} \int_{\mathcal{X} \times \mathcal{Y}} c(x,y) \, d\gamma(x,y),
\end{equation}
где \( \Pi(\mu, \nu) \) обозначает множество всех возможных совместных распределений 
\( \gamma \) на \( \mathcal{X} \times \mathcal{Y} \) с фиксированными маргинальными
распределениями \( \mu \) и \( \nu \), а \( c(x,y) \) — функция стоимости перевозки массы из \( x \) в \( y \).

\textit{Определение} (Канторович): \textbf{Оптимальный транспорт} по Канторовичу  вводится через потенциал $\phi$, 
который минимизирует функционал стоимости:
\begin{equation}
    \inf_{\phi} \left( \int_{\mathcal{X}} \phi(x) \, d\mu(x) + \int_{\mathcal{Y}} \psi(y) \, d\nu(y) \right),
\end{equation}
где \( \psi \) — обратная функция к \( \phi \). Таким образом, 
отображение \( T: \mathcal{X} \rightarrow \mathcal{Y} \) вытекает из градиента потенциала.

Заметим, что постановка Канторовича обобщает постановку Монже\ref{monge_vs_kantarovich}. В отличие от постановки Монже 
оптимальный транспорт по Канторовичу допускает распределение вероятностной массы в непрерывном случае.

\begin{figure}[h]
    \centering
    \includegraphics[width=0.7\textwidth]{assets/math/transport/optimal_transport.excalidraw.png}
    \caption{Различие в подходе по Монже и Кантаровичу. В постановке Канторовича задача релаксирует до 
    непрерывного распределения}
    \label{monge_vs_kantarovich}
\end{figure}

Итоговая стоимость оптимального транспортного плана называется метрикой Вассерштейна.

\textit{Определение:} Пусть \((X, d)\) --- метрическое пространство и \(P(X)\) --- множество всех вероятностных мер на \(X\). 
Для двух вероятностных мер \(\mu\) и \(\nu\) на \(X\) \textbf{метрика Вассерштейна} порядка \(p\), где \(p \geq 1\), определяется как:
\begin{equation}
    W_p(\mu, \nu) = \left( \inf_{\gamma \in \Gamma(\mu, \nu)} \int_{X \times X} d(x, y)^p \, d\gamma(x, y) \right)^{1/p},
\end{equation}
где \(\Gamma(\mu, \nu)\) — множество всех сопряжённых мер \(\gamma\) на \(X \times X\) с маргиналами \(\mu\) и \(\nu\).

Метрика имеет практическое применение для задач физики, биологии и машинного обучения, поскольку задает 
дифференцируемую разность между распределениями.

\textit{Определение:} \textbf{Метрическая производная} кривой $\rho_t,t \in [0,T]$ в вероятностном 
пространстве $\mathcal{P}_2(\mathbb{R}^N)$ запишется как:
\begin{equation}
    |\rho_t'| = \lim_{dt \rightarrow 0} \frac{\mathcal{W}_2(\rho_t, \rho_{t+dt})}{dt}
\end{equation}

Метод оптимального транспорта также активно применяется для анализа стохастических процессов. Базовой моделью, 
описывающей стохастическое движение с смещением, является процесс Ланжевена.

\textit{Определение:} Процесс Ланжевена называется случайнный процесс вида
\begin{equation}
    d X_t = - \nabla \Phi(x) dt + \sqrt{2 \beta^{-1}} d W_t,
\end{equation}
где $\Phi(X)$ --- потенциал задающий снос частицы, $\beta$ - масштаб блуждания и  $d W_t$ - процесс Винера.

\begin{figure}[h]
    \centering
    \includegraphics[width=0.5\textwidth]{assets/math/transport/fokker-plank.excalidraw.png}
    \caption{Эволюция вероятностной массы в уравнение Ланжевена}
    \label{opt_transport}
\end{figure}

Стохастическое усреднение процесса Ланжевена можно описать с помощью уравнения Колмогорова-Фоккера-Планка, 
задающего эволюцию вероятностной массы в дифференциальной форме  $\rho_t(x)$:
\begin{equation}
    \frac{\partial \rho_t}{\partial t} = \text{div}(\nabla \Phi(x) \rho_t) + \beta^{-1} \Delta \rho_t.
\end{equation}

Для естественной работы в энергетических постановках водится функционал, задающий коэффициента сноса с потенциалом $\Phi$.
Таким образом, исходное уравнение можно переписать в вариационной постановке \ref{variation_fp}.

\textit{Определение:} \textbf{Функционал Фоккера-Планка} для распределения $\rho$ записывается как: 
\begin{equation}
    \mathcal{F}_{FP}(\rho) = \int  \Phi(x) d\rho(x) + \beta^{-1} \int \log \rho(x) d \rho(x).
\end{equation}
\begin{figure}[h]
    \centering
    \includegraphics[width=0.5\textwidth]{assets/math/transport/functional.excalidraw.png}
    \caption{Визуализация постановки уравнения Фоккера-Планка в вариационной форме}
    \label{variation_fp}
\end{figure}
Научная группа Йордана-Кинана-Отто в работе \cite{jordan1998variational} показала, что маргинальные вероятностные
меры процесса Ланжевена подчиняются уравнению градиентного потока Вассерштейна относительно функционала Фоккера-Планка.

\textit{Определение:}  \textbf{Схема Йордана-Кинана-Отто} задает правило обновления уравнения вероятности в виде
минимизации функционала энергии и расстояния:
\begin{equation}
    \rho^{n+1} = \underset{\rho}{\operatorname{argmin}} \left( \frac{1}{2\tau} W_2^2(\rho, \rho^n) + \mathcal{F}(\rho) \right),
\end{equation}
где:\begin{itemize}
    \item \(\tau > 0\) --- шаг по времени;
    \item \(W_2(\rho, \rho^n)\) --- метрика Вассерштейна 2-го порядка между плотностями \(\rho\) и \(\rho^n\);
    \item  \(\mathcal{F}(\rho)\) --- функционал свободной энергии, который может включать в себя энтропийный член и 
    потенциальную энергию системы.
\end{itemize}
Функционал свободной энергии\(\mathcal{F}(\rho)\) задается в виде:
\begin{equation}
    \mathcal{F}(\rho) = \int_V f(\rho(x)) \, dx + \int_V V(x) \rho(x) \, dx,
\end{equation}
где \(f(\rho)\) — внутренний энергетический член, зависящий от плотности, а \(V(x)\) — внешний потенциал.





