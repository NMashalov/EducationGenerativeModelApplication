
\texit{Определение} \textbf{Токенизация} процесс в котором текст разбивается на токены. 
Это позволяет применить лемматизацию к каждому слову в тексте независимо от контекста.

Токенизация позволяет характерный порядок слова

 Лемматизация часто используется в различных областях NLP, включая информационный поиск, анализ тональности, машинный перевод и другие.

\textit{Определение } \textbf{Векторное представление} (Вложение или Эмбеддинг)

Практически востребованной оказалась дистрибутивная гипотеза \cite{Schutze},
легшая в основу алгоритма \cite{NIPS2013_9aa42b31}.

В генеративном моделировании естественного языка, встает задача представления слов в виде векторов в многомерном пространстве, что позволяет моделировать семантические и синтаксические аспекты текста в компактной форме. Это представление, известное как "векторное вложение" или "embedding", позволяет выразить смысловые и лингвистические свойства слов, используемых в языке.

Формально, векторное представление \( \mathbf{e}_w \) слова \( w \) представляет собой векторное представление этого слова в многомерном пространстве:

$$

\[ \mathbf{e}_w = (e_{w1}, e_{w2}, ..., e_{wd}) \]

$$

где $d$ - размерность пространства вложения (число измерений), \( e_{wj} \) - \( j \)-ая компонента вектора вложения \( \mathbf{e}_w \).

Эти векторные представления обычно изучаются и извлекаются из больших корпусов текстов с использованием различных алгоритмов, таких как word2vec, GloVe (Global Vectors for Word Representation), FastText и другие. Они обладают свойством сохранения семантической близости слов в пространстве вложения: слова, которые часто встречаются в похожих контекстах, имеют близкие векторные представления.

Векторные вложения слов играют важную роль в генеративном моделировании естественного языка, так как они позволяют моделям представлять слова в виде непрерывных числовых значений, которые могут быть использованы как входные данные для алгоритмов машинного обучения. Это позволяет моделям эффективно изучать зависимости между словами и генерировать тексты семантически богатые и лингвистически осмысленные.






