Генеративное моделирование в сфере образования активно развивается и является одним из актуальных направлений научной работы. 
С 2019 года UNESCO регулярно публикует аналитические доклады \cite{annuvs2024education}\cite{unesco2019beijing},
содержащие обзоры применения искусственного интеллекта в образовании. Одним из наиболее востребованных направлений является адаптация
интеллектуальных ассистентов для персонального обучения. Доступные технологии распределенных вычислений и обработки больших данных обусловили 
создание больших языковых моделей, возможности которых в области работы с естественным языком приближаются к человеческим. Такие модели уже
широко используются в практических задачах, таких как программирование, анализ предметных корпусов и преобразования экспериментальных данных \cite{llamatouvron2023}. 
Коммерческий сектор также активно использует возможности языковых моделей и принимает участие в их развитии, предоставляя корпусы данных и обученные модели.
Модели успешно справляются с задачами перевода, выделения ключевых слов и пересказа. Эти возможности используются для динамического формирования описаний продуктов с 
учетом интересов потребителей, в чат-ботах служб поддержки и оптимизации информационной выдачи.

Целью данной работы является исследование возможностей больших языковых моделей в контексте задач образования. Область исследования
включает анализ преимуществ подхода с использованием языковых моделей и методов компенсации его недостатков. Исходя из цели были поставлены задачи:
\begin{itemize}
    \item подготовка корпуса образовательных данных на русском языке;
    \item организация среды эффективной эксплуатации большой языковой модели;
    \item разработка адаптивного алгоритма подбора сложности заданий для обучаемых.
\end{itemize}

Для их решения автор проводит предметный обзор актуальных подходов к развитию образовательных технологий и
наиболее значимых достижений в области современного машинного обучения. Работа содержит 4 главы.

В первой главе работы представлено теоретическое описание аппарата оптимизации и статистического моделирования. 
Рассмотрены методы градиентного стохастического спуска, обучения графовых вероятностных моделей и решения транспортных задач. 
Выполнен анализ ключевых теорем градиентного спуска и стохастической аппроксимации. 

Во второй главе  рассмотрены современные подходы генеративного моделирования. Описаны техники 
генерации, включающие вариационный автокодировщик, нормализационные потоки и диффузионные модели. Приведены 
методы численно эффективной адаптации моделей  с ключевыми теоретическими выкладками.

Третья глава посвящена применению психометрии в системе образования. Описаны методы 
оценки знаний учащихся с помощью байесовых систем тестирования и подходы теории игр к построению системы рейтинга 
с использованием экономических механизмов. Включены описания этапов детского развития, разработанные психологами Жаном Пиаже \cite{piaget1952origins} и Львом Выготским \cite{выготский2014мышление}.
Описана качественная модель выбора сложности заданий согласно Михею Чиксентмихай \cite{chen2007flow}.

В четвертой главе описан порядок исследований и вывод ключевых теорем работы. Приведено описание подхода к сбору данных из открытых источников и количественное
описание полученных корпусов. Рассмотрен вариант усовершенствования системы оптического распознавания символов на русском языке. Выполнен синтез 
оптимального алгоритма спуска к заданному параметру бернуллевского распределения с откликом в виде логистической функции. 

В заключении работы выполнен анализ итогов и намечены перспективы дальнейших исследований, включающие подход с использованием
 стохастической аппроксимации.

В качестве ключевых результатов работы можно выделить: \begin{itemize}
    \item разработанный алгоритм адаптации сложности заданий $d$, обеспечивающий оптимальную сходимость вероятности решения задачи $s$ к методически рекомендованной $s^*$;
    \item интеграция большой языковой модели в систему адаптивного подбора сложности заданий.
\end{itemize}

Апробация работы была выполнена на конференции МФТИ в форме двух докладов: \begin{itemize}
    \item "Оценка влияния кредитных условий на конкурентные предложения малых поставщиков в сфере образования";
    \item "Разработка пакетного модуля ShuemacherOCR на языке Python для работы с методической литературой".
\end{itemize}
