Генеративный моделирование в сфере образования становится актуальным направлением научной работы. 
Начиная с 2019 года всемирное общество UNESCO регулярно публикует аналитические доклады \cite{unesco2019beijing}\cite{annuvs2024education},
посвященные обзору применений искусственного интеллекта в всемирном образовании. Авторы приходят к выводу, что
наибольшую актуальность имеют современные технологии интеллектуальных ассистентов в случае адаптации их способностей
для формирования образовательной траектории.

Цель данной работы посвящена поиску актуальных направлений генеративного моделирования при формировании индивидуальной образовательной траектории.
Исходя из нее поставлены задачи:
\begin{itemize}
    \item изучить педагогическую ценность 
    \item подобрать проект системы интеллектуального ассистента согласно современным техническим возможностям
    \item подготовить примеры, демонстрирующие потенциал применения ассистента 
\end{itemize}

Для выполнения задач поэтапно автор проводит предметный обзор текущих подходов к формированию образовательных технологий и
современного направления машинного обучения. Внимание обращено к самым актуальным прорывам в сфере машинного обучения.


Апробация работы была выполнена на конференции МФТИ в ходе двух докладов \begin{itemize}
    \item Оценка влияния кредитных условий на конкурентные предложения малых поставщиков в сфере образования
    \item Разработка пакетного модуля ShuemacherOCR на языке Python для работы с методической литературой    
\end{itemize}

Главы работы разделены на 5 основных частей \begin{enumerate}
    \item Введение
    \item Методы оптимизации
    \item 
    \item Выполненная работа
\end{enumerate}
у
