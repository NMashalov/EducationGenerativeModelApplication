Генеративный моделирование в сфере образования актуально и активно развивающееся направлением научной работы. 
Начиная с 2019 года всемирное общество UNESCO регулярно публикует аналитические доклады \cite{unesco2019beijing}\cite{annuvs2024education},
посвященные обзору применений искусственного интеллекта в всемирном образовании. Наибольшую актуальность имеют
направления развития интеллектуальных ассистентов для персонального обучения. Ключевыми задачами
в такой постановке являются выбор тематических текстов,
адаптация научного знания под потребности учащегося и 
создание образовательных материалов под уровень и интересы учащегося.

Развитие техник распределенных вычислений и обработки больших данных привели к 
созданию больших языковых моделей, приближающихся в навыках общения к человеку. Также Модели уже
повсеместно используются в практических постановках, помогая исследователям программировать, изучать предметные корпуса и 
расширять эрудицию в областях интереса. Также существенно и влияние на коммерческие компании.
Модели прекрасно справляются с задачами перевода, выделения ключевых слов и пересказа. Такие навыки имеют
большую ценность в составлении описания товаров под интересы покупателей, автоматический ответ клиентов банков и
выборе книги.


Цель данной работы исследовать применимость больших языковых моделей к постановкам образования. Выделить и изучить
ключевые преимущества подхода и предложить техники компенсации недостатков. Исходя из цели были поставлены задачи:
\begin{itemize}
    \item сбор корпуса образовательных данных на русском языке
    \item подготовка среды эффективной эксплуатации большой языковой модели
    \item разработка адаптивного алгоритма подбора сложности
\end{itemize}

Для выполнения задач поэтапно автор проводит предметный обзор текущих подходов к формированию образовательных технологий и
современного направления машинного обучения. Внимание обращено к самым актуальным прорывам в сфере машинного обучения. Главы работы разделены на 4 основных части, содержащие описание 

В первой части работы описан теоретический аппарат оптимизации и статистического моделирования. 
Описаны методы градиентного стохастического спуска, обучения графовых вероятностных моделей, постановка
оптимального транспорта. Приведены ключевые теоремы 

Во второй части работы приведены современные подходы генеративного моделирования. Описаны современные техники генерации


Третья часть работы посвящена математическому описанию постановок образования. Описаны методы модельного описания. \ref{pedagogic_chapter}


Ключевыми результатами работы можно выделить \begin{itemize}
    \item алгоритма обновления сложности $d$, обеспечивающего оптимальную сходимость вероятности решения задачи $s$ к методически рекомендованной $s^*$.
    \item совмещение большой языковой модели с системой адаптивного подбора сложности задач
\end{itemize}

Апробация работы была выполнена на конференции МФТИ в ходе двух докладов \begin{itemize}
    \item Оценка влияния кредитных условий на конкурентные предложения малых поставщиков в сфере образования
    \item Разработка пакетного модуля ShuemacherOCR на языке Python для работы с методической литературой    
\end{itemize}
