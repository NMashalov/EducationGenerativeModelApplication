В главе описан процесс решения поставленных в  диссертационной работе задач. Ключевым достижением работы является адаптация алгоритма Роббинса-Монро под отклик, 
заданный логистической функцией. В завершающей секции главы описана теорема, задающая в аналитическом виде коэффициенты численной схемы для оптимального 
спуска к заданному уровню попыток. 

Практическим итогом работы стало веб-приложение, организующее персональное обучение шахматам с поддержкой интеллектуального ассистента.
В разработанной системе большая языковая модель отвечает за общение с обучаемым и комментирование игры. Для поддержания интереса к игре алгоритм автоматически 
адаптируется под уровень игрока таким образом, чтобы доля побед была равна $50\%$. При необходимости пользователь 
может сообщить о потребности изменить уровень на выбранный. Приложение также имеет многоуровневую систему достижений, поощряющих 
нестандартную игру и настойчивость в обучении. 
В главе также приведено описание сбора и подготовки для обучения данных образовательной направленности из открытых источников с использованием современных
систем оптического распознания. 










