В главе описаны шаги адаптации большой языковой модели к задачам образования. Описан формат взаимодействия, 
способ обучения, алгоритм оценки знаний и задания сложности, необходимые для компенсации текущих недостатков больших языковых моделей.
Практическим итогом работы стало веб-приложение, организующее персональное обучение шахматам с поддержкой интеллектуального ассистента.
В разработанной системе большая языковая модель отвечает за общение с обучаемым и комментирование игры. Для поддержания интереса к игре алгоритм автоматически 
адаптируется под уровень игрока таким образом, чтобы доля побед была равна $50\%$. При необходимости пользователь 
может сообщить о потребности изменить уровень на более подходящий его предпочтениям. Приложение также имеет многоуровневую систему достижений, поощряющих 
нестандартную игру и настойчивость в обучении. 
В главе также приведено описание сбора и подготовки для обучения данных образовательной направленности из открытых источников с использованием современных
систем оптического распознания. 

Главным теоретическим результатом работы является адаптация алгоритма Роббинса-Монро под отклик, заданный логистической функцией.
В завершающей секции главы описана теорема, задающая в аналитическом виде коэффициенты численной схемы для оптимального 
спуска к заданному уровню попыток. 












