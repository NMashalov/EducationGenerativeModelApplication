Стохастическая аппроксимация --- эффективный метод поиска корней уравнения в условиях случайного несмещенного отклика. В работе
\cite{yazidi2020balanced} была предложена адаптация методов стохастической аппроксимации к задачам оптимального выбора 
сложности заданий. Постановка представляет тест как стохастический ряд вида $\{x\}_{t=0}$, каждый элемент которого является случайной бернуллевской величиной с параметром $s$. 
Для ввода управляющей переменной задается сложность задачи $d$, параметризующая, 
в совокупности с функцией отклика учащегося $f$, переменную $s_t = f(d)$.

Таким образом, задача алгоритма рассчитать функцию $f(d_{t+1}^t,{x}_{i=0}^t)$, обеспечивающую оптимальную сходимость $\lim_{t \rightarrow \infty} \rho(s(d_t),s^*) =1$ согласно с условиями:
 \begin{itemize}
    \item метрика $\rho(x,x') = (x-x')^2$ евклидова;
    \item предполагается наличие банка $W$, возвращающего задачу произвольной сложности $d$;
    \item функция отклика $f(d_t)$ ограничена числом $M$ и монотонно убывает.
\end{itemize}
Авторы выбрали правило обновления сложности согласно правилу:
\begin{equation}
    d_{t+1} = \Pi(d_t+\lambda (x(t) -s^*)),
    \label{yazidi}
\end{equation}
где функция $\Pi$ является ограничивающим оператором вида: \begin{equation}
    \Pi_H(d) = \left\{
        \begin{array}{ll}
            d,\ \text{прт}\ 1<d<0 \\
            1,\ d\ge 1\\
            0, \text{при} \ d \le 0\\
        \end{array}
    \right.
\end{equation}

В секции рассмотрена модификация заданного метода путем явного учета логистического вида отклика учащегося на задание. 

