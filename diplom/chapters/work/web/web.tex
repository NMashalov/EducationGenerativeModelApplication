Разработка сайта была выполнена согласно требованиям WAI-ARIA \cite{craig2009accessible}.
Компоненты веб-сайта имеют высокую контрастность и выраженные контуры, позволяющие выполнять навигацию слабовидящим людям. 
Также функционал приложения ограничено доступен и незрячими людям, использующим специальные приложения для аудио отображения
содержания сайта.

Веб-приложение доступно при подключение через браузер по адресу доменного имени \url{www.mathema-online.xyz}. 
Технологии криптографии обеспечивают безопасность соединения, выпущенные сертификаты доменного имени 
исключают возможность подмены имени.

Современные образовательные материалы и инструменты требуют разработки интерфейсов. 
Одним из ключевых направлений является гештальт-дизайн \cite{wertheimer1938laws}, описанный в работах Гештальт группы.

\texit{Определение} \textbf{Гештальт} - психологическая концепция,акцентирующая внимание на восприятии целостных структур, а не на отдельных элементах, из которых они состоят.

Макса Вертхаймера, Курта Коффки и Вольфганга Кёлера \сite{}. Теория описывает основные принципы общечеловеческого визуального восприятия: 
\begin{enumerate}
    \item близость -стимулы, расположенные рядом, имеют тенденцию восприниматься вместе
    \item схожесть - стимулы, схожие по размеру, очертаниям, цвету или форме, имеют тенденцию восприниматься вместе
    \item целостность -восприятие имеет тенденцию к упрощению и целостности
    \item замкнутость - отражает тенденцию завершать фигуру так, что она приобретает полную форму
    \item смежность - близость стимулов во времени и пространстве. Смежность может предопределять восприятие, когда одно событие вызывает другое),
    \item общая зона - принципы гештальта формируют наше повседневное восприятие наравне с научением и прошлым опытом. Предвосхищающие мысли и ожидания также активно руководят нашей интерпретацией ощущений).
\end{enumerate}


Сайт выполнен в парадигме клиент-серверной разработки. Такая архитектура
позволяет заменять интерфейс и программное обеспечение без необходимости в 

Интерфейс реализован с помощью популярной библиотеки React для языка программирования JavaScript \cite{ackenheimer2015introduction}.
Такой подход позволяет дескриптивно описывать элементы вебсайта, программно реагируя на взаимодействие пользователя. 
Ключевой особенностью подхода является возможность использовать открытые профессионально подготовленные интерфейсы 

При использовании данных из открытых источников используются ссылки согласно требования Гражданского кодекса
Российской федерации \cite{law1274} \cite{law1260}.
