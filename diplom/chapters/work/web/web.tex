Разработка сайта была выполнена согласно требованиям WAI-ARIA \cite{craig2009accessible}.
Компоненты веб-сайта имеют высокую контрастность и выраженные контуры, позволяющие выполнять навигацию слабовидящим людям. 
Также функционал приложения ограничено доступен и незрячими людям, использующим специальные приложения для аудио отображения
содержания сайта.

Веб-приложение доступно при подключение через браузер по адресу доменного имени \url{www.mathema-online.xyz}. 
Технологии криптографии обеспечивают безопасность соединения, выпущенные сертификаты доменного имени 
исключают возможность подмены имени.

Современные образовательные материалы и инструменты требуют разработки интерфейсов. 
Одним из ключевых направлений является гештальт-дизайн \cite{wertheimer1938laws}, описанный в работах Гештальт группы.

Сайт выполнен в парадигме клиент-серверной разработки. Такая архитектура
позволяет заменять интерфейс и программное обеспечение без необходимости в 

Интерфейс реализован с помощью популярной библиотеки React для языка программирования JavaScript \cite{ackenheimer2015introduction}.
Такой подход позволяет дескриптивно описывать элементы вебсайта, программно реагируя на взаимодействие пользователя. 
Ключевой особенностью подхода является возможность использовать открытые профессионально подготовленные интерфейсы 

При использовании данных из открытых источников используются ссылки согласно требования Гражданского кодекса
Российской федерации \cite{law1274} \cite{law1260}.
