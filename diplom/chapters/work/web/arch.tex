Разработка интеллектуального ассистента, как правило, долгосрочная и трудоемкая деятельность, требующая внимания к планированию
и организации. В секции описан 

\textit{Определение} \textbf{Интеллектуальные ассистенты} прикладное программное обеспечение, выполняющее задачи пользователя,
согласно командам на естественном языке. Как правило аccистенты разделяют по их применению:
\begin{itemize}
    \item \textbf{разговорный} ассистент не имеет заранее заданной задачи и общается с пользователей для поддержания беседы
    \item \textbf{деловой} ассистент направлен на решение конкретной задачи
\end{itemize}
Современные системы ассистента организованы модульно. Такой подход позволяет эффективно
организовать совместную работу. Модули разделяются по формальному исполнению.


\begin{figure}[h]
    \centering
    \includegraphics[width=0.5\textwidth]{assets/work/arch/modern_system.excalidraw.png}
    \caption{Модульная система позволяет эффективно организовывать командную работу}
    \label{modular}
\end{figure}




выполнен в парадигме клиент-серверной разработки. Такая архитектура
позволяет заменять интерфейс и программное обеспечение без необходимости в 

Таким образом, для успешной адаптации ассистента необходимы

Итоговым решением, стало  рекомендательную систему как систему долгосрочного планирования и ассистента как эмпатического посредника.





Для адаптации открытой реализации для работы с детьми необходимо создание формата и предмета коммуникации.

Бранные слова были исключены с помощью библиотеки PyMorphy \cite{Korobov2015morph}, выделяющую нормальную форму слова для сравнения
с опорным корпусом неэтичных слов.

\begin{figure}[h]
    \centering
    \includegraphics[width=0.5\textwidth]{assets/work/arch/detox.excalidraw.png}
    \caption{Модули PyMorphy \cite{Korobov2015morph} и CLIP позволяют исключить неэтичное общение и изображение }
    \label{detox}
\end{figure}

Байесовы рейтинговые модели задают правила обновления апостериорных представлений рейтинга.
Наиболее известным примером байесовой модели рейтинга является модель Брэдли-Терри и эквивалентная ей модель Эло.
Ключевым преимуществом байесовых систем является возможность пересчета рейтинга сразу после матча.
 Таким образом система приобретает адаптивность, важную в коммуникациях в настоящем времени.

 В такой постановке ассистент следит за эмоциональным откликом обучающегося на материал,
узнает ее причину и в случае несоразмерной уровню нагрузке сообщает управляющей системе о потребности изменения сложности.

Плюсами такого подхода является \begin{itemize}
    \item интерпретируемость
    \item сужение ассистента до строгой постановки поиска наилучших навыков общения исходя из A/B тестирования.
\end{itemize}



