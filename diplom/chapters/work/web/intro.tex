Современные интеллектуальные ассистенты являются многокомпонентными системами, организующими
среду взаимодействия, целевой сценарий диалога, персонализацию и приватность переписки. Конкретные компоненты
выбираются из целей и возможности команды разработчиков. Для исследовательских целей, как правило, используется
открытое программное обеспечение, требующее лишь незначительной адаптации под постановку. В этой секции 
будет описан подход к созданию цифрового ассистента для проведения исследования. 

Основой современных ассистентов является большая языковая модель,
обученная посредством техник оптимизации выполнять инструкции, описанные естественным языком. 
Модель прекрасно справляется с задачами коммуникации, придерживается делового этикета и демонстрирует эмпатические
внимание к проблемам пользователя \cite{jiang2023mistral}\cite{llamatouvron2023}. Языковая модель
также помогает в решению повседневных и деловых задач, студийно сокращая академические статьи и помогая составлять
планы проектов. Последние достижения также позволяют задавать вопросы по изображениям, что существенно облегчает
выбор одежды и мебели \cite{bai2023qwen}. Тем не менее современные ассистенты не способны 
к выполнению формальных логические операции: арифметического сложения, решения абстрактных логических задач, соблюдение 
правил стратегических игр.


\textit{Определение} \textbf{Интеллектуальные ассистенты} прикладное программное обеспечение, выполняющее задачи пользователя, согласно командам на естественном языке. 
Как правило чат-ботов разделяют по их применению:
\begin{itemize}
    \item \textbf{разговорный} ассистент не имеет заранее заданной задачи и общается с пользователей для поддержания беседы
    \item \textbf{деловой} ассистент направлен на решение конкретной задачи
\end{itemize}

В образование интеллектуальные ассистенты применяются для обучения русском языку \cite{аль2019интеллектуальный} 
и рисования поясняющих графиков \cite{bulusuautomated}. Примерами коммерческого
использования ассистентов в образовании являются компании Merlin Mind
и OpenAI Education. Ключевым преимуществом решений является адаптация к общеобразовательным программ стран, 
взаимодействие с интерактивной доской и проприетарно подготовленная база знания регулярно обновляющая предметными экспертами.
