В основе ассистента лежит большая языковая модель, обученная посредством техник оптимизации воспроизводить язык.

Успехи в области обработки и генерации естественного языка расширили
возможности виртуальных ассистентов в помощи по решению повседневных и деловых задач. На данный мо


\textit{Определение} \textbf{Интеллектуальные ассистенты} прикладное программное обеспечение, выполняющее задачи пользователя, согласно командам на естественном языке. 
Как правило чат-ботов разделяют по их применению:
\begin{itemize}
    \item \textbf{разговорный} ассистент не имеет заранее заданной задачи и общается с пользователей для поддержания беседы
    \item \textbf{деловой} ассистент направлен на решение конкретной задачи
\end{itemize}


Для адаптации ассистента к задачам необходимо разработать интерактивное приложение.




