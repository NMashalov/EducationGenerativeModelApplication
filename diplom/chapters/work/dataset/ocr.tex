Системы оптического распознавания позволяют включать в корпусы текста для обработки цифровые документы с неподготовленным текстовым слоем. 
Особенность данных в образовании состоит в том, что они представлены, в основном, в цифровых документах в цифровых документах, 
существенно различающихся версткой. Еще одной особенностью является обилие иллюстраций, графиков и формул.

\textit{Определение:} \textbf{Оптическое распознавание символов} (OCR) представляет собой процесс автоматического преобразования текста,
 представленного в виде изображения или сканированного документа, в текстовый формат.
 
Исходное изображение документа подвергается предварительной обработке, в частности выполняется удаление шума и коррекция искажений. 
Следующим этапом является сегментация изображения, то есть разделение его структуры на отдельные символы или группы символов.
Собственно ри помощи алгоритмов распознавания, включающих методы машинного обучения и компьютерного зрения, 
символы на изображении анализируются и сопоставляются с соответствующими символами из набора знаков. Этот этап включает в себя распознавание формы символов, их контекста и других характеристик, что позволяет определить, какие символы были изображены на сканированном документе.

В завершение, распознанные символы объединяются в слова, предложения и абзацы, формируя полноценный текстовый документ. 
Точность и эффективность процесса OCR зависят от качества изображения, используемых алгоритмов распознавания, а также от языка и структуры текста.

Существует множество открытых пакетов для выполнения OCR \begin{itemize}
    \item Nougat \cite{blecher2023nougat}
    \item Tesseract \cite{smith2007overview}
    \item LayoutParser  \cite{shen2021layoutparser}
\end{itemize}

К сожалению, доступные открытые решения либо не поддерживают русский язык, либо не предназначены для работы с формулами.
Для решения автора был разработан открытый программный пакет для Python ShuemacherOCR
, предназначенный для масштабного анализа русской естественно-научной методической литературы.
В состав пакета входят модули как для обработки отдельных изображений, 
так и полноценных документов, позволяющие, извлекать данные в структурированном виде. 
Для решения задачи пакет использует нейросетевые алгоритмы. 
Ключевой особенностью пакета  является возможность выделять в тексте на русском языке строчные математические формулы.
Установка выполняется из открытого реестра пакетов PyPI с помощью менеджера pip или
из репозитория \footnote{\url{https://github.com/NMashalov/SchumacherOCR}}. 

Данные задач были собраны из открытых педагогических источников \cite{libmipt}\cite{mathedu} 
c обязательным указанием при публикации ссылок на источники.

Распознание текста по изображению выполняется нейросетью архитектуры Nougat \cite{blecher2023nougat}.
Особенностью данной архитектуры является быстрая адаптация под новые виды данных и работа с целым изображением, 
без необходимости промежуточного поиска регионов с текстом. 

Обучение сети проводилось на корпусе препринтов статей  \cite{clement2019use},
переведенных на русский язык с помощью интеллектуального ассистента ChatGPT \cite{ouyang2022training}. 
Выбор был связан с возможностью сохранять оригинальную разметку TeX-документов. 

Для валидации результатов был разработан открытый датасет, 
позволяющий измерить качество распознавания \footnote{\url{https://huggingface.co/datasets/NMashalov/ru_educational_book_datasets}} .

Разметка для обучения проводилось с помощью обращения к сервису компании MathPix. Метрики качества  приведены в таблице 1 и сопоставимы с результатами оригинальной модели Nougat.

\begin{center}
    \begin{tabular}{||c c c||} 
     \hline
     \text{Параметр} & \text{Тренировочная выборка} & \text{Отложенная выборка} \\
     \hline\hline
     \text{BLEU} & 83.2 & 80.4  \\ 
     \hline
     \text{Edit distance} & 0.15 & 0.17 \\
     \hline
    \end{tabular}
\end{center}

Для обучения на полученных данных была использована нейронная сеть YOLO \cite{redmon2016you}. Эта архитектура нейронной сети имеет способность эффективно дообучаться на небольших выборках данных, что позволяет достигать удовлетворительных результатов.
Для ситуаций, где число аннотаций и число изображений на изображении не совпадало, применялся алгоритм на двудольном графе, направленный на максимизацию числа пар.
 
Для получения обучающей выборки была проведена разметка части датасета. Каждое изображение включает в себя текстовую информацию, а также различные чертежи и формулы, характерные для данной области знаний.

Процесс разметки включал создание аннотаций для каждого изображения, 
а именно выделение границ объектов, таких как текстовые блоки, формулы и чертежи. 
Этот процесс требовал точности и внимательности для корректного определения границ объектов на изображении и их соответствия с аннотациями.

Для расширения датасета и обеспечения его разнообразия была применена аугментация данных. 
Применялись повороты, масштабирование, изменение освещения и отражение, позволили создать дополнительные вариации входных данных. 
Это способствовало увеличению разнообразия обучающей выборки и повышению устойчивости модели к различным вариациям данных, что важно для обеспечения ее эффективности в реальных условиях различной разметки страницы.

\begin{figure}[h]
    \centering
    \includegraphics[width=0.5\textwidth]{assets/work/dataset/kirik_labeling.png}
    \caption{Пример аннотированной иллюстрации из книги Генденштейн, Кирик, Гельфгат: 1001 задача по физике}
    \label{annotation}
\end{figure}

Метрическая оценка результатов выделения иллюстрации и аннотации 

\begin{center}
    \begin{tabular}{||c c c||} 
     \hline
     \text{Параметр} & \text{Тренировочная выборка} & \text{Отложенная выборка} \\
     \hline\hline
     \text{mAp} & 78.4 & 75.4  \\ 
     \hline
     \text{Точность распознавания ребер  “изображение-аннотация”}  & 75.2 & 72.0 \\
     \hline
    \end{tabular}
\end{center}

Автор продолжает развитие пакета для среды распределенных вычислений, использующих акторную
модель для представления данных.

\begin{figure}[h]
    \centering
    \includegraphics[width=0.5\textwidth]{assets/work/dataset/saga.excalidraw.png}
    \caption{Итоговая разметка выполняется посредством распределенных вычислений}
    \label{saga}
\end{figure}
