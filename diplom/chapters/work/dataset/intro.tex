Прогресс в области машинного обучения и разработки интеллектуальных ассистентов ведет к росту потребности в
высококачественных корпусах текстов и аннотированных изображений. Обучение на предметном корпусе
позволяет улучшать количественные метрические показатели достоверности передаваемых знаний 
на десятки процентных пунктов \cite{tinn2023fine}. Тем не менее сбор данных для научных проектов из
потребности воспроизводимости эксперимента должен проходить из открытых источников. Таковыми, например,
могут быть книги из открытых источников, задачи из олимпиад и предметные вебсайты. Также 
сбор можно проводить из уже подготовленных научным сообществом корпусов, задавая экспертные критерии
отбора и выполняя перевод с иностранного языка. 

Сбор данных, представленных в виде электронных документов, не имеющих
подготовленного текствого слоя осуществлялся при помощи технологий оптического распознавания символов.
В описании секции также частично включен методы, разработанные в рамках данного исследования.

В состав открытого тренировочного набора входит более тысячи аннотированных изображений  
Результат моделирования предоставлены на открытых ресурсах\footnote{
\url{https://github.com/NMashalov/Generative-modeling-appliance-for-creating-educational-tasks}
и \url{https://huggingface.co/datasets/NMashalov/task_illustrations_dataset}
} с обязательным указанием источником данных.