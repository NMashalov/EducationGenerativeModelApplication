Прогресс в области машинного обучения и разработки интеллектуальных ассистентов ведет к росту потребности в
высококачественных корпусах текстов и аннотированных изображений. Обучение на предметном корпусе
позволяет улучшать количественные метрические показатели достоверности передаваемых знаний 
на десятки процентных пунктов \cite{tinn2023fine}. Тем не менее открытые корпусы русского языка 
\cite{hung2022multi2woz} \cite{dmitrieva2023automatic} \cite{ivanov2023new} почти не содержат образовательной тематики.
Представленный в секции корпус является вариантом решения проблемы недостатка данных в направлении образования. 
Результаты моделирования предоставлены на открытых ресурсах\footnote{
\url{https://github.com/NMashalov/Generative-modeling-appliance-for-creating-educational-tasks}
и \url{https://huggingface.co/datasets/NMashalov/task_illustrations_dataset}
} с указанием источников данных. Полный список источников приведен в приложении к работе \ref{}.

 
