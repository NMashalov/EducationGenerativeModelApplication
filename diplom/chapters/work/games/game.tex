Современные большие языковые модели пока не способны к полноценному ведению игры. В работе \cite{Adam2024} проведен анализ силы игры ассистента ChatGPT
путем сравнения с шахматным движком Stockfish. Статистические исследования показывают, что текущий уровень игры
модели соответствует рейтингу Эло 1600\cite{elo1967proposed}. Такой уровень соответствует начальном уровню игрока в шахматы. Исследователи также отмечают неспособность ассистента к строгому исполнению правил игры
и наблюдает противоречия в стратегии даже в краткосрочной перспективе.

Исходя из текущих возможностей был предложен гибридный подход, заключающийся в совмещение языковой модели с 
открытым шахматным движком StockFish \cite{acher2016large}. 
В такой постановке интеллектуальный ассистент отвечает на вопросы пользователя по ходу игры и
рассказывает о возможных стратегических решениях, исходя из ситуации на доске. 
Движок задает уровень оппонента и помогает в 

Обновление рейтинга выполняется согласно модели Эло, в зависимости от сложности оппонента по правилу:
\begin{equation}
    x_{i+1}=x_i + K(\beta) \left[X_i-P(X_i=1)\right]
\end{equation}
, где $P(X_i=1)$ задает вероятность победы в игре.

Алгоритм базовой игры может быть описан как последовательность шагов:
\begin{enumerate}
    \item пользователь начинает игру 
    \item ассистент по ходу игры , пользователь может  
    \item по результатам игры
\end{enumerate}

Для поощрения настойчивости и развития интереса к игре была разработана
система достижений, включающая как количественную оценку прогресса, так и выполнение нестандартных задач.
Правила получения награды определяются по порогу отсечки согласно алгоритму, описанному в \ref{badge}.
Такая система поощряет дух соревнования для достижений выдающихся результатов.

\begin{figure}[h]
    \centering
    \includegraphics[width=0.5\textwidth]{assets/work/games/achieve.png}
    \caption{Достижение на примере награды за победу в сеансе одновременной игры}
    \label{achievement}
\end{figure}

Разработки игры по рисованию была выполнена с использованием открытой диффузионной модели \ref{diffusion},
составляющей рисунок по текстовому запросу. Для простоты выполнения рисунка предложено выполнение мозаики из
пикселей, представляющих базовый элемент растровой сетки изображения. Адаптация базовой модели для стилизации
рисунка выполняется с помощью открытого низкорангового адаптера. Исходный запрос для модели должен быть 
сформулирован на английском языке, лаконично описывать объекты на изображении и обстановку. 
В силу случайности генерации пользователь может подобрать для себя наиболее интересный вариант изображения.
Сложность рисунка определяется из наличия фона, декорации и сложной композиции. 

Алгоритм выполнения рисунка состоит из трех последовательных этапов \begin{enumerate}
    \item пользователь вводит предмет интереса на русском языке
    \item ассистент переводит запрос на английский язык с учетом его рейтинга и предоставляет пользователю выбор из заданных рисунков
    \item кодировщик сранивает рисунок с исходным изображением и определяет балл для обновления рейтинга
\end{enumerate}

Модель дополнительно снабжена фильтром цензуры, позволяющей избегать неэтичного рисунка \cite{radford2021learning}.
Уровень сложности регулируется путем изменения композиции и наличия фона.
\begin{figure}[h]
    \centering
    \includegraphics[width=0.5\textwidth]{assets/work/games/difficulty.excalidraw.png}
    \caption{Сложность задания задается организацией рисунка}
    \label{draw}
\end{figure}
