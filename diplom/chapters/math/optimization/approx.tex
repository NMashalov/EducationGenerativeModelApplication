Для случая, в котором 
Стохастическая аппроксимация 


\textit{Определение} \textbf{Стохастическая аппроксимация} - метод решения задач статистического оценивания,
 строящихся в виде последовательного приближения на основании наблюдений, представленных случайной величиной.

Стохастическая аппроксимации

$$
    P(\mathbf{T}| \mathbf{X},\mathbf{\Theta})
$$

стохастической аппроксимации,
 изложенному в работе Герберта Робинсона и Суттона Монро \cite{robbins1951stochastic}

 Схема Поляка-Руперта-Юдицкого. 
\begin{equation}
    x^{k+1} = x^{k} - \gamma_k \phi(\nabla_x f(x^k,\xi^k))
\end{equation}
Шаги $h_k \sim k^{-\alpha}$, $\alpha in (\frac{1}{2},1)$.
При этом ошибка считается для среднего
\begin{equation}
    \bar{x_n} = \frac{1}{N} \sum_{k=1} x^k
\end{equation}

