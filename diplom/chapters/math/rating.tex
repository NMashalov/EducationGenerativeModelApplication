Модели рейтинга нужны для линейного упорядочивания ряда объектов, которые можно сравнить только попарно.

Согласно модели рейтинга Эло сила игрока задается случайной величиной $\xi$. \textit{Рейтинг} задается матожиданием силы $\mathcal{E} \xi$.
Согласно модели Эло сила игрока задается нормально, причем дисперсия $\sigma$ фиксирована для всех игроков
Тогда сила игры согласно предположению определяется как:

\begin{equation}
    p(x) = \frac{1}{\sigma \sqrt{2\pi}} \exp^{- \frac{1}{2\sigma^2}{(x-s)^2}}
\end{equation}
Таким образом, рейтинг является латентной переменной. В литературе также популярна модель Брэдли-Терри,задающая
вероятность победы зависит как:
\begin{equation}
    P(\theta)
\end{equation}
Заметим, что подмена $\gamma ~ exp(-\theta/\beta^2)$ позволяет отождествить подходы.



В шахматной практике волатильность считается определенной и имеет стандартное отклонение равное 20:

$$
    \frac{1}{1+10^\frac{R_B-R_A}{400}}
$$

Задача

\textit{}

Рейтинг Эло - подход к расчету силы игроков, основанный на предсказательной силе

Эло используется в образовании для адаптации системы тестирования \cite{}
- [](https://www.fi.muni.cz/~xpelanek/publications/CAE-elo.pdf) - использование в [адаптационном подходе](/bases/test) в образовании. 
- [](https://aclanthology.org/W19-4451.pdf)


### Теоретическое описание

Рейтинг строится по принципу потенциала. Нулевой уровень задается произвольно. Вероятность победы задается через разницу уровней игроков, считающимися явно необозримыми переменыыми:

$$
    P(R_{ij}=1) = \frac{1}{1 + e^{-(\theta_i - \theta_j)}}
$$

, где $\theta_i$ - задает рейтинга участника. Подход к оценки совпадает с подходом Item Response Theory случайности совпадает с IRT[/bases/test]. 







:::tip

    Экспоненциальный вид графика связан с предположением о том, что в стратегических играх существенное различие в навыке гарантирует победу

:::

По окончанию соревнования рейтинг каждого игрока обновляется

Обновление рейтинга выполняется по правилу

$$
    R^'_A = R_A
$$


Таким образом, в каждой игре.

### Примечательные свойство Эло


1. В отсутствии роста игрока его уровень не будет изменяться.

$$
    E_p(\theta) = p 
$$


2. Новый рейтинг задается 
 
```tip

    В модели Эло новый рейтинг

```


3.



## Система Глико

Модель Глико при присваивании балла также учитывает. http://www.glicko.net/glicko/glicko.pdf

:::note

    Модель Глико популярна в онлайн играх, поскольку пользователи могут играть нерегулярно

:::

Вводится переменная $RD$ отвечающая за волатильность.

$$
    
$$


##

$$

$$