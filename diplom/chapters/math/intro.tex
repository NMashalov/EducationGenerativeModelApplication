Математический аппарат искуственного интеллекта был  преимущественно описан в 60-е годы
Борисом Теодоровичем Поляком в теории оптимизации \cite{kantorovich1960mathematical}, 
Вапником Владимир Наумовичем и Алексеем Яковлевичем Червоненкисем \cite{вапник1974теория} в теории минимизации эмпирического риска
и Леонидом Кантаровичем в теории оптимального планирования \cite{kantorovich1960mathematical}.
Современные достижения во многом являются результатом
осмысления разработанного аппарата применительно к современному времени с
повсеместным распостранением мощных вычислительных средств.
наиболее удачной экспериментальной 

В секции будут описаны математические методы,
использующиеся для описания моделей  


