Математический аппарат искусственного интеллекта был предложен в 60-е годы прошлого века
Б. Т. Поляком в теории оптимизации \cite{kantorovich1960mathematical}, 
В. В. Наумовичем и А. Я. Червоненкисом \cite{вапник1974теория} в теории минимизации эмпирического риска
и Л. В. Кантаровичем в теории оптимального планирования \cite{kantorovich1960mathematical}.
Современные достижения во многом являются результатом адаптации разработанного теоретического аппарата к применению 
мощных вычислительных систем, ставших доступными последнее десятилетие.

В главе приведено описание математического аппарата оптимизации, включающие техники
градиентного спуска, стохастической аппроксимации, распределенного обучения и оптимального транспорта.

Также необходимо отметить важность этапа, предшествующего поиску и исполнению оптимального алгоритма оптимизации:
определение аналитической функции оптимизации, отвечающей потребностям предметной постановки. 
Такая функция должна быть скалярной величиной, но может зависеть от многих факторов. Функция должна быть "удобной" для 
оптимизации и оценки. Обычно используется монотонная функция с гладкими производными, при убывании которой 
система наблюдаемо изменяет свои свойства. Также важны ограничения на допустимость сочетания параметров, что 
позволяет снизить вероятность оптимизируемой системы в нежелательный режим.
