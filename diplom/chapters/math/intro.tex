Математический аппарат искусственного интеллекта был предложен в 60-е годы
Б. Т. Поляком в теории оптимизации \cite{kantorovich1960mathematical}, 
В. В. Наумовичем и А. Я. Червоненкисом \cite{вапник1974теория} в теории минимизации эмпирического риска
и Л. В. Кантаровичем в теории оптимального планирования \cite{kantorovich1960mathematical}.
Современные достижения во многом являются результатом адаптации разработанного аппарата к применению 
мощных вычислительных средств, ставших доступными последнее десятилетие.

В главе приведено описание математического аппарата оптимизации, включающие техники
градиентного спуска, оптимального транспорта.

Также отметим важность этапа, предшествующего поиску и исполнению оптимального алгоритма оптимизации:
определение аналитической функции оптимизации, отвечающей потребностям предметной постановки. 
Такая функция должна быть скалярной величиной, но может зависеть от многих факторов. Удобной для 
оптимизации и оценки ее результат будет монотонная функция с гладкими производными, при убывании которой 
система наблюдаемо изменяет свой вид. Также важно задавать требования на допустимые сочетания параметров, исключая возможность
перехода оптимизируемой системы к нежелательному режим
