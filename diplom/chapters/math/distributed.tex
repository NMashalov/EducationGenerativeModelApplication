Существенным препятствием распределенного обучения является 
проблема передачи данных и обновление представлений участников об общем состоянии системы.
В секции будут разобраны проблемы и пути решения, включающие ролевые

Виды распределенного обучения \begin{itemize}
    \item кластерное - единая организация с доверенными узлами
    \item коллаборативное - распределенняя организация
    \item федеративное - на устройствах пользователей  с применением
\end{itemize}

Виды компрессии 

\texit{Опредление} \textbf{Несмещенной компрессией} называется компрессия $\Pi$ со свойствами \begin{itemize}
    \item $\mathbb{E}[\Pi(x)] = x$
    \item $\mathbb{E}[\|\Pi(x)\|^2_2] \le \omega \| x \|^2_2$, где $\omega \ge 1$
\end{itemize}


\texit{Опредление} \textbf{Случайной спарсификацией} \cite{richtarik2016parallel}называется оператор, выбирающий из вектора компоненты по правилу
\begin{equation}
    Rankd(x) = \frac{d}{k} \sum_{i \in S} \[x\]_i e_i,
\end{equation}
где $i$ - случайно выбранные компоненты из базиса.

\texit{Опредление} \textbf{Случайной спарсификацией} \cite{alistarh2017qsgd} называется оператор, выбирающий из вектора компоненты по правилу

$\xi_i \sim Bern(\frac{|x_i|}{\|x\|_2})$  

\begin{equation}
    Rankd(x) = \frac{d}{k} \sum_{i \in S} \[x\]_i e_i,
\end{equation}



Достижение консенсуса выполняется через алгоритмы распределенных вычислений. 

\textit{Определение}\textbf{Консенсус} является результатом достижения согласованного состояния между несколькими независимыми 
процессами или узлами в системе, которые могут взаимодействовать друг с другом. 

Для достижения консесуса необходимо выполнить условия \begin{enumerate}
    \item корректности: $\forall i \in \{1, \ldots, n\}, \text{если } \text{input}(N_i) = v, \text{ то } \forall j \in \{1, \ldots, n\}, \text{output}(N_j) = v$.
     Все узлы начинают с одним и тем же начальным значением v, то любое значение, принятое в результате выполнения протокола консенсуса, должно быть равно \( v \).
    \item единогласие: $\forall i, j \in \{1, \ldots, n\}, \text{если } \text{output}(N_i) = v, \text{ то } \text{output}(N_j) = v$.
     Если один узел завершает протокол с некоторым значением \( v \), то все другие узлы, которые также завершили протокол, должны иметь то же самое значение \( v \).
    \item завершение:$\forall i \in \{1, \ldots, n\}, \text{ узел } N_i$ 
    завершает выполнение протокола в конечное время.
\end{enumerate}

Наиболее полулярными алгоритмами достижения конcенсуса являются Raft \cite{lamport2019time} и Paxos \cite{pease1980reaching}. Методы
предлагают разделение на роли.

