\texit{Определение} \textbf{Метод градиентного спуск} метод нахождения экстремума функции
посредством обновления с учетом градиента функции.
\begin{equation}
    x_{t+1} = x_t - f(\nabla L(x_t))  
\end{equation}


#
Известными развновидностями методами являются:\begin{enumerate}
    \item Метод Полякова:
        \begin{equation}
            \begin{aligned}
                &x_{t+1}= x_t + \mathbf{v}_t \\
                &v_t = \mu v_{t-1} - \eta \nabla L(x_t)
            \end{aligned}
        \end{equation}
    \item RMSProp - AdaGrad + exponential decay:
        \begin{equation}
            \begin{aligned}
                &x_{t+1} = x_t - \frac{\eta}{\sqrt{g_t+\epsilon}} \cdot \nabla L(x_t) \\
                &g_t = \mu g_{t-1} + (1-\mu)\nabla L(x_t) \cdot \nabla  
            \end{aligned}
        \end{equation}
    \item Adam \сite{kingma2014adam}:
        \begin{equation}
            \begin{aligned}
                &x_{t+1} = x_t - \frac{\eta}{\frac{\sqrt{g_t+\epsilon}}{1-\mu^t}} \cdot \frac{v_{t+1}}{1-\beta^t} \\
                &v_{t+1} = \beta v_t + (1-\beta) \nabla L(x_t) \\
                &g_t = \mu g_{t-1} + (1-\mu)\nabla L(x_t) \cdot \nabla  L(x_t)
            \end{aligned}
        \end{equation}
\end{enumerate}











