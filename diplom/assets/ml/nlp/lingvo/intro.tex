Педагогика использует лингвистические исследования для разработки методов обучения, 
таких как коммуникативный подход и логопедические метод, основанные на понимание языка.

Как правило, лингвисты сотрудничают с педагогами для создания учебных материалов, которые учитывают языковые особенности и когнитивные процессы усвоения языка.

В секции будут рассмотрена формальная вычислительная теория языка.
Подход был предложен Ноамом Хомски разработанной для разработки
в его работе "Синтаксические структуры" \cite{chomsky2002syntactic}. Направление изучает алгоритмические 
методы по изменению морфемного состава слова, формированию представления о связи слов в тексте.





